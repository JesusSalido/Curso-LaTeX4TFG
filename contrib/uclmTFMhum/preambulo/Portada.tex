% OJO: Editar este fichero a voluntad para ajustarlo al resultado deseado.
% OJO: Poner especial atención en usar los adjetivos de género correctos.
% -------------------------
% -------------------------
% -------------------------
% PORTADAS: (Incluidas páginas para créditos y dedicatoria)
% -------------------------
% -------------------------------------------------------------------------
% -------------------------------------------------------------------------
% -------------------------------------------------------------------------
%--- PORTADA TFM LETRAS
\begin{titlepage}
    \begin{center}
        {\includegraphics[width=3cm]{uclm3arcos}}\\[0.4cm]
        {\LARGE UNIVERSIDAD DE CASTILLA-LA MANCHA}
        \vspace{1cm}
        {\par\Large FACULTAD DE LETRAS} 
        \vspace{0.6cm}
    {\par\textsc{MÁSTER UNIVERSITARIO EN INVESTIGACIÓN\\ 
    EN LETRAS Y HUMANIDADES}}
	\vfill
    {\huge\textsc{Título del trabajo (español)}}\par
    \bigskip
    \parbox{.6\linewidth}{\dingline{167}}
    \bigskip
    \par{\Large\textsc{Título del trabajo (inglés)}}
    \par\vspace{2cm}
    \textsc{(Autor/a)}
    \medbreak
    Director/a: Prof. Nombre Apellidos (Director/a)\par
    Depto. del Director/a\par
    \vspace{4cm}
    \noindent%
    Ciudad real, fecha
    \end{center}
    \vfill
    \par\noindent\makebox[2.5in]{\hrulefill} \hfill\makebox[2.5in]{\hrulefill}%
    \par\noindent\makebox[2.5in]{Firma Autor/a}      
    \hfill\makebox[2.5in]{Firma Director/a}%
\end{titlepage}
%---

% OPT.: DEDICATORIA (1 pág. máximo) comentar si no se desea incluir.
% Aunque opcional, no se debería perder la oportunidad de poder 
% dedicar el trabajo a alguien MUY especial. Debe ocupar como mucho dos líneas
% (no confundir con los agradecimientos).
% EDITAR: Dedicatoria.

\null\vspace{\stretch{1}}
\begin{flushright}
\emph{A alguien muy especial\\
Por el motivo que lo hace especial}
\end{flushright}
\vspace{\stretch{2}}\null
\cleardoublepage