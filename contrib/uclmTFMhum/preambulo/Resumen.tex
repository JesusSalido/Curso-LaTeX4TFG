%--- Ajustes del documento.
\pagestyle{plain}	% Páginas sólo con numeración inferior al pie

% -------------------------
%
% RESUMEN:
% 
%

% EDITAR: Resumen (máx. 1 pág.)
%\cleardoublepage % Se incluye para modificar el contador de página antes de añadir bookmark
\phantomsection  % Necesario con hyperref
\addcontentsline{toc}{chapter}{Resumen} % Añade al TOC.
\selectlanguage{spanish} % Selección de idioma del resumen.
\makeatletter
\begin{center} %
   {\textsc{TRABAJO FIN DE MÁSTER - FACULTAD DE LETRAS (UCLM)}\par} % Tipo de trabajo
   \vspace{1cm} %  
   {\textbf{\Large Título en español}\par}  % Título del trabajo
   \vspace{0.4cm} %
   {\@autor \\ \@cityTF,{} \@mesTF{} \@yearTF\par} 
   \vspace{0.9cm} %
   {\textbf{\large\textsf{Resumen}}\par} % Título de resumen
\end{center}   
\makeatother %
En una página como máximo, el resumen explicará de modo breve la problemática que trata de resolver el trabajo \emph{(<<Qué>>)}, la metodología para  abordar su solución  \emph{<<Cómo>>)} y los resultados obtenidos. En los trabajos cuyo idioma principal sea el inglés, el orden de \textsf{Resumen} y \textsf{Abstract} se invertirá.



%---
\cleardoublepage % Se incluye para modificar el contador de página antes de añadir 





% EDITAR: Abstract (máx. 1 pág.)
%---
\phantomsection  % Necesario con hyperref
\addcontentsline{toc}{chapter}{Abstract} % Añade al TOC.
\selectlanguage{english} % Selección de idioma del resumen.
\makeatletter
\begin{center} %
   {\textsc{MASTER THESIS - FACULTAD DE LETRAS (UCLM)}\par}
   \vspace{1cm} %  
   {\textbf{\Large Title in English}\par}
   \vspace{0.4cm} %
   {\@autor \\ \@cityTF,{} \@monthTF{} \@yearTF\par} 
   \vspace{0.9cm} %
   {\textbf{\large\textsf{Abstract}}\par} 
\end{center}   
\makeatother %
\emph{English version for the abstract.}
\cleardoublepage % Se incluye para modificar el contador de página antes de añadir 

