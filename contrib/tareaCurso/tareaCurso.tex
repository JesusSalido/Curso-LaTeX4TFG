\documentclass[paper=a4,fontsize=11pt]{scrartcl}
\usepackage[utf8]{inputenx}
\usepackage[spanish]{babel}
\usepackage{amsmath,amsfonts,amsthm}


% Tipografía
\usepackage{libertine}
\usepackage[libertine]{newtxmath}
\usepackage[T1]{fontenc} 


\usepackage{sectsty} 
\allsectionsfont{\centering \normalfont\scshape} 
% Todas las secciones centradas, con la fuente normal y versalitas

\usepackage{fancyhdr}
\pagestyle{fancyplain}

\fancyhead{}
\fancyfoot[L]{} % Pie izquierdo vacío
\fancyfoot[C]{} % Pie central vacío
\fancyfoot[R]{\thepage} % Pie dcho. con núm. de página
\renewcommand{\headrulewidth}{0pt} % Elimina línea de cabecera
\renewcommand{\footrulewidth}{0pt} % Elimina línea de pie
\setlength{\headheight}{13.6pt} % Altura de la cabecera

\numberwithin{equation}{section} % Añade núm. de sección (i.e. 1.1, 1.2, 2.1, 2.2 en lugar de 1, 2, 3, 4)
\numberwithin{figure}{section}
\numberwithin{table}{section}

\setlength\parindent{0pt} % Elimina sangría en los párrafos (evitar espacio en blanco)

\newcommand{\horrule}[1]{\rule{\linewidth}{#1}} % Crea línea horizontal con la altura como argumento


% EDITAR: Datos del trabajo
\title{	
\normalfont \normalsize 
\textsc{Universidad de Castilla-La Mancha, Escuela Superior de Informática} \\ [25pt]
\horrule{0.5pt} \\[0.4cm] % Línea fina
\huge <Título de la Tarea> \\
\horrule{2pt} \\[0.5cm] % Línea gruesa
}

\author{<Nombre Apellido>}

\date{\normalsize\today}

\begin{document}

\maketitle

\section{Título del Problema}

Phasellus viverra nulla ut metus varius laoreet. Quisque rutrum. Aenean imperdiet. Etiam ultricies nisi vel augue. Curabitur ullamcorper ultricies

\begin{align} 
\begin{split}
(x+y)^3 	&= (x+y)^2(x+y)\\
&=(x^2+2xy+y^2)(x+y)\\
&=(x^3+2x^2y+xy^2) + (x^2y+2xy^2+y^3)\\
&=x^3+3x^2y+3xy^2+y^3
\end{split}					
\end{align}

Phasellus viverra nulla ut metus varius laoreet. Quisque rutrum. Aenean imperdiet. Etiam ultricies nisi vel augue. Curabitur ullamcorper ultricies

%------------------------------------------------

\subsection{Cabecera de nivel 2 (subsection)}

Lorem ipsum dolor sit amet, consectetuer adipiscing elit. 
\begin{align}
A = 
\begin{bmatrix}
A_{11} & A_{21} \\
A_{21} & A_{22}
\end{bmatrix}
\end{align}
Aenean commodo ligula eget dolor. Aenean massa. Cum sociis natoque penatibus et magnis dis parturient montes, nascetur ridiculus mus. Donec quam felis, ultricies nec, pellentesque eu, pretium quis, sem.



\subsubsection{Cabecera de nivel 3 (subsubsection)}

Phasellus viverra nulla ut metus varius laoreet. Quisque rutrum. Aenean imperdiet. Etiam ultricies nisi vel augue. Curabitur ullamcorper ultricies

Phasellus viverra nulla ut metus varius laoreet. Quisque rutrum. Aenean imperdiet. Etiam ultricies nisi vel augue. Curabitur ullamcorper ultricies



\end{document}