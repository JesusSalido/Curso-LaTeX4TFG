%%%%%%%%%%%%%%%%%%%%%%%%%%%%%%%%%%%%%%%%%%%%%%%%%%%%%%%%%%%%%%%
%
% Open Research Europe is an open access publishing platform for the publication of research stemming from Horizon 2020 funding across all subject areas. The platform makes it easy for Horizon 2020 beneficiaries to comply with the open access terms of their funding and offers researchers a publishing venue to share their results and insights rapidly and facilitate open, constructive research discussion.
%
%%%%%%%%%%%%%%%%%%%%%%%%%%%%%%%%%%%%%%%%%%%%%%%%%%%%%%%%%%%%%%%
%
% This template is for all article types; for information on specific article type requirements please visit https://open-research-europe.ec.europa.eu/for-authors/article-guidelines
%
% For more information on the Open Research Europe publishing model please see:  https://open-research-europe.ec.europa.eu/about

\documentclass[10pt,a4paper]{article}
\usepackage{f1000_styles}

%% Default: numerical citations
\usepackage[numbers]{natbib}

%% Uncomment this lines for superscript citations instead
% \usepackage[super]{natbib}

%% Uncomment these lines for author-year citations instead
% \usepackage[round]{natbib}
% \let\cite\citep

\begin{document}
\pagestyle{fancy}

\title{Open Research Europe Article Template}
\titlenote{Please provide a concise and specific title that clearly reflects the content of the article.}
\author[1]{Author Name-1}
\author[2]{Author Name-2}
\affil[1]{Address of author-1}
\affil[2]{Address of author-2}

\maketitle
\thispagestyle{fancy}

Please list all authors that played a significant role in writing the article. As a guide, authors should refer to the criteria for authorship that have been developed by The International Committee of Medical Journal Editors \href{http://www.icmje.org/recommendations/browse/roles-and-responsibilities/defining-the-role-of-authors-and-contributors.html}{(ICMJE)}. Please provide full affiliation information (including full institutional address, ZIP code and e-mail address) for all authors, and identify who is/are the corresponding author(s).
\\
\\
\begin{abstract}

Abstracts should be up to 300 words and provide a succinct summary of the article. Although the abstract should explain why the article might be interesting, care should be taken not to inappropriately over-emphasise the importance of the work described in the article. Citations should not be used in the abstract, and abbreviations, if needed, should be spelled out in full.

\end{abstract}

\section*{\color{OREblue}Keywords}

Please list up to eight relevant keywords that describe the subject of their article. These will improve the visibility of your article.


\clearpage
\pagestyle{fancy}

\section*{Plain language summary}

It is recommended that authors include a plain language summary in their article. Advice for writing a plain language summary can be found \href{https://www.dcc.ac.uk/guidance/how-guides/write-lay-summary}{here}.

\section*{Main body}

The format of the main body of the article is flexible: it should be concise, making it easy to read and review, and presented in a format that is appropriate for the type of study presented.

\section*{Format}

The format of articles can be flexible, as long as the methods and sources are described in sufficient detail for others to be able to understand and repeat the work. A typical format may include an introduction, methods and materials, results and analysis, and discussions and conclusions, but this will vary depending on the article type. Please refer to our \href{https://open-research-europe.ec.europa.eu/for-authors/article-guidelines}{Article Guidelines} for more information on the requirements for specific article types.

\subsection*{Sub-headings}
Please use this format to denote a subheading or subsection within the main sections of an article.

\subsection*{Tables}
Use \textbackslash table and \textbackslash tabledata for basic tables. See \autoref{exampletable}, for example.
\begin{table}
    \hrule height 0.05cm  \vspace{0.1cm}
	\caption{\label{exampletable}An example of a simple table with caption.}
	\centering
	\begin{tabledata}{$l^l^r} 
		\header First name & Last Name & Grade \\ 
		\row John & Doe & 7.5 \\ 
		\row Richard & Miles & 2 
	\end{tabledata}
\end{table}

\subsection*{Figures}
You can upload a figure (JPEG, PNG or PDF) using the files menu. To include it in your document, use  \textbackslash includegraphics (see the example in the source code below). All figures should be discussed in the article text.

Please give figures appropriate filenames eg: figure1.pdf, figure2.png.

Figure legends should briefly describe the key messages of the figure such that the figure can stand alone from the main text, and avoid lengthy descriptions of the methods. Each legend should have a concise title of no more than 15 words. Please ensure all abbreviations used in your figures and legends are defined to allow them to stand independently from the main body of the text.

If reusing a figure or table from a previous publication, the authors are responsible for obtaining permission from the copyright holder and for the payment of any fees (if applicable). Please include a note in the legend to state that: ‘This figure/table has been reproduced with permission from \textit{[include original publication citation]}’.

\begin{figure}
	\centering
	\includegraphics[width=0.8\textwidth]{ORE Header.png}
	\caption{\label{fig:your-figure}Your figure legend goes here; it should be succinct, while still explaining all symbols and abbreviations. }
\end{figure}

\subsection*{Mathematical scripts}

There are no strict rules on the format of mathematical scripts however, here is some useful advice:
\begin{itemize}
\item Special care should be taken with mathematical scripts, especially subscripts and superscripts and differentiation between the letter “ell” (l) and the figure one (1), and the letter “oh” (o) and the figure zero (0).
\item It is important to differentiate between: K and k; X, x and x (multiplication); asterisks intended to appear when published as multiplication signs and those intended to remain as asterisks, etc.
\item In both displayed equations and in text, scalar variables must be in italics, with non-variable matter in upright type.
\item For simple fractions in the text, the solidus “/” should be used instead of a horizontal line, with care being taken to insert parentheses where necessary to avoid ambiguity. Exceptions are the proper fractions available (e.g., ¼, ½, ¾).
\item The solidus is not generally used for units: m s\textsuperscript{-1} not m/s, but note electrons/s, counts/channel, etc.
\item Displayed equations referred to in the text should be numbered serially ((1), (2), etc.) on the right-hand side. Short expressions not referred to by any number will usually be incorporated into the text.
\item The following styles are preferred: upright bold sans serif r for tensors, bold serif italic r for vectors, upright bold serif r for matrices, and medium-face sloping serif r for scalar variables. In mathematical expressions, the use of “d” for differential should be made clear, and coded in roman, not italic.
\item Braces, brackets, and parentheses are used in the order \{ [( )] \}, except where mathematical convention dictates otherwise (e.g., square brackets for commutators and anticommutators; braces for the exponent in exponentials).
\item For units and symbols, the SI system should be used. Where measurements are given in other systems, please insert conversions.
\end{itemize}

\section*{Data and software availability} % Required
Use this section to provide the raw data that support their findings. Readers should be able to view the raw data, replicate the study, and re-analyse and/or reuse the data (with appropriate attribution). Please take a look at the Open Research Europe guidelines on \href{https://open-research-europe.ec.europa.eu/for-authors/data-guidelines}{data preparation}.
Raw data should be uploaded to an approved repository before submission, a list of which can be found on the \href{https://open-research-europe.ec.europa.eu/for-authors/data-guidelines#hosting}{data guidelines page}.

This section should be completed in the following format:

\subsection*{Source data}

If the data has been published previously, details of the dataset and where it can be accessed should be provided here.

\subsection*{Underlying data}

This section should detail all novel data collected and used as part of your article. Details of the repository where your data are hosted, a description of the data files and the license under which they are held should be included. See the Open Research Europe Data Guidelines for more information. The following formatting should be used:
\begin{quote}
Repository: Manually annotated miRNA-disease and miRNA-gene interaction corpora.\\
https://doi.org/10.5256/repository.4591.d34639.
\\
\\
This project contains the following underlying data:
\begin{itemize}
	\item Data file 1. (Description of data.)
	\item Data file 2. (Description of data.)
\end{itemize}

Data are available under the terms of the Creative Commons Zero "No rights reserved" data waiver (CC0 1.0 Public domain dedication).
\end{quote}
\subsection*{Extended data}

Additional materials that support the key claims in the paper but are not absolutely required to follow the study design and analysis of the results (e.g. questionnaires, supporting images or tables) should be included as extended data. Details of the repository where these materials are hosted, a description of the extended data files and the license under which they are held should be included. See the Open Research Europe Data Guidelines for more information.

\subsection*{Software availability}
If you are describing new software, please make the source code available on a Version Control System (VCS) such as GitHub, BitBucket or SourceForge, and provide details of the repository and the license under which the software can be used in the article.

\section*{Competing interests}
All financial, personal, or professional competing interests for any of the authors that could be construed to unduly influence the content of the article must be disclosed and will be displayed alongside the article. If there are no relevant competing interests to declare, please add the following: 'No competing interests were disclosed'.

\section*{Grant information}
Please provide details of the Horizon 2020 project ID and project title that supported the work presented in the article, and, if applicable, of any other funders or employers who funded the work. For each funder, please state the funder’s name, the grant number where applicable and known, and the individual to whom the grant was assigned.
Please do not list funding that is not relevant to this specific piece of research.

\section*{Acknowledgements}
This section should acknowledge anyone who contributed to the research or the article but who does not qualify as an author based on the criteria provided earlier (e.g. someone or an organization that provided writing assistance). Please state how they contributed; authors should obtain permission to acknowledge from all those mentioned in the Acknowledgements section.

Please do not list grant funding in this section.

{\small\bibliographystyle{unsrtnat}
\bibliography{sample}}

Include a reference list in your .tex file - if using a .bib file this can be generated with \textbackslash bibliography as demonstrated above. References can be listed in any standard referencing style and should be consistent between references within a given article. In-line references should be formatted using \textbackslash cite, for example \cite{Smith:2012qr} and \cite{Smith:2013jd} 


\section*{Using LaTeX}
In order to ensure smooth and successful processing of your LaTeX manuscript, please follow these guidelines. Before submitting ensure that your PDF appears correctly on Overleaf, to avoid delays in processing. You can view and outstanding errors on the Logs and Output Files tab, just to the right of the green Recompile button.

As you prepare your LaTeX manuscript, please bear in mind the following general guidelines.  
Keep it simple:

\begin{enumerate}
    \item[~]
	\begin{enumerate}
		\item Keep your LaTeX files as simple as possible; do not use elaborate local macros or highly customized style files. Preferably, use the template provided for formatting your paper.
		\item Preferably prepare only one .tex file. 
		\item Do not use external style files or packages, except for f1000styles.sty and those packages already referenced in the main.tex template. If you need additional macros, please keep them simple and include them in the .tex document preamble.
		\item Source code should be structured so that all .sty and .bst files called by the main .tex file are in the same directory as the main .tex file.
		\item AMS math commands are recommended when inserting math equations into your manuscript.
		\item When using URLs in the text these should be incorporated as hyperlinks using the \textbackslash hyperref package and \textbackslash href\{\}\{\} function where possible.
		\item References to figures and tables within the manuscript should use \textbackslash autoref\{\}
	\end{enumerate}
\end{enumerate}

References:
\begin{itemize}
	\item Reference management systems provide options for exporting bibliographies as BibTEX files (.bib). This template contains an example of such a file, sample.bib, which can be replaced with your own.
	\item Use only the generic \textbackslash cite\{\} command for referencing in the text (like this [1] and this [2]), not other commands built on special macros. Also, make sure that there is no space between reference keynames within the braces (i.e., \textbackslash cite\{refone,reftwo,refthree\}, not \textbackslash cite\{refone, reftwo, refthree\}).
\end{itemize}

\section*{Submitting your article}
Generate a PDF file of your project and submit this alongside a zip file containing all project files (including the source files, style files, and PDF) using our \href{https://open-research-europe.ec.europa.eu/for-authors/publish-your-research}{online submission form}. 

% See this guide for more information on BibTeX:
% http://libguides.mit.edu/content.php?pid=55482&sid=406343

% Please note that this template results in a draft pre-submission PDF document.
% Articles will be professionally typeset when accepted for publication.

% We hope you find the Open Research Europe LaTex template useful, please contact us if you have any feedback.

\end{document}