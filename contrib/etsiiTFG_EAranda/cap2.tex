%!TEX root=master.tex

\thispagestyle{empty}
\chapter{Salida al amanecer}

   Hechas, pues, estas prevenciones, no quiso aguardar más tiempo a
    poner en efecto su pensamiento, apretándole a ello la falta que él
    pensaba que hacía en el mundo su tardanza, según eran los agravios
    que pensaba deshacer, tuertos que enderezar, sinrazones que
    enmendar, y abusos que mejorar, y deudas que satisfacer; y así,
    sin dar parte a persona alguna de su intención, y sin que nadie le
    viese, una mañana, antes del día (que era uno de los calurosos del
    mes de Julio), se armó de todas sus armas, subió sobre Rocinante,
    puesta su mal compuesta celada, embrazó su adarga, tomó su lanza,
    y por la puerta falsa de un corral, salió al campo con grandísimo
    contento y alborozo de ver con cuánta facilidad había dado
    principio a su buen deseo. Mas apenas se vió en el campo, cuando
    le asaltó un pensamiento terrible, y tal, que por poco le hiciera
    dejar la comenzada empresa: y fue que le vino a la memoria que no
    era armado caballero, y que, conforme a la ley de caballería, ni
    podía ni debía tomar armas con ningún caballero; y puesto qeu lo
    fuera, había de llevar armas blancas, como novel caballero, sin
    empresa en el escudo, hasta que por su esfuerzo la ganase.
    
    Estos pensamientos le hicieron titubear en su propósito; mas
    pudiendo más su locura que otra razón alguna, propuso de hacerse
    armar caballero del primero que topase, a imitación de otros
    muchos que así lo hicieron, según él había leído en los libros que
    tal le tenían. En lo de las armas blancas pensaba limpiarlas de
    manera, en teniendo lugar, que lo fuesen más que un armiño: y con
    esto se quietó y prosiguió su camino, sin llevar otro que el que
    su caballo quería, creyendo que en aquello consistía la fuerza de
    las aventuras. Yendo, pues, caminando nuestro flamante aventurero,
    iba hablando consigo mismo, y diciendo: z.Quién duda sino que en
    los venideros tiempos, ciando salga a luz la verdadera historia de
    mis famosos hechos, que el sabio que los escribiere, no ponga,
    cuando llegue a contar esta mi primera salida tan de mañana, de
    esta manera? ``Apenas había el rubicundo Apolo tendido por la faz
    de la ancha y espaciosa tierra las doradas hebras de sus hermosos
    cabellos, y apenas los pequeños y pintados pajarillos con sus
    arpadas lenguas habían saludado con dulce y meliflua armonía la
    venida de la rosada aurora que dejando la blanda cama del celoso
    marido, por las puertas y balcones del manchego horizonte a los
    mortales se mostraba, cuando el famoso caballero D. Quijote de la
    Mancha, dejando las ociosas plumas, subió sobre su famoso caballo
    Rocinante, y comenzó a caminar por el antiguo y conocido campo de
    Montiel.'' (Y era la verdad que por él caminaba) y añadió diciendo:
    ``dichosa edad, y siglo dichoso aquel adonde saldrán a luz las
    famosas hazañas mías, dignas de entallarse en bronce, esculpirse
    en mármoles y esculpirse en mármoles y pintarse en tablas para
    memoria en lo futuro. ¡Oh tú, sabio encantador, quienquiera que
    seas, a quien ha de tocar el ser coronista de esta peregrina
    historia! Ruégote que no te olvides de mi buen Rocinante compañero
    eterno mío en todos mis caminos y carreras.'' Luego volvía
    diciendo, como si verdaderamente fuera enamorado: ``¡Oh, princesa
    Dulcinea, señora de este cautivo corazón! Mucho agravio me habedes
    fecho en despedirme y reprocharme con el riguroso afincamiento de
    mandarme no parecer ante la vuestra fermosura. Plégaos, señora, de
    membraros de este vuestro sujeto corazón, que tantas cuitas por
    vuestro amor padece.''
    
\section{Caminando todo el día}

    Con estos iba ensartando otros disparates, todos al modo de los
    que sus libros le habían enseñado, imitando en cuanto podía su
    lenguaje; y con esto caminaba tan despaico, y el sol entraba tan
    apriesa y con tanto ardor, que fuera bastante a derretirle los
    sesos, si algunos tuviera. Casi todo aquel día caminó sin
    acontecerle cosa que de contar fuese, de lo cual se desesperaba,
    poerque quisiera topar luego, con quien hacer experiencia del
    valor de su fuerte brazo.
    
    Autores hay que dicen que la primera aventura que le avino fue la
    de Puerto Lápice; otros dicen que la de los molinos de viento;
    pero lo que yo he podido averiguar en este caso, y lo que he
    hallado escrito en los anales de la Mancha, es que él anduvo todo
    aquel día, y al anochecer, su rocín y él se hallaron cansados y
    muertos de hambre; y que mirando a todas partes, por ver si
    descubriría algún castillo o alguna majada de pastores donde
    recogerse, y adonde pudiese remediar su mucha necesidad, vió no
    lejos del camino por donde iba una venta, que fue como si viera
    una estrella, que a los portales, si no a los alcázares de su
    redención, le encaminaba. Dióse priesa a caminar, y llegó a ella a
    tiempo que anochecía. Estaban acaso a la puerta dos mujeres mozas,
    de estas que llaman del partido, las cuales iban a Sevilla con
    unos arrieros, que en la venta aquella noche acertaron a hacer
    jornada; y como a nuestro aventurero todo cuanto pensaba, veía o
    imaginaba, le parecía ser hecho y pasar al modo de lo que había
    leído, luego que vió la venta se le representó que era un castillo
    con sus cuatro torres y chapiteles de luciente plata, sin faltarle
    su puente levadizo y honda cava, con todos aquellos adherentes que
    semejantes castillos se pintan.
    
    Fuese llegando a la venta (que a él le parecía castillo), y a poco
    trecho de ella detuvo las riendas a Rocinante, esperando que algún
    enano se pusiese entre las almenas a dar señal con alguna trompeta
    de que llegaba caballero al castillo; pero como vió que se
    tardaban, y que Rocinante se daba priesa por llegar a la
    caballeriza, se llegó a la puerta de la venta, y vió a las dos
    distraídas mozas que allí estaban, que a él le parecieron dos
    hermosas doncellas, o dos graciosas damas, que delante de la
    puerta del castillo se estaban solazando. En esto sucedió acaso
    que un porquero, que andaba recogiendo de unos rastrojos una
    manada de puercos (que sin perdón así se llaman), tocó un cuerno,
    a cuya señal ellos se recogen, y al instante se le representó a D.
    Quijote lo que deseaba, que era que algún enano hacía señal de su
    venida, y así con extraño contento llegó a la venta y a las damas,
    las cuales, como vieron venir un hombre de aquella suerte armado,
    y con lanza y adarga, llenas de miedo se iban a entrar en la
    venta; pero Don Quijote, coligiendo por su huida su miedo,
    alzándose la visera de papelón y descubriendo su seco y polvoso
    rostro, con gentil talante y voz reposada les dijo: non fuyan las
    vuestras mercedes, nin teman desaguisado alguno, ca a la órden de
    caballería que profeso non toca ni atañe facerle a ninguno, cuanto
    más a tan altas doncellas, como vuestras presencias demuestran.
    
    Mirábanle las mozas y andaban con los ojos buscándole el rostro
    que la mala visera le encubría; mas como se oyeron llamar
    doncellas, cosa tan fuera de su profesión, no pudieron tener la
    risa, y fue de manera, que Don Quijote vino a correrse y a
    decirles: Bien parece la mesura en las fermosas, y es mucha sandez
    además la risa que de leve causa procede; pero non vos lo digo
    porque os acuitedes ni mostredes mal talante, que el mío non es de
    al que de serviros.
    
    El lenguaje no entendido de las señoras, y el mal talle de nuestro
    caballero, acrecentaba en ellas la risa y en él el enojo; y pasara
    muy adelante, si a aquel punto no saliera el ventero, hombre que
    por ser muy gordo era muy pacífico, el cual, viendo aquella figura
    contrahecha, armada de armas tan desiguales, como eran la brida,
    lanza, adarga y coselete, no estuvo en nada en acompañar a las
    doncellas en las muestras de su contento; mas, en efecto, temiendo
    la máquina de tantos pertrechos, determinó de hablarle
    comedidamente, y así le dijo: si vuestra merced, señor caballero,
    busca posada, amén del lecho (porque en esta venta no hay
    ninguno), todo lo demás se hallará en ella en mucha abundancia.
    Viendo Don Quijote la humildad del alcaide de la fortaleza (que
    tal le pareció a él el ventero y la venta), respondió: para mí,
    señor castellano, cualquiera cosa basta, porque mis arreos son las
    armas, mi descanso el pelear, etc.
    
    Pensó el huésped que el haberle llamado castellano había sido por
    haberle parecido de los senos de Castilla, aunque él era andaluz y
    de los de la playa de Sanlúcar, no menos ladrón que Caco, ni menos
    maleante que estudiante o paje. Y así le respondió: según eso, las
    camas de vuestra merced serán duras peñas, y su dormir siempre
    velar; y siendo así, bien se puede apear con seguridad de hallar
    en esta choza ocasión y ocasiones para no dormir en todo un año,
    cuanto más en una noche. Y diciendo esto, fue a tener del estribo
    a D. Quijote, el cual se apeó con mucha dificultad y trabajo, como
    aquel que en todo aquel día no se había desayunado. Dijo luego al
    huésped que le tuviese mucho cuidad de su caballo, porque era la
    mejor pieza que comía pan en el mundo.

\section{El ventero}

    Miróle el ventero, y no le pareció tan bueno como Don Quijote
    decía, ni aun la mitad; y acomodándole en la caballeriza, volvió a
    ver lo que su huésped mandaba; al cual estaban desarmando las
    doncellas (que ya se habían reconciliado con él), las cuales,
    aunque le habían quitado el peto y el espaldar, jamás supieron ni
    pudieron desencajarle la gola, ni quitarle la contrahecha celada,
    que traía atada con unas cintas verdes, y era menester cortarlas,
    por no poderse queitar los nudos; mas él no lo quiso consentir en
    ninguna manera; y así se quedó toda aquella noche con la celada
    puesta, que era la más graciosa y extraña figura que se pudiera
    pensar; y al desarmarle (como él se imaginaba que aquellas traídas
    y llevadas que le desarmaban, eran algunas principales señoras y
    damas de aquel castillo), les dijo con mucho donaire:
    
    Nunca fuera caballero de damas tan bien servido, como fuera D.
    Quijote cuando de su aldea vino; doncellas curaban dél, princesas
    de su Rocino.
    
    O Rocinante, que este es el nombre, señoras mías, de mi caballo, y
    Don Quijote de la Mancha el mío; que puesto que no quisiera
    descubrirme fasta que las fazañas fechas en vuestro servicio y pro
    me descubrieran, la fuerza de acomodar al propósito presente este
    romance viejo de Lanzarote, ha sido causa que sepáis mi nombre
    antes de toda sazón; pero tiempo vendrá en que las vuestras
    señorías me manden, y yo obedezca, y el valor de mi brazo descubra
    el deseo que tengo de serviros. Las mozas, que no estaban hechas a
    oír semejantes retóricas, no respondían palabra; sólo le
    preguntaron si quería comer alguna cosa. Cualquiera yantaría yo,
    respondió D. Quijote, porque a lo que entiendo me haría mucho al
    caso. A dicha acertó a ser viernes aquél día, y no había en toda
    la venta sino unas raciones de un pescado, que en Castilla llaman
    abadejo, y en Andalucía bacalao, y en otras partes curadillo, y en
    otras truchuela.
    
    Preguntáronle si por ventura comería su merced truchuela, que no
    había otro pescado que darle a comer. Como haya muchas truchuelas,
    respondió D. Quijote, podrán servir de una trueba; porque eso se
    me da que me den ocho reales en sencillos, que una pieza de a
    ocho. Cuanto más, que podría ser que fuesen estas truchuelas como
    la ternera, que es mejor que la vaca, y el cabrito que el cabrón.
    Pero sea lo que fuere, venga luego, que el trabajo y peso de las
    armas no se puede llevar sin el gobierno de las tripas. Pusiéronle
    la mesa a la puerta de la venta por el fresco, y trájole el
    huésped una porción de mal remojado, y peor cocido bacalao, y un
    pan tan negro y mugriento como sus armas. Pero era materia de
    grande risa verle comer, porque como tenía puesta la celada y
    alzada la visera, no podía poner nada en la boca con sus manos, si
    otro no se lo daba y ponía; y así una de aquellas señoras sería de
    este menester; mas el darle de beber no fue posible, ni lo fuera
    si el ventero no horadara una caña, y puesto el un cabo en la
    boca, por el otro, le iba echando el vino. Y todo esto lo recibía
    en paciencia, a trueco de no romper las cintas de la celada.
    
    Estando en esto, llegó acaso a la venta un castrador de puercos, y
    así como llegó sonó su silbato de cañas cuatro o cinco veces, con
    lo cual acabó de confirmar Don Quijote que estaba en algún famoso
    castillo, y que le servían con música, y que el abadejo eran
    truchas, el pan candeal, y las rameras damas, y el ventero
    castellano del castillo; y con esto daba por bien empleada su
    determinación y salida. Mas lo que más le fatigaba era el no  verse
    armado caballero, por parecerle que no se podría poner
    legítimamente en aventura alguna sin recibir la órden de
    caballería.
