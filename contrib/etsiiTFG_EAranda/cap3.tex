\chapter{UNIX}
\thispagestyle{empty}
\section{Historia}
UNIX es uno de los sistemas operativos mundialmente más famosos a
causa de su amplia distribuición y base soportada. Se desarrolló
originalmente en la AT\&T como sistema multitarea para miniordenadores
y mainframes en la década de los 70, pero desde entonces ha crecido
hasta convertirse en uno de los sistemas más ampliamente usados por
doquier, a pesar de su interfaz, a veces confuso, y de su falta de
estandarización por parte de una entidad centralizadora.

Muchos hackers sienten que Unix es Lo Que Vale la Pena, el único
sistema operativo de verdad. De ahí{} proviene el desarrollo de por
parte de un grupo, siempre en aumento, de hackers del Unix que quieren
``currárselo'' con un sistema que puedan llamar propio.

Existen versiones de Unix para muchos sistemas, desde ordenadores
personales hasta superordenadores como el Cray Y-MP. La mayor parte de
las versiones de Unix para ordenadores personales son caras y
difíciles. Al escribir esto, una versión de UNIX System V para una
sola máquina 386 cuesta unos 1500\$ USA.

Linux es una versión libre de Unix desarrollada originalmente por
Linus Torvalds en la Universidad de Helsinki en Finlandia, con la
ayuda, a través de Internet, de numerosos programadores y expertos de
Unix. Cualquiera que tenga el instinto y los conocimientos suficientes
puede desarrollar y modificar el sistema. El núcleo de Linux no utiliza
código de AT\&T o de cualquier otra fuente propietaria, y gran parte
del software disponible para Linux ha sido desarrollado por el proyecto GNU
de la Free Software Foundation en Cambridge, Massachusetts (EEUU). No
obstante, también programadores de todo el mundo contribuyen a que
aumente cada vez más el software disponible para Linux.

Linux, el núcleo o kernel de GNU/Linux, se desarrolló originalmente
como un proyecto que Linus Torvalds emprendió en su tiempo libre. Se
inspiró en Minix, un sistema Unix básico desarrollado por Andy
Tanenbaum. Las primeras discusiones acerca del núcleo Linux tuvieron
lugar en el grupo de noticias de Usenet comp.os.minix. Estas
discusiones se centraban sobre todo en el desarrollo de un sistema
pequeño y académico de Unix para usuarios de Minix que querían algo
más.

El primitivo desarrollo del núcleo Linux se centró en las
características multitarea del interfaz en modo protegido del 80386,
escrito en código ensamblador. Linus escribe:

\begin{quote}
Después de todo, ha sido una navegación tranquila; el código
era
tremendo, pero tenía algunos dispositivos, y la depuración fue más
fácil. En esta etapa comencé a usar C, y ciertamente acelera el
desarrollo. También fue entonces cuando me empecé a poner serio en mi
megalomaníaca idea de hacer `un Minix mejor que Minix'. Esperaba poder
recompilar gcc bajo el núcleo Linux algún día\dots


Dos meses para la configuración básica, pero luego sólo un poco más
hasta que tuve un controlador de disco (gravemente plagado de errores,
pero resultó que funcionaba en mi ordenador) y un pequeño sistema de
ficheros. Fue por aquel entonces cuando dejé disponible la versión
0.01 (más o menos a finales de agosto de 1991): no era bonito, no
tenía controlador de disquetera, y no podía hacer mucho en ningún
sentido. No creo siquiera que nadie compilara jamás esa versión. Pero
para entonces ya estaba enganchado, y no quería parar hasta conseguir
dejar fuera a Minix.
\end{quote}

Nunca se hizo un anuncio de la versión 0.01. Las fuentes del 0.01 ni
siquiera eran ejecutables. Contenían sólo los rudimentos básicos de
las fuentes del núcleo y daban por supuesto que se tenía acceso a una
máquina con Minix para compilarlas y experimentar con ellas.

\begin{figure}
\centering
\includegraphics[width=4cm]{uclm}
\caption{Escudo de la Universidad}
\end{figure}

El 5 de octubre de 1991 Linus anunció la primera versión ``oficial''
del núcleo Linux, la versión 0.02. En este punto, Linus podía ejecutar
bash (el GNU Bourne Again Shell) y gcc (el compilador C GNU) pero no
mucho más. De nuevo, estaba pensado como un sistema para hackers. La
orientación principal fue el desarrollo del núcleo; el soporte de
usuarios, la documentación y la distribución todavía no habían sido
atendidos. Aún hoy, la comunidad parece que aún trata estas cosas como
secundarias frente a la ``programación de verdad'' (el desarrollo del
núcleo).

Según escribió Linus en comp.os.minix,

\begin{quote}
¿Suspiras por los fabulosos días de Minix-1.1, cuando los hombres
eran hombres y escribían sus propios controladores de dispositivo? ¿Te
encuentras sin un buen proyecto y te mueres por hincar los dientes a
un sistema operativo que puedas intentar modificar para tus
necesidades? Encuentras frustrante que todo en Minix funcione? ¿Se
acabaron las amanecidas para conseguir que funcione ese programa
cañero? Entonces este mensaje puede que sea para ti.

Tal y como mencioné hace un mes, estoy trabajando en una versión
libre de una especie de Minix para ordenadores AT-386. Por fin ha
alcanzado el estado en el que incluso se puede usar (aunque a lo mejor
no se puede, depende de para qué lo quieras), y deseo dejar el código
fuente libre para que alcance mayor distribución. Sólo es la versión
0.02\dots, pero ya he ejecutado con éxito bash, gcc, gnu-make, gnu-sed,
compress, etcétera, bajo este sistema.
\end{quote}

\begin{table}
\centering
\begin{tabular}{lr@{\,--\,}lp{5.8cm}}
\multicolumn{1}{c}{\bf Cap\'\i tulo}
        & \multicolumn{2}{c}{\bf P\'aginas}
           & \multicolumn{1}{l}{\bf Resumen}\\\hline
\textbf{1. N\'umeros} & 1 & 8 &%
Se establece el lenguaje b\'asico y se definen los <<conjuntos num\'ericos>>  y
sus propiedades b\'asicas.\\
\textbf{2. Continuidad} & 9 & 20 &%
Se define el concepto de funci\'on
continua y se estudian la propiedades de las funciones continuas.

En particular, se estudian el teorema de los valores intermedios y el teorema de
Weierstrass.\\\hline
\end{tabular}
\caption{Contenido del curso}
\end{table}

Después de la versión 0.03 Linus dio el salto a la versión 0.10, según
empezó a trabajar más gente en el sistema. Después de varias
revisiones posteriores, Linus incrementó el número de versión a la
0.95 en marzo de 1992 para reflejar su impresión de que el sistema
estaba preparado para un inminente lanzamiento ``oficial''.
(Generalmente a un programa no se le numera con la versión 1.0 hasta
que no está en teoría completo, o libre de errores). Casi a\~no y medio
después, a finales de diciembre de 1993, el núcleo  estaba todavía
en la versión 0.99.p114, acercándose asintóticamente a la versión 1.0.
En el momento de escribir esto, la versión estable actual es la 2.2.10
y está en desarrollo la versión 2.3.

Casi todos los paquetes de programas UNIX importantes libremente
redistribuibles han sido portados a Linux, y también hay abundante software
comercial. La lista de hardware soportado es mayor que la del núcleo
original. Mucha gente ha ejecutado benchmarks (pruebas de rendimiento)
en sistemas Linux 80486 y han encontrado que es comparable a
estaciones de trabajo de Sun Microsystems y Digital Equipment
Corporation. ¿Quién hubiera adivinado que este ``pequeño'' clónico de
UNIX iba a crecer tanto como para dominar el mundo de la computación
personal en su totalidad? Citación del libro \cite{CLMPS}


