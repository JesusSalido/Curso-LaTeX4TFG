%%%%%%%%%%%%%% 
% Fichero: EjCodeList
% Autor: J. Salido (http://www.esi.uclm.es/www/jsalido)
% Fecha: febrero, 2017
% Descripción: Ejemplos de entornos para incluir listados de 
% código de leguajes de programación.
% Ejemplo del curso: “LaTeX esencial para preparación de TFG, Tesis
% y otros documentos académicos” (Esc. Sup. Informática-UCLM)
%%%%%%%%%%%%%%




%%%%%%%%%%%%%%
% Preámbulo del documento
%%%%%%%%%%%%%%
\documentclass[11pt,a4paper]{article} 
\usepackage[utf8]{inputenx} 
\usepackage[spanish]{babel} 
\usepackage[left=2cm,right=2cm,top=2cm,bottom=2cm]{geometry} % Márgenes 

% Tipografía
\usepackage{newpxtext}
\usepackage{newpxmath}

\usepackage{marvosym}
\usepackage{pifont} % Generación de símbolos especiales
\usepackage{textcomp}
\usepackage{ccicons}

\usepackage[T1]{fontenc} % Codificación de salida    
\usepackage{microtype} % Mejoras de microtipografía en la obtención de PDF (sólo para pdflatex)

\usepackage[
	type={CC},
	modifier={by-sa},
	version={3.0},
]{doclicense}



% Generación de hiperenlaces
\usepackage[pdftex]{hyperref}
\urlstyle{sf}


% Ajustes de paquete hyperref 
\hypersetup{
	breaklinks,
	%	hidelinks,       % Oculta el color y borde de los links
    colorlinks=true,
	linkcolor=red,    % Color de los links
	citecolor=blue,  % Color de la citas
	urlcolor=blue,   % Color de las URL
	bookmarksnumbered=true, % Incluye números en bookmarks		
	pdftitle={Fundamentos de LaTeX para principiantes}, % Título
	pdfauthor={Jesús Salido}, % Autor
	pdfsubject={LaTeX}, % Tema
	pdftoolbar=true, % Muestra la toolbar de Acrobat
	pdfmenubar=true % Muestra la menubar de Acrobat
}


% Definición de colores
\usepackage[usenames,dvipsnames,svgnames,x11names,table]{xcolor}


% Paquetes especiales para Informática

% Algoritmos
\usepackage[ruled,vlined,spanish]{algorithm2e}
%\usepackage[lined,boxed,commentsnumbered,spanish]{algorithm2e}
% Paquete específico para la generación de algorithmos en LaTeX. Por su complejidad se requiere consultar la documentación del paquete para su correcto uso.


% Listados de código
\usepackage{listings} % Inclusión de listados de código 
% Personalización del entorno lstlisting (ver documentación para más infor.)

% Definición incluida para que puedan interpretarse correctamente los caracteres unicode incluidos en las porciones de código. El problema deriva de que el paquete listings sólo reconoce la codificación de caracteres con 1 byte.

% -------------------------
% AJUSTES PARA LISTADOS DE CÓDIGO (paquete listings)
% Definición incluida para que puedan interprestarse correctamente los caracteres unicode incluidos en las porciones de código. El problema deriva de que el paquete listings sólo reconoce la codificación de caracteres con 1 byte.
\lstset{inputencoding=utf8,
	extendedchars=true,
	literate=%
	{á}{{\'a}}1 {é}{{\'e}}1 {í}{{\'i}}1 {ó}{{\'o}}1 {ú}{{\'u}}1
	{Á}{{\'A}}1 {É}{{\'E}}1 {Í}{{\'I}}1 {Ó}{{\'O}}1 {Ú}{{\'U}}1
	{à}{{\`a}}1 {è}{{\`e}}1 {ì}{{\`i}}1 {ò}{{\`o}}1 {ù}{{\`u}}1
	{À}{{\`A}}1 {È}{{\'E}}1 {Ì}{{\`I}}1 {Ò}{{\`O}}1 {Ù}{{\`U}}1
	{ä}{{\"a}}1 {ë}{{\"e}}1 {ï}{{\"i}}1 {ö}{{\"o}}1 {ü}{{\"u}}1
	{Ä}{{\"A}}1 {Ë}{{\"E}}1 {Ï}{{\"I}}1 {Ö}{{\"O}}1 {Ü}{{\"U}}1
	{â}{{\^a}}1 {ê}{{\^e}}1 {î}{{\^i}}1 {ô}{{\^o}}1 {û}{{\^u}}1
	{Â}{{\^A}}1 {Ê}{{\^E}}1 {Î}{{\^I}}1 {Ô}{{\^O}}1 {Û}{{\^U}}1
	{œ}{{\oe}}1 {Œ}{{\OE}}1 {æ}{{\ae}}1 {Æ}{{\AE}}1 {ß}{{\ss}}1
	{ű}{{\H{u}}}1 {Ű}{{\H{U}}}1 {ő}{{\H{o}}}1 {Ő}{{\H{O}}}1
	{ç}{{\c c}}1 {Ç}{{\c C}}1 {ø}{{\o}}1 {å}{{\r a}}1 {Å}{{\r A}}1
	{€}{{\euro}}1 {£}{{\pounds}}1 {«}{{\guillemotleft}}1
	{»}{{\guillemotright}}1 {ñ}{{\~n}}1 {Ñ}{{\~N}}1 {¿}{{?`}}1
	{¡}{{\textexclamdown}}1
}


% Las opciones aquí incluidas deben servir de ejemplo, ya que para una lista exhaustiva se debe consultar la documentación del paquete.
\lstset{ % Estilo por defecto
	belowcaptionskip=5pt, % Espacio bajo el título
	basicstyle={\footnotesize\ttfamily}, % Estilo básico para el texto
	%stringstyle=\textsl,        % Estilo para las cadenas
	stringstyle={\color{Red1}\ttfamily\bfseries},
	commentstyle={\color{Green4}\sffamily\bfseries},% Estilo para los comentarios
	keywordstyle={\color{Blue1}\bfseries},% Estilo para las palabras clave
	%	keywordstyle=[1]\textbf,    % Posibilidad de particularizar el estilo 
	%	keywordstyle=[2]\textbf,    %
	%	keywordstyle=[3]\textbf,    %
	%	keywordstyle=[4]\textbf,    %
	%	deletekeywords={}, 			% Quita keywords separadas por comas
	captionpos=t,               % Ajusta la posición de títulos 
	numbers=left,               % Posición de números de línea
	numberstyle={\color{DarkSlateGray}\tiny\sffamily\bfseries},          % Tamaño del número de línea
	numberfirstline=false,
	firstnumber=1, 				%  Nº de la primera línea
	stepnumber=1,               % Paso de línea numerada
	numbersep=10pt,             % Separación del texto al número de línea
	tabsize=2,                  % Tamaño del tabulador
	extendedchars=true,         % Gestiona en empleo de caracteres 	
	%	extendedchars=\true,        % Gestiona en empleo de caracteres extendidos utf8 % Hack-
	texcl=true,				    % Necesario para unicode en los comentarios
	breaklines=true,            % Ajusta división automática de líneas
	breakatwhitespace=true,     % Indica si la división automática sólo ocurre en los espacios en blanco
	frame=single,               % none, leftline, topline, bottomline, lines, single, shadowbox 
	frameround=tttt, 			% Redondea los bordes del frame
	rulecolor={\color{DarkSlateGray}},    % Color del frame
	showspaces=false,           % Muestra espacios en blanco
	showtabs=false,             % Muestra tabuladores
	showstringspaces=true,      % Muestra espacios en blanco en las cadenas        
	xleftmargin=1cm,xrightmargin=1cm,
	breaklines=true,
	postbreak=\mbox{\textcolor{red}{$\hookrightarrow$}\space}, % Flecha al saltar de linea
	prebreak=\mbox{\textcolor{red}{$\hookleftarrow$}\space}, % Flecha al saltar de linea	%	framexleftmargin=17pt
	%	framexrightmargin=5pt,
	%	framexbottommargin=4pt,
%	backgroundcolor={\color{Lavender}} % Color del fondo
	backgroundcolor={\color{FloralWhite}} % Color del fondo
}

% Algunos ejemplos de definiciones de estilos.
% Para ajustar los colores se recomienda ehcar un vistazo al manual de xcolor donde están todos los colores predefinidos con sus nombres.


\lstdefinestyle{consola}{%
	basicstyle={\color{White}\scriptsize\bf\ttfamily},
	backgroundcolor={\color{Black}},
	frame=none,
	showspaces=true
}

%\lstdefinestyle{consola}{%
%	basicstyle=\scriptsize\bf\ttfamily
%}
  
 
\lstdefinestyle{C-ruled}{%
	language=C,
	frame=L,
	rulesep=.1pt,
	rulecolor=\color{black}
}


\lstdefinestyle{Python-color}{%
	language=Python,
	basicstyle=\scriptsize,
	otherkeywords={self},          
	keywordstyle=\bfseries\color{NavyBlue},
	emphstyle=\bfseries\color{DarkRed},    
	stringstyle=\color{ForestGreen}
}


\lstdefinestyle{C-color}{%
  	breaklines=true,
  	language=C,
  	basicstyle=\scriptsize,
  	keywordstyle=\bfseries\color{green!40!black},
  	commentstyle=\itshape\color{purple!40!black},
  	identifierstyle=\color{blue},
  	stringstyle=\color{orange}
}


\lstdefinestyle{CSharp}{%
	basicstyle=\scriptsize
	language=[Sharp]C,
	escapeinside={(*@}{@*)},
	keywordstyle=\bfseries
}
	

\lstdefinestyle{C++-color}{%
  	breaklines=true,
  	language=C++,
  	basicstyle=\scriptsize,
  	keywordstyle=\bfseries\color{green!40!black},
  	commentstyle=\itshape\color{purple!40!black},
  	identifierstyle=\color{blue},
  	stringstyle=\color{orange}
}
    
	
\lstdefinestyle{PHP-color}{%
	basicstyle=\scriptsize,
	language=PHP,
	keywordstyle    = \color{DarkBlue},
  	stringstyle     = \color{red},
  	identifierstyle = \color{DarkGreen},
  	commentstyle    = \color{gray},
  	emph            =[1]{php},
  	emphstyle       =[1]\color{black},
  	emph            =[2]{if,and,or,else},
  	emphstyle       =[2]\color{Gold}
}
  
	
\lstdefinestyle{Latex-color}{%
	language=[LaTeX]{Tex},
    basicstyle=\scriptsize,
    commentstyle=\color{DarkGreen},
    identifierstyle=\color{black},
    literate={\$}{{{\bfseries\color{Dandelion}\$}}}1, % Colorea el simbolo dollar
    alsoletter={\\,*,\&},
    emph =[1]{\\begin,\\end,\\caption,\\label,\\centering,\\FloatBarrier,
              \\lstinputlisting,\\scalefont,\\addplot,\\input,
              \\legend,\\item,\\subitem,\\includegraphics,\\textwidth,
              \\section,\\subsection,\\subsubsection,\\paragraph,
              \\cite,\\citet,\\citep,\\gls,\\bibliographystyle,\\url,
              \\citet*,\\citep*,\\todo,\\missingfigure,\\footnote},
  	emphstyle =[1]\bfseries\color{RoyalBlue},
  	emph = [2]{equation,subequations,eqnarray,figure,subfigure,
  			   condiciones,flalign,tikzpicture,axis,lstlisting,
  			   itemize,description
  			   },
  	emphstyle =[2]\bfseries,
    numbers=none
}

\lstdefinestyle{Java-color}{%
	basicstyle=\scriptsize,
	language=Java,
  	keywordstyle=\color{blue},
  	commentstyle=\color{DarkGreen},
  	stringstyle=\color{DarkOrchid}
}


\lstdefinestyle{R-color}{%
	language=R,                     
  	basicstyle=\scriptsize,
  	keywordstyle=\bfseries\color{RoyalBlue}, 
  	commentstyle=\color{YellowGreen},
  	stringstyle=\color{ForestGreen}  
}


\lstdefinestyle{Matlab-color}{%
	basicstyle=\scriptsize,
	language=Matlab,
  	keywordstyle=\color{blue},
  	commentstyle=\color{DarkGreen},
  	stringstyle=\color{DarkOrchid}
}


% OJO: Definición de dos macros para el tratamiento correcto de comentarios con % en Matlab
% (sólo necesario si se desea incluir código Matlab)
\makeatletter
\def\spanishplainpercent{\let\es@sppercent\@empty}
\def\spanishpercent{\def\es@sppercent{\unskip\textormath{$\m@th\,$}{\,}}}
\makeatother



\usepackage{tikz} % Paquete especializado en gráficos
\usetikzlibrary{shadows} % Necesario para poder crear nuevo comando de indicación de pulsación de tecla.


% Añade un comando para crear indicaciones de pulsación de teclas
% Esta definición sólo requiere el paquete tikz pero queda mucho más completo si se emplea el paquete "menukeys"
\newcommand*\tecla[1]{%   
	\tikz[baseline=(key.base)]
	\node[%
	draw,
	fill=white,
	drop shadow={shadow xshift=0.25ex,shadow yshift=-0.25ex,fill=black,opacity=0.75},
	rectangle,
	rounded corners=2pt,
	inner sep=1pt,
	line width=0.5pt,
	font=\scriptsize\sffamily
	](key) {#1\strut}
	;
}

%--- OPT.: Paquete para incluir menús, paths y teclas de modo "elegante"
\usepackage[os=win]{menukeys} 
% OJO: Este paquete presenta algunas incompatibilidades, debe cargarse el 
% último (manejar con cuidado).

% Estas definiciones permiten cambiar el estilo de los elementos. Si se desean otros estilos o su configuación es preciso recurrir a la documentación del paquete (no lo recomiendo).
\renewmenumacro{\menu}[:]{menus} % OPT.: default: menus
\renewmenumacro{\directory}[/]{pathswithblackfolder} % OPT.: default: paths
\renewmenumacro{\keys}[+]{shadowedroundedkeys} % OPT.: default: roundedkeys
%---


\usepackage[margin=10pt,labelfont=bf]{caption} % Configuración de caption en objetos float


% Con estas instrucciones se ajustan los valores del índice
\setcounter{secnumdepth}{2} % Ajusta el valor del último nivel numerado
\setcounter{tocdepth}{2} %Ajusta el valor del último nivel que aparece en TOC


\author{Jesús Salido}
\title{Inclusión de listados de código y algoritmos con \LaTeX{}}
\date{\today}

%%%%%%%%%%%%%%
% Comienzo del documento
%%%%%%%%%%%%%%
\begin{document}

\maketitle

\begin{abstract}
	Ejemplos de aplicación de los paquetes \texttt{listings} y \texttt{algorithm2e} para incluir listados de código y descripción de algoritmos en documentos preparados con  \LaTeX{}.
\end{abstract}


% Personalización de los títulos de los listados y del índice de listados
\renewcommand{\lstlistlistingname}{Índice de listados} % Nombre asignado
\renewcommand{\lstlistingname}{Listado} % Nombre que figura en el pie

\tableofcontents
\lstlistoflistings
\listofalgorithms


\section{Listados de programas}
En el caso de las titulaciones técnicas es muy habitual tener que explicar en el texto en preparación (p.~ej.\ un TFG/PFC o una Tesis, etc.) alguna porción de código fuente (p.~ej.\ algoritmo, función, etc.).\footnote{La inclusión de código en el texto debe estar justificada pues los listados exhaustivos deben dejarse para un CD que acompañe a la documentación, nunca deben incluirse <<tal cual>> en un documento.} Para facilitar la tarea de escribir código fuente \LaTeX{} proporciona el entorno \texttt{verbatim} para imprimir texto <<tal cual>> se escribe en el fichero de entrada. Sin embargo, este entorno es muy limitado y para ello se proporcionan paquetes que aumentan las posibilidades a la hora de tratar con el texto <<tal cual>> para que su aspecto final sea más profesional y flexible. Los dos paquetes que conviene mencionar aquí son: \texttt{listings} y \texttt{fancyvrb}.


\subsection{Listados de código con el paquete \texttt{listings}}
El paquete \texttt{listings}\footnote{\url{https://osl.ugr.es/CTAN/macros/latex/contrib/listings/listings.pdf}} está pensado para tratar especialmente con código fuente. En este caso se reconoce el lenguaje\footnote{Se reconoce un número muy amplio de lenguajes.} en que está escrito el código y ésto condiciona el modo de impresión del código (véase el listado~\ref{lst:java}). Este paquete tiene mucha flexibilidad y permite tratar con los listados de código como si fueran objetos deslizantes, de modo similar a como se tratan las figuras y las tablas. Una consecuencia de esto es que los listados no quedan divididos entre páginas. Por supuesto se admiten muchas de las opciones disponibles para los objetos deslizantes en \LaTeX{} como referencias cruzadas índice de elementos, etc. El número de opciones es tan numeroso que su comentario excede el propósito del curso por lo que se recomienda la consulta de la documentación del paquete a aquellos que estén más interesados.


% Ejemplo:
% ============
\begin{lstlisting}[language=Java,style=Java-color,float=ht,caption={[Código fuente en Java]Ejemplo de código fuente en lenguaje Java},label=lst:java]
// @author www.javadb.com
public class Main {    
  // Este método convierte un String a
  // un vector de bytes

   public void convertStringToByteArray() {
        
     String stringToConvert = "This String is 15";      
     byte[] theByteArray = stringToConvert.getBytes();        
     System.out.println(theByteArray.length);        
   }
    
   // argumentos de línea de comandos 
   public static void main(String[] args) {
     new Main().convertStringToByteArray();
   }
}
\end{lstlisting}

Un aspecto a tener presente cuando se trabaja con el paquete \texttt{listings} es que no funciona completamente bien cuando se emplea codificación unicode (UTF8) y el texto del entorno \texttt{lstlisting} tiene caracteres especiales (acentos, interrogación, etc.). Esto es así porque este paquete sólo entiende codificación de caracteres con un byte y por tanto incapaz de reconocer los caracteres unicode. En estos casos se pueden adoptar varias soluciones: 

\begin{enumerate}
	\item Emplear codificación UTF8 e indicar mediante la opción \texttt{literate} de la definición \texttt{lstset} cómo deben sustituirse los caracteres <<extraños>>. [Opción preferida]
	\item Trabajar con codificación UTF8 (opción \texttt{utf8} para el paquete \texttt{inputenx}) e incluir en la definición \texttt{lstset} la opción \texttt{extendedchars=true} y \texttt{texcl=true}. De este modo se reconocen los caracteres extendidos pero aún así se presentan errores con algunos de ellos.
	\item Trabajar con codificación ANSI (opción \texttt{ansinew} o \texttt{latin1}) y emplear los caracteres que se desee.
\end{enumerate}

En los siguientes ejemplos veremos varios listados generados empleando el paquete \texttt{listings}. El listado~\ref{lst:java} muestra el resultado cuando se emplean unas opciones asociadas a un estilo creado específicamente para código fuente Java. También ilustra cómo es manejado el código como un elemento deslizante \emph{(float)} y cómo se puede incluir incluso referencias cruzadas.

En los ejemplos siguientes se muestra el código fuente de un pequeño programa C para el que se emplea un ajuste diferente de los atributos del entorno \texttt{lstlisting}. Para configurar aspectos relacionados con el color se recomienda emplear los colores predefinidos por el paquete \texttt{xcolor}.\footnote{\url{https://osl.ugr.es/CTAN/macros/latex/contrib/xcolor/xcolor.pdf}}

\begin{lstlisting}[style=C-color,float=ht,caption={Ejemplo de código C},label=lst:codC]
// Este código se ha incluido tal cual está 
// en el fichero \LaTeX{}
#include <stdio.h>
int main(int argc, char* argv[]) {
  puts("¡Hola mundo y España!");
}
\end{lstlisting}



\noindent Por el contrario este otro se ha generado incluyendo el código desde un fichero externo \texttt{``HolaMundo.c''}.

\lstinputlisting[style=C-color,float=ht,caption={Ejemplo de código C desde un fichero externo},label=lst:codCfile]{../code/HolaMundo.c}


También es posible configurar un estilo específico para indicar comandos de consola del computador, como en el siguiente ejemplo dónde se señala cómo compilar un fichero usando \texttt{gcc} (detener pulsando \tecla{Ctrl+d}): %Aquí se muestra cómo incluir en un manual la pulsación de teclas

\begin{lstlisting}[style=Consola, numbers=none]
$ gcc -o Hola HolaMundo.c
\end{lstlisting}


Aquí se incluye un listado configurado con un estilo que emplea más el color y el  sangrado del código para que quede más elegante.

\lstinputlisting[language=Python,style=Python-color, float=ht,caption={Ejemplo de código Python desde un fichero externo},label=lst:pyfile]{../code/pyexample.py}

La inclusión de código en Matlab presenta algunas particularidades en idioma español ya que \texttt{babel} modifica el espacio que separa el signo `\%'. Por este motivo es preciso incluir las macros especiales \verb|\spanishplainpercent| y \verb|\spanishpercent| para que dicho carácter sea tratado correctamente. El ejemplo~\ref{lst:matlab} muestra un listado correspondiente a código Matlab.

\spanishplainpercent
\begin{lstlisting}[style=Matlab-color,float=ht,caption={Ejemplo escrito en Matlab},label=lst:matlab]
function f = fibonacci(n)
 % FIBONACCI  Fibonacci sequence
 %	f = FIBONACCI(n) generates the first n Fibonacci numbers.
 %	Copyright 2014 Cleve Moler
 %	Copyright 2014 The MathWorks, Inc.
f = zeros(n,1); 
f(1) = 1;
f(2) = 2;
for k = 3:n
f(k) = f(k-1) + f(k-2);
end
\end{lstlisting}
\spanishpercent

\subsection{Algoritmos con el paquete \texttt{algorithm2e}}
Como ya se ha comentado en los textos científicos relacionados con las TIC\footnote{Por supuesto en un TFG o tesis de una Escuela de Informática.} (Tecnologías de la Información y Comunicaciones) suelen aparecer porciones de código en los que se explica alguna función o característica relevante del trabajo que se expone. Muchas veces lo que se quiere ilustrar es un algoritmo o método en que se ha resuelto un problema abstrayéndose del lenguaje de programación concreto en que se realiza la implementación. El paquete \texttt{algorithm2e}\footnote{\url{https://osl.ugr.es/CTAN/macros/latex/contrib/algorithm2e/doc/algorithm2e.pdf}} proporciona un entorno \texttt{algorithm} para la impresión apropiada de algoritmos tratándolos como objetos flotantes y con muchas flexibilidad de personalización. En el algoritmo \ref{alg:como} se muestra cómo puede emplearse dicho paquete. En este curso no se explican las posibilidades del paquete más en profundidad ya que excede el propósito del curso. A todos los interesados se les remite a la documentación del mismo.


% Ejemplo:
% ============
\IncMargin{1em}
\begin{algorithm}
\SetKwInOut{Input}{Datos}\SetKwInOut{Output}{Resultado}
\LinesNumbered
\SetAlgoLined

\Input{este texto} 
%\KwIn{este texto}
\Output{como escribir algoritmos con \LaTeX2e}
%\KwOut{como escribir algoritmos con \LaTeX2e}

inicialización\;
\While{no es el fin del documento}{
	leer actual\;
	\eIf{comprendido}{
		ir a la siguiente sección\;
		la sección actual es esta\;
	}{
		ir al principio de la sección actual\;
	}
}

% Aunque el captión aparece abajo siempre se pone arriba como en tablas y listados
\caption{Cómo escribir algoritmos}\label{alg:como}
\end{algorithm}\DecMargin{1em}


\subsection{Menús, paths y teclas con el paquete \texttt{menukeys}}
Cada vez es más usual que los trabajos en ingeniería exijan el uso de 
software. Para poder especificar de modo elegante el uso menús, pulsación de 
teclas y directorios se recomienda el uso del paquete 
\texttt{menukeys}.\footnote{\url{https://osl.ugr.es/CTAN/macros/latex/contrib/menukeys/menukeys.pdf}}
 \index{CTAN} Este paquete nos permite especificar el acceso a un menú, por 
ejemplo:\\

\noindent \menu{Herramientas:Órdenes:PDFLaTeX}\\

\noindent También un conjunto de teclas. Por ejemplo:
\keys{\ctrl + \shift + T}\\

\noindent O un directorio:
\directory{C:/user/LaTeX/Ejemplos}\\

\noindent Aunque este paquete permite muchas opciones de configuración de los estilos aplicados, no es necesario hacerlo para obtener unos resultados muy elegantes.



\section{Licencias de protección de propiedad intelectual}
Existen paquetes especializados para facilitar la utilización de licencias de protección de propiedad intelectual entre las que destacan las de tipo Creative Commons. Para tratar con dichas licencias en los documentos preparados con \LaTeX{} se recomiendan los paquetes:
\begin{itemize}
\item \texttt{ccicons}\footnote{\url{https://osl.ugr.es/CTAN/fonts/ccicons/ccicons.pdf}} que está especializado en la generación de los iconos de los distintos tipos de licencias, como por ejemplo: \ccbyncsa.

\item \texttt{doclicense}\footnote{\url{https://osl.ugr.es/CTAN/macros/latex/contrib/doclicense/doclicense.pdf}} que proporciona un conjunto importante de macros para generar información relativa a las licencias de tipo Creative Commons.
\end{itemize}

\doclicenseThis % Generación automática de la licencia


\end{document}

