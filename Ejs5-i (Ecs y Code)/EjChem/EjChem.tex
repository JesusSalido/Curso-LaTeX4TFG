\documentclass[11pt,a4paper]{article}
% Por defecto todas las ecuaciones quedan centradas en la página pero con la opción fleqn se consigue su alineación a la izda.
% \documentclass[11pt,a4paper,fleqn]{article}
\usepackage[spanish,es-noindentfirst]{babel} 
\usepackage[left=2cm,right=2cm,top=2cm,bottom=2cm]{geometry} % Márgenes 

%--- Tipografía
\usepackage{amsmath,amssymb,amsfonts}	% Símbolos matemáticos
%--- Tipografía (Opción 1)
\usepackage{newpxtext}	% La opción osf proporciona números en old style.
\usepackage{newpxmath}
%---
%--- Tipografía (Opción 2)
%\usepackage[tt=false]{libertine}
%\usepackage[libertine]{newtxmath}
%---

\usepackage{marvosym,pifont,textcomp,fontawesome5}

\usepackage[T1]{fontenc} % Codificación de salida    
\usepackage[
    protrusion=true,
    activate={true,nocompatibility},
    final,
    tracking=true,
    kerning=true,
    spacing=true,
    factor=1100]{microtype}
\SetTracking{encoding={*}, shape=sc}{40}


% Generación de hiperenlaces
\usepackage[%
   pdftex,
   breaklinks,
   hidelinks=true,      % Oculta colores en los enlaces (negro)
    linktocpage=true,    % true = enlace al nº de pág., false=texto completo
%    colorlinks=true,         % true=colorea texto del enlace, false=recuadra el texto
	citecolor=red, % Color de la citas
	urlcolor=blue, % Color de las URL
	bookmarksnumbered=true % Incluye números en bookmarks
]{hyperref}
\usepackage{url}
\urlstyle{sf} % Estilo de URL sin serifas


% Listas
\usepackage{enumitem} % Mayor control de listas
\usepackage{multicol} % Elementos en varias columnas


% Gráficos
\usepackage{graphicx}  % Inclusión de figuras y escalado de cajas
\usepackage{float}
% Declaración del path donde están los archivos de figuras. 
% También se puede incluir el path en el nombre del fichero.
\graphicspath{{../figs/}}  
\DeclareGraphicsExtensions{.pdf,.png,.jpg}
% Lista de extensiones de ficheros por orden de precedencia. De este modo no hace falta indicar la extensión del fichero y en caso de existir dos fichero con el mismos nombre y extensión diferente se emplea el que tiene una extensión con mayor prioridad.


\usepackage{caption} % Configuración de caption en objetos float

% Paquetes especiales para Química
\usepackage[version=3]{mhchem}
\usepackage{chemfig}


\author{Jesús Salido}
\title{Composición básica de formulación química}
\date{\today}

%%%%%%%%%%%%%%
% Comienzo del documento
%%%%%%%%%%%%%%
\begin{document}

\maketitle

\begin{abstract}
	Explicación breve sobre cómo se componen fórmulas químicas con \LaTeX{}.
\end{abstract}

\section{Fórmulas químicas}
\LaTeX{} también puede emplearse para la inclusión de fórmulas y estructuras químicas. Para ello se proporciona un sinfín de paquetes que pueden ayudar en la tarea.\footnote{\url{http://www.mychemistry.eu/known-packages/}} A continuación se muestran ejemplos creados con los paquetes \texttt{mhchem} y \texttt{chemfig}.



\subsection{Fórmulas químicas con el paquete \texttt{mhchem}}
Es un paquete bastante sencillo que permite escribir formulación química simple.

\begin{center}
\ce{Zn^2+
<=>[\ce{+ 2OH-}][\ce{+ 2H+}]
$\underset{\text{amphoteres Hydroxid}}{\ce{Zn(OH)2 v}}$
<=>C[+2OH-][{+ 2H+}]
$\underset{\text{Hydroxozikat}}{\cf{[Zn(OH)4]^2-}}$
}
\end{center}


A diferencia de las ecuaciones matemáticas no existe un entorno que genere la ecuación con un título y un tratamiento similar al de los objetos flotantes por lo que si se requiere este tratamiento hay que configurarlo.

Al igual que se puede hacer con las ecuaciones matemáticas a las fórmulas químicas se les puede añadir un recuadro como muestra el ejemplo siguiente:

\begin{center}
{\fboxsep 8pt \fboxrule 0.5pt
	\fbox{
			\ce{Zn^2+
			<=>[\ce{+ 2OH-}][\ce{+ 2H+}]
			$\underset{\text{amphoteres Hydroxid}}{\ce{Zn(OH)2 v}}$
			<=>C[+2OH-][{+ 2H+}]
			$\underset{\text{Hydroxozikat}}{\cf{[Zn(OH)4]^2-}}$
			}
	}
}
\end{center}


\subsection{Fórmulas químicas con el paquete \texttt{chemfig}}
Es un paquete que aprovecha las capacidades gráficas del paquete \texttt{tikz} (Ti\textit{k}Z)\footnote{Si no se ha cargado previamente \texttt{chemfig} lo cargará.} y es muy flexible. Sin embargo, no hay que perder de vista que aprender a utilizarlo ya puede representar un esfuerzo importante. En este caso una solución alternativa es emplear programas dedicados de dibujo de fórmulas químicas importándolas en el documento \LaTeX{} como un fichero PDF.

A continuación se muestran algunos ejemplos de fórmulas químicas empleando el paquete \texttt{chemfig}. En la Fig.~\ref{fig:cafeina} se muestra como es posible tratar como figura (objeto \emph{float}) una fórmula generada con \texttt{chemfig}.


% Ejemplo: Cafeina
% ============
\begin{figure}[H]
	\begin{center}
		\chemfig{H_3C-[:30]N**6(-(=O)-(**5(-N(-CH_3)--N-))--N(-CH_3)-(=O)-)}
	\end{center}
	\caption{Fórmula química de la cafeina}\label{fig:cafeina}
\end{figure}


Y un ejemplo más mostrando la fórmula de la adrenalina:
% Ejemplo: Adrenalina
% ============
\begin{center}
\definesubmol{&}{-[,,,,draw=none]}
\definesubmol{&&}{-[,,,2,draw=none]}
\chemfig{*6((-HO)-=*6(!&!{&&}HN(-CH_3)-[,,2]-(<OH)-)-=-(-HO)=)}
\end{center}
\end{document}
