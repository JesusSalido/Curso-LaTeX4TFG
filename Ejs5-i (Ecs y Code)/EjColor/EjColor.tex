% La generación de caracteres especiales con LaTeX

%%%%%%%%%%%%%%
% Preámbulo del documento
%%%%%%%%%%%%%%
\documentclass[11pt,a4paper]{article} 
\usepackage[utf8]{inputenx}	% Codificación de entrada
\usepackage[spanish]{babel}	% Idioma del texto
\usepackage[left=2cm,right=2cm,top=2cm,bottom=2cm]{geometry} % Márgenes 

% Tipografía
\usepackage{newpxtext}
\usepackage{newpxmath}

\usepackage{marvosym}
\usepackage{pifont} % Generación de símbolos especiales
\usepackage{textcomp}


\usepackage[T1]{fontenc} % Codificación de salida    
\usepackage{microtype} % Mejoras de microtipografía en la obtención de PDF (sólo para pdflatex)

\usepackage{url} % Para escritura de URL
\urlstyle{sf}


% Ejemplos de definición de color
\usepackage[usenames,dvipsnames,svgnames,x11names,table]{xcolor}
\definecolor{palered}{rgb}{0.78, 0.03, 0.08} % Elección propia
\definecolor{UCLMred}{cmyk}{0.0, 1.0, 0.65, 0.28} % UCLM red
%\definecolor{UCLMred}{HTML}{B30033} % UCLM red


% Generación de hiperenlaces
\usepackage[pdftex,breaklinks,colorlinks,
citecolor=blue, % Color de la citas
urlcolor=blue, % Color de las URL
bookmarksnumbered=true, % Incluye números en bookmarks
pdftitle={Fundamentos de LaTeX para principiantes},
pdfauthor={Jesús Salido},
pdfsubject={LaTeX}]{hyperref}



%%%%%%%%%%%%%%
% Comienzo del documento
%%%%%%%%%%%%%%
\begin{document}
Este es un ejemplo de texto con propiedades de color. Veamos algunos ejemplos:

% Ejemplo:
% ============
\textcolor{red}{Este texto está a \textbf{color}.}

% Ejemplo:
% ============
También se puede hacer {\color{green} con una definición alternativa.}

% Ejemplo:
% ============
\colorbox{blue}{Se pueden crear cajas de {\color{yellow}\textbf{color}}.}

La especificación de color de \LaTeX{} con el paquete \texttt{xcolor} es muy potente y permite la definición de colores propios en varios esquemas de color (gray, rgb, RGB, HTML y cmyk). De todas ellas quizás la única que sirva a nuestros propósitos sea la especificación en niveles de gris pues el color es un recurso que sólo en contadas ocasiones será preciso.

% Ejemplo:
% ============
\definecolor{sombra}{gray}{0.75}
\colorbox{sombra}{Texto resaltado con una caja sombreada.}

% Ejemplo:
% ============
\setlength{\fboxrule}{4pt}
\fcolorbox{sombra}{white}{Otro texto resaltado.}

% Ejemplo:
% ============
\setlength{\fboxrule}{0.5pt} % Ajustamos el ancho del filete de la caja.
\begin{center}
\fbox{\colorbox{yellow}{
\parbox{0.5\linewidth}{Esto es
	un ejemplo de lo que puede
	hacerse de manera sencilla en
	este estupendo
	procesador de textos.}}}
\end{center}



Aquí se ven algunas otras posibilidades.

%-------------------
\newcount\WL \unitlength.75pt
\begin{picture}(460,60)(355,-10)
\sffamily \tiny \linethickness{1.25\unitlength} \WL=360
\multiput(360,0)(1,0){456}%
{{\color[wave]{\the\WL}\line(0,1){50}}\global\advance\WL1}
\linethickness{0.25\unitlength}\WL=360
\multiput(360,0)(20,0){23}%
{\picture(0,0)
\line(0,-1){5} \multiput(5,0)(5,0){3}{\line(0,-1){2.5}}
\put(0,-10){\makebox(0,0){\the\WL}}\global\advance\WL20
\endpicture}
\end{picture}
%--------------------------

\end{document}