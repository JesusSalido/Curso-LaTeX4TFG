%%%%%%%%%%%%%% 
% Fichero: EjListas
% Autor: J. Salido (http://www.esi.uclm.es/www/jsalido)
% Fecha: febrero, 2017
% Descripción: Generación de listas en LaTeX
% Ejemplo del curso: “LaTeX esencial para preparación de TFG, Tesis
% y otros documentos académicos” (Esc. Sup. Informática-UCLM)
%%%%%%%%%%%%%%

% Atención: En la última versión de babel con idioma español se cambia la forma de los bullets en los entornos itemize.


%%%%%%%%%%%%%%
% Preámbulo del documento
%%%%%%%%%%%%%%
\documentclass[11pt,a4paper]{article} 
\usepackage[utf8]{inputenx} 
\usepackage[spanish]{babel} 
\usepackage[left=2cm,right=2cm,top=2cm,bottom=2cm]{geometry}

\usepackage{newpxtext}
\usepackage{newpxmath}
\usepackage{marvosym}
\usepackage{pifont} % Generación de símbolos especiales
\usepackage{textcomp}
\usepackage[T1]{fontenc} % Codificación de salida    
\usepackage{microtype} % Mejoras de microtipografía en la obtención de PDF (sólo para pdflatex)
\usepackage{hyperref} % Para escritura de URL
\urlstyle{sf} % Estilo de URL sin serifas para que tengan un mejor aspecto

% Paquetes para obtener un mayor control de las listas
\usepackage{paralist} % Mayor control de listas
\usepackage{multicol} % Elementos en varias columnas


\author{Jesús Salido}
\title{Listas en \LaTeX{}}
\date{\today}
%%%%%%%%%%%%%%
% Comienzo del documento
%%%%%%%%%%%%%%
\begin{document}

\maketitle

\begin{abstract}
	Explicación breve sobre cómo emplear en \LaTeX{} algunos entornos relacionados con las listas, justificación y notas al margen. Este documento contiene varios ejemplos de listas con los que se puede experimentar empleando la vista previa disponible en IDE como \TeX studio. 
\end{abstract}


Las listas se emplean cuando se desea enumerar una serie características, objetos, etc. A continuación veremos algunos ejemplos de listas.

\noindent Ejemplo de entorno {\tt itemize}:

% Ejemplo: Lista con balas
% ============
\begin{itemize}
	\item peras
	\item manzanas
	\item limones
	\item naranjas
\end{itemize}


Las listas también se pueden anidar:
% Ejemplo: Lista anidada con balas
% ============
\begin{itemize}
	\item peras
	\begin{itemize}
		\item conferencia
		\item ercolina
	\end{itemize}
	\item manzanas
	\begin{itemize}
		\item granny
		\item golden 
	\end{itemize}
	\item naranjas
\end{itemize}



\noindent Ejemplo de entorno {\tt itemize} en el que se sustituye el símbolo (\emph{bullet}) por defecto:

% Ejemplo:
% ============
\begin{itemize}
	\item[*] peras
	\item manzanas
	\item[\ding{170}] naranjas
\end{itemize}


Personalmente pienso que \LaTeX{} proporciona demasiado espaciado en las listas. Pero con el paquete \texttt{paralist} se puede compactar las listas. A continuación se muestra un ejemplo de entorno {\tt compactitem} (proporcionado por el paquete \texttt{paralist}):

% Ejemplo:
% ============
\begin{compactitem}
	\item[*] peras
	\item manzanas
	\item[\ding{170}] naranjas
\end{compactitem}



\bigskip % Salto vertical equivalente a 1 línea (\medskip: 1/2 línea, \smallskip: 1/4 línea)

\noindent Ejemplo de entorno {\tt enumerate} para listas numeradas:

% Ejemplo:
% ============
\begin{enumerate}
	\item peras
	\item manzanas
	\item naranjas
\end{enumerate}


\newpage

\noindent Ejemplo de entorno {\tt compactenum} (proporcionado por el paquete \texttt{paralist}) para listas numeradas:

% Ejemplo:
% ============
\begin{compactenum}
	\item peras
	\item manzanas
	\item naranjas
\end{compactenum}


\bigskip

\noindent Este es otro ejemplo de línea compacta creada en varias columnas con ayuda de los paquetes \texttt{paralist} y \texttt{multicol}:
% Ejemplo:
% ============
\begin{multicols}{2} % El parámetro es el número de columnas de la lista
	\begin{compactenum}
		\item peras
		\item manzanas
		\item naranjas
		\item patatas
		\item calabazas
		\item fresas
	\end{compactenum}
\end{multicols}


\noindent Las listas incluso se pueden personalizar (aunque debería estar justificado el cambiar el estilo por defecto, haciéndolo con mucha prudencia):

% Ejemplo:
% ============
\begin{dinglist}{52} % Lista en la que se cambia el bullet por un símbolo dingbat
	\item peras
	\item manzanas
	\item naranjas
\end{dinglist}


\noindent Así también:

% Ejemplo:
% ============
\begin{dingautolist}{202} % Lista con símbolos sucesivos de la tabla dingbat
	\item peras
	\item manzanas
	\item naranjas
\end{dingautolist}


\noindent y por supuesto anidar:

% Ejemplo:
% ============
\begin{enumerate}
  \item Cítricos
  \begin{enumerate}
    \item Limón
    \item Naranja
  \end{enumerate}
  \item Legumbres
  \item Hortalizas \ldots
\end{enumerate}


\noindent Ejemplo de entorno {\tt description} para listas:

% Ejemplo:
% ============
\begin{description}
	\item[Estupideces] no mejoran por ponerlas en una lista por bonita que esta sea. Aunque la mona se vista de seda, mona se queda.
	\item[Lucideces] sin embargo, pueden parecer hermosas en
	una lista.
\end{description}

\noindent El ejemplo anterior formateado de otro modo:

% Ejemplo:
% ============
\begin{description}
	\item[Estupideces]\hfill \\ 
			no mejoran por ponerlas en una lista por bonita que esta sea. Aunque la mona se vista de seda, mona se queda.
	\item[Lucideces]\hfill \\ 
			sin embargo, pueden parecer hermosas en una lista.
\end{description}

\LaTeX{} permite una configuración más avanzada para presentar listas más sofisticadas, aunque la mayoría de las veces esto no es necesario, ya que es suficiente con las listas convencionales.



\end{document}

