%%%%%%%%%%%%%% 
% Fichero: EjListas
% Autor: J. Salido (http://www.esi.uclm.es/www/jsalido)
% Fecha: febrero, 2017
% Rev.: febrero 2022
% Descripción: Generación de listas en LaTeX
% Ejemplo del curso: “LaTeX esencial para preparación de TFG, Tesis
% y otros documentos académicos” (Esc. Sup. Informática-UCLM)
%%%%%%%%%%%%%%

% Atención: En la última versión de babel con idioma español se cambia la forma de los bullets en los entornos itemize.


%%%%%%%%%%%%%%
% Preámbulo del documento
%%%%%%%%%%%%%%
\documentclass[11pt,a4paper]{article} 
\usepackage[spanish]{babel} 
\usepackage[left=2cm,right=2cm,top=2cm,bottom=2cm]{geometry}

\usepackage{newpxtext}
\usepackage{newpxmath}
\usepackage{marvosym}
\usepackage{pifont} % Generación de símbolos especiales
\usepackage{textcomp}
\usepackage{fontawesome5}
\usepackage[T1]{fontenc} % Codificación de salida    
% OJO: Nuevas versiones con problemas
\usepackage{microtype} % Mejoras de microtipografía en la obtención de PDF (sólo para pdflatex)

% Paquetes para obtener un mayor control de las listas
\usepackage{multicol} % Elementos en varias columnas
\usepackage[shortlabels,inline]{enumitem} % Paquete para configurar listas

\author{Jesús Salido}
\title{Listas en \LaTeX{}}
\date{\today}
%%%%%%%%%%%%%%
% Comienzo del documento
%%%%%%%%%%%%%%
\begin{document}

\maketitle

\begin{abstract}
Explicación breve sobre cómo emplear en \LaTeX{} algunos entornos relacionados con las listas, justificación y notas al margen. Este documento contiene varios ejemplos de listas con los que se puede experimentar empleando la vista previa disponible en IDE como \TeX studio. 
\end{abstract}


Las listas se emplean cuando se desea enumerar una serie de características, objetos, etc. A continuación veremos algunos ejemplos de listas.

\noindent Ejemplo de entorno {\tt itemize}:

% Ejemplo: Lista con viñetas anidada
% ============
\begin{itemize}
	\item peras
	\item manzanas
    \item cítricos
        \item limones
        \item naranjas
\end{itemize}

\noindent La misma lista cambiando la viñeta (\texttt{pifont}), gracias a paquete \texttt{enumitem}:

% Ejemplo: Lista con viñeta dingbat y un icono de fontawesome5
% ============
\begin{itemize}[\ding{68}]
\item peras
\item[\faIcon{apple}] manzanas
\item limones
\item naranjas
\end{itemize}

\noindent Otra personalización:
% Ejemplo:
% ============
\begin{itemize}[\ding{52},noitemsep] % Lista en la que se cambia el bullet por un símbolo dingbat
	\item peras
	\item[\ding{55}] manzanas
	\item naranjas
\end{itemize}

\noindent Las listas también se pueden anidar:
% Ejemplo: Lista anidada con balas
% ============
\begin{itemize}
	\item peras
	\begin{itemize}
		\item conferencia
		\item ercolina
    	\begin{itemize}
    		\item mini conferencia
    		\item extra ercolina
    	\end{itemize}
	\end{itemize}
	\item manzanas
	\begin{itemize}
		\item granny
		\item golden 
	\end{itemize}
	\item naranjas
\end{itemize}

Personalmente, pienso que \LaTeX{} proporciona demasiado espaciado en las listas. Con el paquete \texttt{enumitem} se puede compactar las listas. A continuación se muestra un ejemplo de entorno {\tt itemize} con opciones para cambiar el espaciado vertical entre ítems y cambio de viñeta por defecto (proporcionada por el paquete \texttt{enumitem}):

% Ejemplo: Cambia el espaciado entre items y la viñeta por defecto.
% ============
\begin{itemize}[\textbullet,noitemsep]
	\item peras
	\item manzanas
	\item naranjas
\end{itemize}

\noindent Ejemplo de entorno {\tt enumerate} para listas numeradas:

% Ejemplo:
% ============
\begin{enumerate}
	\item peras
	\item manzanas
	\item naranjas
\end{enumerate}


\noindent Ejemplo de entorno {\tt enumerate} con cambio de numeración (letras minúsculas) y opción \texttt{noitemsep} (proporcionada por el paquete \texttt{enumitem}) para listas numeradas:

% Ejemplo:
% ============
\begin{enumerate}[A.-,noitemsep]
	\item peras
	\item manzanas
	\item naranjas
\end{enumerate}

\noindent Este es otro ejemplo de línea compacta creada en varias columnas con ayuda de los paquetes \texttt{enumitem} y \texttt{multicol}:
% Ejemplo:
% ============
\begin{multicols}{2} % El parámetro es el número de columnas de la lista
	\begin{enumerate}[i.,noitemsep]
		\item peras
		\item manzanas
		\item naranjas
		\item patatas
		\item calabazas
		\item fresas
	\end{enumerate}
\end{multicols}


\noindent Las enumeraciones pueden personalizar de muchos modos, aunque debería estar justificado el cambiar el estilo por defecto, haciéndolo con mucha prudencia:
\noindent Así también:

% Ejemplo:
% ============
\begin{dingautolist}{182} % Lista con símbolos sucesivos de la tabla dingbat
	\item peras
	\item manzanas
	\item naranjas
\end{dingautolist}






\noindent y por supuesto anidar:

% Ejemplo:
% ============
\begin{enumerate}
  \item Cítricos
  \begin{enumerate}
    \item Limón
    \item Naranja
  \end{enumerate}
  \item Legumbres
  \item Hortalizas...
\end{enumerate}


\noindent Por último, el entorno {\tt description} se emplea para listas de términos o definiciones, como se muestra en los ejemplos siguientes:

% Ejemplo:
% ============
\begin{description}[noitemsep]
	\item[Estupideces] no mejoran por ponerlas en una lista por bonita que esta sea. Aunque la mona se vista de seda, mona se queda.
	\item[Lucideces] sin embargo, pueden parecer hermosas en
	una lista.
\end{description}

\noindent El ejemplo anterior formateado de otro modo:

% Ejemplo:
% ============
\begin{description}[style=nextline]
	\item[Estupideces] no mejoran por ponerlas en una lista por bonita que esta sea. Aunque la mona se vista de seda, mona se queda.
	\item[Lucideces] sin embargo, pueden parecer hermosas en una lista.
\end{description}

\LaTeX{} permite la configuración de listas, pero la mayoría de las veces esto no es necesario, ya que es suficiente el estilo por defecto proporcionado.
\end{document}

