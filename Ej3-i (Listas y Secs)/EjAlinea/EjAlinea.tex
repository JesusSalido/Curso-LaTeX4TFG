%%%%%%%%%%%%%% 
% Fichero: EjAlinea
% Autor: J. Salido (http://www.esi.uclm.es/www/jsalido)
% Fecha: febrero, 2017
% Descripción: Alineación de texto
% Ejemplo del curso: “LaTeX esencial para preparación de TFG, Tesis
% y otros documentos académicos” (Esc. Sup. Informática-UCLM)
%%%%%%%%%%%%%%


%%%%%%%%%%%%%%
% Preámbulo del documento
%%%%%%%%%%%%%%
\documentclass[11pt,a4paper]{report} 
\usepackage[spanish]{babel} 
\usepackage[left=2cm,right=2cm,top=2cm,bottom=2cm]{geometry}

\usepackage{newtxtext}
\usepackage{newtxmath}

\usepackage{marvosym,pifont,textcomp,fontawesome5}

\usepackage[T1]{fontenc} % Codificación de salida    
\usepackage[
    protrusion=true,
    activate={true,nocompatibility},
    final,
    tracking=true,
    kerning=true,
    spacing=true,
    factor=1100]{microtype}
\SetTracking{encoding={*}, shape=sc}{40}

\usepackage{epigraph} % Inclusión de epígrafes al comienzo de capítulos


\title{Ejemplos de alineamiento de texto con \LaTeX}
\author{Jesús Salido}
\date{\today}



%%%%%%%%%%%%%%
% Comienzo del documento
%%%%%%%%%%%%%%
\begin{document}
\maketitle

\begin{abstract}
	Con \LaTeX{} también se puede configurar la alineación de los párrafos con distintos propósitos. Aunque se trata de situaciones poco frecuentes, en este ejemplo se muestran algunos ejemplos de uso.
\end{abstract}

\chapter{Métodos de alineamiento de texto}
\epigraph{Una cosa es saber y otra saber enseñar}{\textit{Cicerón}}
Por defecto, \LaTeX{} emplea justificación completa de los párrafos. Sin embargo, en alguna ocasión puede interesar modificar este tipo de justificación. 

A continuación se muestran algunos ejemplos relacionados con la justificación de párrafos:

% Ejemplo:
% ============
\begin{flushleft}
	Este texto está alineado a la izquierda. \\
	\LaTeX{} no trata de justificar las líneas, \\ 
	así que así quedan.
\end{flushleft}

\begin{flushright}
	Texto alineado a la derecha. \\
	\LaTeX{} no trata de justificar las líneas.
\end{flushright}

\begin{center}
	En el centro\\
	de la Tierra
\end{center}


\noindent Y uno sobre el empleo de citas:


% Ejemplo:
% ============
\begin{quote}
\emph{<<
Una regla empírica tipográfica para la longitud de renglón es: En promedio, ningún renglón debería tener más de 66 signos porque así lo establece la propia experiencia.>>}
\end{quote}

\noindent Aunque también se podría haber dicho:

% Ejemplo:
% ============
\begin{quote}
\emph{<<La experiencia establece que, en promedio, ninguna línea de texto debería tener más de 66 signos. Esta y otras reglas derivadas de la experiencia se tendrán en cuenta a la hora de preparar documentos técnicos con corrección...>>}
\end{quote}

\noindent En el caso de citas que ocupan varios párrafos:

% Ejemplo:
% ============
\begin{quotation}
<<¡Oh, memoria, enemiga mortal de mi descanso!...

La virtud más es perseguida de los malos que amada de los buenos...

La ingratitud es hija de la soberbia...

La razón de la sinrazón que a mi razón se hace, de tal manera mi razón enflaquece, que con razón me quejo de la vuestra fermosura.>>
\end{quotation}

Por ello, las páginas de \LaTeX{} tienen márgenes tan anchos por omisión, y los periódicos usan múltiples columnas. 

En alguna situación puede interesar realizar un formateado del los párrafos empleando tabuladores. Su manejo se puede obtener mediante el entorno \texttt{tabbing} para conseguir la alineación deseada en los elementos del párrafo:

% Ejemplo:
% ============
% Observar que los espacios en blanco son irrelevantes
% \= fija la posición del tabulador
% \> avanza hasta la posición del tabulador
\begin{tabbing}
IF	\= \textbf{está} lloviendo                \\
    \> THEN \= \textbf{calzar} botas de agua, \\
    \>      \> \textbf{coger} paraguas;       \\
    \> ELSE \> \textbf{sonreir}.              \\
\textbf{Salir} de casa.
\end{tabbing}

\end{document}
