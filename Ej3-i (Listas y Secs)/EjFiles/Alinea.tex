\section{Alineación de textos en \LaTeX{}}
\label{sec:alinea} % En esta sección se ha incluido el contenido del documento EjAlinea.tex

\LaTeX{} emplea justificación completa de los párrafos por defecto. Pero este comportamiento puede interesarnos modificarlo. A continuación se muestran algunos ejemplos relacionados con la justificación de párrafos:

% Ejemplo:
% ============
\begin{flushleft}
	Este texto está alineado a la izquierda. \\
	\LaTeX{} no trata de justificar las líneas, \\
	así que así quedan.
\end{flushleft}

\begin{flushright}
	Texto alineado a la derecha. \\
	\LaTeX{} no trata de justificar las líneas.
\end{flushright}

\begin{center}
	En el centro\\
	de la Tierra
\end{center}

\noindent Y uno sobre el empleo de citas:

Una regla empírica tipográfica para la longitud de renglón es:

% Ejemplo:
% ============
\begin{quote}
	En promedio, ningún renglón debería tener más de 66 signos porque así lo establece la propia experiencia.
\end{quote}

Por ello las páginas de \LaTeX{} tienen márgenes tan anchos por omisión, y los periódicos usan múltiples columnas. 

En alguna situación puede interesar realizar un formateado del los párrafos empleando tabuladores. El entorno \texttt{tabbing} proporciona el uso de tabuladores para conseguir la alineación deseada en los elementos del párrafo:

% Ejemplo:
% ============
% Observar que los espacios en blanco son irrelevantes
% \= fija la posición del tabulador
% \> avanza hasta la posición del tabulador
\begin{tabbing}
	IF \= \textbf{está} lloviendo                \\
	\> THEN \= \textbf{calzar} botas de agua, \\
	\>      \> \textbf{coger} paraguas;       \\
	\> ELSE \> \textbf{sonreir}.              \\
	\textbf{Salir} de casa.
\end{tabbing}






