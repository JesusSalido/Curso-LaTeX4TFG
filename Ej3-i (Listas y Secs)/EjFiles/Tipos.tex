\section{Tipografía con \LaTeX{}}
\label{sec:tipos} % En esta sección se ha incluido el contenido del documento EjTipos.tex

\LaTeX{} genera su salida con los tipos \emph{Computer Modern} (\textbf{CM}) creados por D.~Knuth con ayuda del programa \textsc{metafont}. Por suerte \LaTeX{} está configurado en la actualidad para incorporar estas <<fuentes>> en los ficheros PDF como una fuente \emph{Postscript} Tipo 1 (vectorial). Con las fuentes Tipo 1 se obtiene una calidad al obtenido con las fuentes Tipo 3 o de mapa de bits (\textbf{PK}) que sólo ofrecen máxima calidad a la escala para la que fueron creadas. Por suerte en la actualidad \LaTeX{} ofrece la posibilidad de incluir las fuentes estándar \emph{Postscript} de \textsf{Adobe} utilizando los paquetes apropiados. Así es muy sencillo alternar en un texto entre las tres familias disponibles: Roman (redonda), \textsf{Sans Serif (paloseco, sin serifa o sin adornos)} y \texttt{Teletype (teletipo o monoespaciada)}. Y por último unas MAYÚSCULAS MÁS \textsc{pequeñas llamadas versalitas}. Por favor, no utilizes el \underline{subrayado} para \emph{enfatizar}. Aunque en algún documento (p.ej.\ ejercicios) su uso tampoco es grave.