%%%%%%%%%%%%%% 
% Fichero: EjFiles
% Autor: J. Salido (http://www.esi.uclm.es/www/jsalido)
% Fecha: febrero, 2017
% Descripción: Unión de varios ejemplos separados en 
% diferentes ficheros.
% Ejemplo del curso: “LaTeX esencial para preparación de TFG, Tesis
% y otros documentos académicos” (Esc. Sup. Informática-UCLM)
%%%%%%%%%%%%%%




%%%%%%%%%%%%%%
% Preámbulo del documento
%%%%%%%%%%%%%%
\documentclass[11pt,a4paper]{article} 
\usepackage[left=2cm,right=2cm,top=2cm,bottom=2cm]{geometry} % Márgenes 
\usepackage[spanish]{babel}

\usepackage{newtxtext}
\usepackage{newtxmath}

\usepackage{marvosym,pifont,textcomp,fontawesome5}

\usepackage[T1]{fontenc} % Codificación de salida    
\usepackage[
    protrusion=true,
    activate={true,nocompatibility},
    final,
    tracking=true,
    kerning=true,
    spacing=true,
    factor=1100]{microtype}
\SetTracking{encoding={*}, shape=sc}{40}

\usepackage[shortlabels]{enumitem} % Mayor control de listas
\usepackage{multicol} % Elementos en varias columnas

% EDITAR: Inclusión de hiperenlaces.
% Generación de hiperenlaces
\usepackage[%
   pdftex,
   breaklinks,
   hidelinks,      % Oculta colores en los enlaces (negro)
    linktocpage=true,    % true = enlace al nº de pág., false=texto completo
%    colorlinks=true,         % true=colorea texto del enlace, false=recuadra el texto
	citecolor=red, % Color de la citas
	urlcolor=blue, % Color de las URL
	bookmarksnumbered=true % Incluye números en bookmarks
]{hyperref}
%\usepackage{url}
\urlstyle{sf} % Estilo de URL sin serifas


% EDITAR: Ajustes de numeración de secciones e inclusión en índice
%\setcounter{secnumdepth}{2} % Ajusta el valor del último nivel numerado
%\setcounter{tocdepth}{2} %Ajusta el valor del último nivel que aparece en TOC
\includeonly{Sec2}
\author{Jesús Salido}
\title{Fundamentos de \LaTeX{} para principiantes}
\date{\today}
\begin{document}

\maketitle 

\begin{abstract}
En este ejemplo se han reunido varios de los ejemplos anteriores para mostrar el uso de secciones. El contenido de cada ejemplo se ha organizado como una sección de este nuevo documento, pero las secciones se han separado en ficheros independientes.
\end{abstract}

% NOTA: Inclusión de tabla de contenidos.
\hrule % Adornos para enmarcar la tabla de contenido
\tableofcontents % Así de fácil
\vspace{0.5cm} % Espacio vertical adicional
\hrule


\section{Alineación de textos en \LaTeX{}}
\label{sec:alinea} % En esta sección se ha incluido el contenido del documento EjAlinea.tex

\LaTeX{} emplea justificación completa de los párrafos por defecto. Pero este comportamiento puede interesarnos modificarlo. A continuación se muestran algunos ejemplos relacionados con la justificación de párrafos:

% Ejemplo:
% ============
\begin{flushleft}
	Este texto está alineado a la izquierda. \\
	\LaTeX{} no trata de justificar las líneas, \\ 
	así que así quedan.
\end{flushleft}

\begin{flushright}
	Texto alineado a la derecha. \\
	\LaTeX{} no trata de justificar las líneas.
\end{flushright}

\begin{center}
	En el centro\\
	de la Tierra
\end{center}


\noindent Y uno sobre el empleo de citas:

Una regla empírica tipográfica para la longitud de renglón es:

% Ejemplo:
% ============
\begin{quote}
\emph{<<En promedio, ningún renglón debería tener más de 66 signos porque así lo establece la propia experiencia.>>}
\end{quote}


Aunque también se podría haber dicho:


% Ejemplo:
% ============
\begin{quote}
\emph{<<La experiencia establece que, en promedio, ninguna línea de texto debería tener más de 66 signos. Esta y otras reglas derivadas de la experiencia se tendrán en cuenta a la hora de preparar documentos técnicos con corrección...>>}
\end{quote}

En el caso de citas que ocupan varios párrafos:


% Ejemplo:
% ============
\begin{quotation}
{\em <<¡Oh, memoria, enemiga mortal de mi descanso!...
La virtud más es perseguida de los malos que amada de los buenos...
La ingratitud es hija de la soberbia...
La razón de la sinrazón que a mi razón se hace, de tal manera mi razón enflaquece, que con razón me quejo de la vuestra fermosura.>>}
\end{quotation}

Por ello las páginas de \LaTeX{} tienen márgenes tan anchos por omisión, y los periódicos usan múltiples columnas. 

En alguna situación puede interesar realizar un formateado del los párrafos empleando tabuladores. El entorno \texttt{tabbing} proporciona el uso de tabuladores para conseguir la alineación deseada en los elementos del párrafo:





% Ejemplo:
% ============
% Observar que los espacios en blanco son irrelevantes
% \= fija la posición del tabulador
% \> avanza hasta la posición del tabulador
\begin{tabbing}
IF	\= \textbf{está} lloviendo                \\
    \> THEN \= \textbf{calzar} botas de agua, \\
    \>      \> \textbf{coger} paraguas;       \\
    \> ELSE \> \textbf{sonreir}.              \\
\textbf{Salir} de casa.
\end{tabbing}

\section{Organización de contenido mediante listas}
\label{sec:listas} % En esta sección se ha incluido el contenido del documento EjListas.tex


Las listas se emplean cuando se desea enumerar una serie de características, objetos, etc. A continuación veremos algunos ejemplos de listas.

\noindent Ejemplo de entorno {\tt itemize}:

% Ejemplo: Lista con viñetas anidada
% ============
\begin{itemize}
	\item peras
	\item manzanas
    \item cítricos
        \item limones
        \item naranjas
\end{itemize}

\noindent La misma lista cambiando la viñeta (\texttt{pifont}), gracias a paquete \texttt{enumitem}:

% Ejemplo: Lista con viñeta dingbat y un icono de fontawesome5
% ============
\begin{itemize}[\ding{68}]
\item peras
\item[\faIcon{apple}] manzanas
\item limones
\item naranjas
\end{itemize}

\noindent Otra personalización:
% Ejemplo:
% ============
\begin{itemize}[\ding{52},noitemsep] % Lista en la que se cambia el bullet por un símbolo dingbat
	\item peras
	\item[\ding{55}] manzanas
	\item naranjas
\end{itemize}

\noindent Las listas también se pueden anidar:
% Ejemplo: Lista anidada con balas
% ============
\begin{itemize}
	\item peras
	\begin{itemize}
		\item conferencia
		\item ercolina
    	\begin{itemize}
    		\item mini conferencia
    		\item extra ercolina
    	\end{itemize}
	\end{itemize}
	\item manzanas
	\begin{itemize}
		\item granny
		\item golden 
	\end{itemize}
	\item naranjas
\end{itemize}

Personalmente, pienso que \LaTeX{} proporciona demasiado espaciado en las listas. Con el paquete \texttt{enumitem} se puede compactar las listas. A continuación se muestra un ejemplo de entorno {\tt itemize} con opciones para cambiar el espaciado vertical entre ítems y cambio de viñeta por defecto (proporcionada por el paquete \texttt{enumitem}):

% Ejemplo: Cambia el espaciado entre items y la viñeta por defecto.
% ============
\begin{itemize}[\textbullet,noitemsep]
	\item peras
	\item manzanas
	\item naranjas
\end{itemize}

\noindent Ejemplo de entorno {\tt enumerate} para listas numeradas:

% Ejemplo:
% ============
\begin{enumerate}
	\item peras
	\item manzanas
	\item naranjas
\end{enumerate}


\noindent Ejemplo de entorno {\tt enumerate} con cambio de numeración (letras minúsculas) y opción \texttt{noitemsep} (proporcionada por el paquete \texttt{enumitem}) para listas numeradas:

% Ejemplo:
% ============
\begin{enumerate}[A.-,noitemsep]
	\item peras
	\item manzanas
	\item naranjas
\end{enumerate}

\noindent Este es otro ejemplo de línea compacta creada en varias columnas con ayuda de los paquetes \texttt{enumitem} y \texttt{multicol}:
% Ejemplo:
% ============
\begin{multicols}{2} % El parámetro es el número de columnas de la lista
	\begin{enumerate}[i.,noitemsep]
		\item peras
		\item manzanas
		\item naranjas
		\item patatas
		\item calabazas
		\item fresas
	\end{enumerate}
\end{multicols}


\noindent Las enumeraciones pueden personalizar de muchos modos, aunque debería estar justificado el cambiar el estilo por defecto, haciéndolo con mucha prudencia:
\noindent Así también:

% Ejemplo:
% ============
\begin{dingautolist}{192} % Lista con símbolos sucesivos de la tabla dingbat
	\item peras
	\item manzanas
	\item naranjas
\end{dingautolist}






\noindent y por supuesto anidar:

% Ejemplo:
% ============
\begin{enumerate}
  \item Cítricos
  \begin{enumerate}
    \item Limón
    \item Naranja
  \end{enumerate}
  \item Legumbres
  \item Hortalizas...
\end{enumerate}


\noindent Por último, el entorno {\tt description} se emplea para listas de términos o definiciones, como se muestra en los ejemplos siguientes:

% Ejemplo:
% ============
\begin{description}[noitemsep]
	\item[Estupideces] no mejoran por ponerlas en una lista por bonita que esta sea. Aunque la mona se vista de seda, mona se queda.
	\item[Lucideces] sin embargo, pueden parecer hermosas en
	una lista.
\end{description}

\noindent El ejemplo anterior formateado de otro modo:

% Ejemplo:
% ============
\begin{description}[style=nextline]
	\item[Estupideces] no mejoran por ponerlas en una lista por bonita que esta sea. Aunque la mona se vista de seda, mona se queda.
	\item[Lucideces] sin embargo, pueden parecer hermosas en una lista.
\end{description}

\LaTeX{} permite la configuración de listas, pero la mayoría de las veces esto no es necesario, ya que es suficiente el estilo por defecto proporcionado.


\end{document}

