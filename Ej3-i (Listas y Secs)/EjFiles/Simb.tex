\section{Símbolos y caracteres especiales en \LaTeX{}}
\label{sec:simb} % En esta sección se ha incluido el contenido del documento EjSimb.tex

En \TeX{} y \LaTeX{} las palabras reservadas o <<comandos>> del lenguaje están precedidos por la barra inclinada o \emph{backslash} (\textbackslash). Otros caracteres especiales son: \# \$ \% \textasciicircum \& \_ \{ \} \~{}. Para escribir estos caracteres se emplea:\\
\verb!\# \$ \% \textasciicircum \& \_ \{ \} \~!

Recordar los usos de las ``comillas dobles'' y las <<latinas>>.

\LaTeX{} también puede generar un conjunto muy amplio de símbolos especiales como el \EUR{} o \texteuro, \ding{45} y \Coffeecup. En los textos informáticos un carácter habitual es \verb+~+ empleado en las direcciones URL. Este carácter se puede generar de varias formas (\verb+~+, \~{}, $\sim$). Aunque empleando el paquete \texttt{url} la escritura de direcciones electrónicas se simplifica, por ejemplo:

{\footnotesize\url{http://osl.ugr.es/CTAN/info/symbols/comprehensive/symbols-a4.pdf}}

\noindent dirección URL de \emph{The Comprehensive \LaTeX{} Symbol List} de Scott Pakin (2009) donde se hace un repaso de todos los símbolos y caracteres que se pueden generar en \LaTeX{}.

\LaTeX{} genera su salida con los tipos Computer Modern (CM) (ver sección \ref{sec:tipos}, pág.~\pageref{sec:tipos}) creados por D.~Knuth con ayuda del programa \textsc{metafont}. Por suerte \LaTeX{} está configurado en la actualidad para incorporar estas <<fuentes>> en los ficheros PDF como una fuente Postscript Tipo 1 (vectorial). Con las fuentes Tipo 1 se obtiene una calidad al obtenido con las fuentes Tipo 3 o de mapa de bits (PK) que sólo ofrecen máxima calidad a la escala para la que fueron creadas. Por suerte en la actualidad \LaTeX{} ofrece la posibilidad de incluir las fuentes estándar Postscript de Adobe utilizando los paquetes <<apropiados>>. Así es muy sencillo alternar en un texto entre las tres familias disponibles: Roman (redonda), \textsf{Sans Serif (paloseco, sin serifa o sin adornos)} y \texttt{Teletype (teletipo o monoespaciada)}. 

Al componer documentos en español hay que tener en cuenta las peculiaridades de la tipografía española frente a la inglesa para hacer un uso correcto de los recursos ofrecidos por \LaTeX.

\noindent Con el comando \verb+\verb+ se puede generar texto que \LaTeX{} no procesa.

El entorno:
\begin{verbatim}
verbatim permite hacer lo mismo en un texto más extenso.
\end{verbatim} 



