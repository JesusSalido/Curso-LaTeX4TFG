
\hrule % Adornos para enmarcar la tabla de contenido

% NOTA: Inclusión de tabla de contenidos.
\tableofcontents % Así de fácil
\vspace{0.5cm} % Espacio vertical adicional
\hrule

\section{Tipografía con \LaTeX{}}
\label{sec:tipos} % En esta sección se ha incluido el contenido del documento EjTipos.tex
\LaTeX{} genera su salida con los tipos \emph{Computer Modern} (\textbf{CM}) creados por D.~Knuth con ayuda del programa \textsc{metafont}. Por suerte \LaTeX{} está configurado en la actualidad para incorporar estas <<fuentes>> en los ficheros PDF como una fuente \emph{Postscript} Tipo 1 (vectorial). Con las fuentes Tipo 1 se obtiene una calidad superior a la obtenida con las fuentes Tipo 3 o de mapa de bits (\textbf{PK}) que solo ofrecen máxima calidad a la escala para la que han sido creadas. Por suerte en la actualidad \LaTeX{} ofrece la posibilidad de incluir las fuentes estándar \emph{Postscript} de \textsf{Adobe} utilizando los paquetes apropiados. Así es muy claro alternar en un texto entre las tres familias disponibles: Roman (redonda), \textsf{Sans Serif (paloseco, sin serifa o sin adornos)} y \texttt{Teletype (teletipo o monoespaciada)}. Y por último unas MAYÚSCULAS MÁS \textsc{pequeñas llamadas versalitas}. {\fontfamily{qcr}\selectfont
This text uses a different font typeface
}
Por favor, no utilices el \underline{subrayado} para \emph{enfatizar}. Aunque en algún documento (p.~ej.w\ ejercicios) su uso tampoco es grave.

Por supuesto \LaTeX{} tiene en cuenta las particularidades de la tipografía en modo matemático como en: $f(x)=y^2$ y números $1,2,3,4,5,6,7,8,9$, o en modo texto: 0123456789.

$$g(x)=\int_{0}^{\infty}y^{3}dy$$



\section[Símbolos]{Símbolos y caracteres especiales en \LaTeX{}}
\label{sec:simb} % En esta sección se ha incluido el contenido del documento EjSimb.tex

En \TeX{} y \LaTeX{} las palabras reservadas o <<comandos>> del lenguaje están precedidos por la barra inclinada o \emph{backslash} (\textbackslash). Otros caracteres especiales son: \# \$ \% \textasciicircum \& \_ \{ \} \~{}. Para escribir estos caracteres se emplea:\\
\verb!\# \$ \% \textasciicircum \& \_ \{ \} \~!

Recordar los usos de las ``comillas dobles'' y las <<latinas>>.

\LaTeX{} también puede generar un conjunto muy amplio de símbolos especiales como el \EUR{} o \texteuro, \ding{45} y \Coffeecup. En los textos informáticos un carácter habitual es \verb+~+ empleado en las direcciones URL. Este carácter se puede generar de varias formas (\verb+~+, \~{}, $\sim$). Aunque empleando el paquete \texttt{url} la escritura de direcciones electrónicas se simplifica, por ejemplo:

\url{http://osl.ugr.es/CTAN/info/symbols/comprehensive/symbols-a4.pdf}

\noindent dirección URL de \emph{The Comprehensive \LaTeX{} Symbol List} de Scott Pakin (2009) donde se hace un repaso de todos los símbolos y caracteres que se pueden generar en \LaTeX{}.

Un gran conjunto de iconos empleados en los textos informáticos se pueden obtener mediante el uso de la tipografía \texttt{awesome5}, como por ejemplo: \faNodeJs, \faNode, \faGooglePlay, \faInternetExplorer, \faGithub, \faGit*, \faWhatsapp, etc.

\LaTeX{} genera su salida con los tipos Computer Modern (CM) creados por D.~Knuth con ayuda del programa \textsc{metafont}. Por suerte \LaTeX{} está configurado en la actualidad para incorporar estas <<fuentes>> en los ficheros PDF como una fuente Postscript Tipo 1 (vectorial). Con las fuentes Tipo 1 se obtiene una calidad al obtenido con las fuentes Tipo 3 o de mapa de bits (PK) que sólo ofrecen máxima calidad a la escala para la que fueron creadas. Por suerte en la actualidad \LaTeX{} ofrece la posibilidad de incluir las fuentes estándar Postscript de Adobe utilizando los paquetes <<apropiados>>. Así es muy sencillo\footnote{No tan sencillo.} alternar en un texto entre las tres familias disponibles: Roman (redonda), \textsf{Sans Serif (paloseco, sin serifa o sin adornos)} y \texttt{Teletype (teletipo o monoespaciada)}. 

Al componer documentos en español hay que tener en cuenta las peculiaridades de la tipografía española frente a la inglesa para hacer un uso correcto de los recursos ofrecidos por \LaTeX.

\noindent Con el comando \verb+\verb+ se puede generar texto que \LaTeX{} no procesa.\footnote{Emplearlo con precaución.}



El entorno:
\begin{verbatim}
   verbatim permite hacer lo 
   mismo en un texto más extenso.
\end{verbatim} 

Las versiones con estrella permiten destacar los espacios en blanco así:

\verb*|Texto sin  procesar|

Y también en el entorno verbatim:

\begin{verbatim*}
   verbatim permite hacer lo 
   mismo en un texto más extenso.
\end{verbatim*} 



\section{Alineación de textos en \LaTeX{}}
\label{sec:alinea} % En esta sección se ha incluido el contenido del documento EjAlinea.tex

\LaTeX{} emplea justificación completa de los párrafos por defecto. Pero este comportamiento puede interesarnos modificarlo. A continuación se muestran algunos ejemplos relacionados con la justificación de párrafos:

% Ejemplo:
% ============
\begin{flushleft}
	Este texto está alineado a la izquierda. \\
	\LaTeX{} no trata de justificar las líneas, \\
	así que así quedan.
\end{flushleft}

\begin{flushright}
	Texto alineado a la derecha. \\
	\LaTeX{} no trata de justificar las líneas.
\end{flushright}

\begin{center}
	En el centro\\
	de la Tierra
\end{center}

\noindent Y uno sobre el empleo de citas:

Una regla empírica tipográfica para la longitud de renglón es:

% Ejemplo:
% ============
\begin{quote}
	En promedio, ningún renglón debería tener más de 66 signos porque así lo establece la propia experiencia.
\end{quote}

Por ello las páginas de \LaTeX{} tienen márgenes tan anchos por omisión, y los periódicos usan múltiples columnas. 

En alguna situación puede interesar realizar un formateado del los párrafos empleando tabuladores. El entorno \texttt{tabbing} proporciona el uso de tabuladores para conseguir la alineación deseada en los elementos del párrafo:

% Ejemplo:
% ============
% Observar que los espacios en blanco son irrelevantes
% \= fija la posición del tabulador
% \> avanza hasta la posición del tabulador
\begin{tabbing}
	IF \= \textbf{está} lloviendo                \\
	\> THEN \= \textbf{calzar} botas de agua, \\
	\>      \> \textbf{coger} paraguas;       \\
	\> ELSE \> \textbf{sonreir}.              \\
	\textbf{Salir} de casa.
\end{tabbing}







\section{Listas}
\label{sec:listas} % En esta sección se ha incluido el contenido del documento EjListas.tex


Las listas se emplean cuando se desea enumerar una serie de características, objetos, etc. A continuación veremos algunos ejemplos de listas.

\noindent Ejemplo de entorno {\tt itemize}:

% Ejemplo: Lista con viñetas
% ============
\begin{itemize}
	\item peras
	\item manzanas
	\item limones
	\item naranjas
\end{itemize}

\noindent La misma lista cambiando la viñeta, gracias a paquete \texttt{enumitem}:
% Ejemplo: Lista con viñeta dingbat
% ============
\begin{itemize}[\ding{111}]
	\item peras
	\item manzanas
	\item limones
	\item naranjas
\end{itemize}


Las listas también se pueden anidar:
% Ejemplo: Lista anidada con balas
% ============
\begin{itemize}
	\item peras
	\begin{itemize}
		\item conferencia
		\item ercolina
    	\begin{itemize}
    		\item mini conferencia
    		\item extra ercolina
    	\end{itemize}
	\end{itemize}
	\item manzanas
	\begin{itemize}
		\item granny
		\item golden 
	\end{itemize}
	\item naranjas
\end{itemize}


\noindent Ejemplo de entorno {\tt itemize} en el que se sustituye el símbolo (\emph{bullet}) por defecto:

% Ejemplo:
% ============
\begin{itemize}
	\item[*] peras
	\item manzanas
	\item[\ding{170}] naranjas
\end{itemize}


Personalmente, pienso que \LaTeX{} proporciona demasiado espaciado en las listas. Con el paquete \texttt{enumitem} se puede compactar las listas. A continuación se muestra un ejemplo de entorno {\tt itemize} con opciones para cambiar el espaciado vertical entre ítems y cambio de viñeta por defecto (proporcionada por el paquete \texttt{enumitem}):

% Ejemplo:
% ============
\begin{itemize}[*,noitemsep]
	\item peras
	\item manzanas
	\item naranjas
\end{itemize}

\noindent Ejemplo de entorno {\tt enumerate} para listas numeradas:

% Ejemplo:
% ============
\begin{enumerate}
	\item peras
	\item manzanas
	\item naranjas
\end{enumerate}


\noindent Ejemplo de entorno {\tt enumerate} con cambio de numeración (letras minúsculas) y opción \texttt{noitemsep} (proporcionada por el paquete \texttt{enumitem}) para listas numeradas:

% Ejemplo:
% ============
\begin{enumerate}[a.-,noitemsep]
	\item peras
	\item manzanas
	\item naranjas
\end{enumerate}

\noindent Este es otro ejemplo de línea compacta creada en varias columnas con ayuda de los paquetes \texttt{enumitem} y \texttt{multicol}:
% Ejemplo:
% ============
\begin{multicols}{2} % El parámetro es el número de columnas de la lista
	\begin{enumerate}[i.,noitemsep]
		\item peras
		\item manzanas
		\item naranjas
		\item patatas
		\item calabazas
		\item fresas
	\end{enumerate}
\end{multicols}


\noindent Las listas incluso se pueden personalizar (aunque debería estar justificado el cambiar el estilo por defecto, haciéndolo con mucha prudencia):

% Ejemplo:
% ============
\begin{enumerate}[\ding{52},noitemsep] % Lista en la que se cambia el bullet por un símbolo dingbat
	\item peras
	\item[\ding{55}] manzanas
	\item naranjas
\end{enumerate}


\noindent Así también:

% Ejemplo:
% ============
\begin{dingautolist}{202} % Lista con símbolos sucesivos de la tabla dingbat
	\item peras
	\item manzanas
	\item naranjas
\end{dingautolist}


\noindent y por supuesto anidar:

% Ejemplo:
% ============
\begin{enumerate}
  \item Cítricos
  \begin{enumerate}
    \item Limón
    \item Naranja
  \end{enumerate}
  \item Legumbres
  \item Hortalizas \ldots
\end{enumerate}


\noindent Ejemplo de entorno {\tt description} para listas:

% Ejemplo:
% ============
\begin{description}[noitemsep]
	\item[Estupideces] no mejoran por ponerlas en una lista por bonita que esta sea. Aunque la mona se vista de seda, mona se queda.
	\item[Lucideces] sin embargo, pueden parecer hermosas en
	una lista.
\end{description}

\noindent El ejemplo anterior formateado de otro modo:

% Ejemplo:
% ============
\begin{description}[style=nextline]
	\item[Estupideces] no mejoran por ponerlas en una lista por bonita que esta sea. Aunque la mona se vista de seda, mona se queda.
	\item[Lucideces] sin embargo, pueden parecer hermosas en una lista.
\end{description}

\LaTeX{} permite una configuración más avanzada para presentar listas más sofisticadas, aunque la mayoría de las veces esto no es necesario, ya que es suficiente con las listas convencionales.


\end{document}

