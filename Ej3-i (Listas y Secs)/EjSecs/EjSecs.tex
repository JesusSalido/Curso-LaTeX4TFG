%%%%%%%%%%%%%% 
% Fichero: EjSecs
% Autor: J. Salido (http://www.uclm.es/profesorado/jsalido)
% Fecha: febrero, 2017
% Descripción: Unión de varios ejemplos
% Ejemplo del curso: “LaTeX esencial para preparación de TFG, Tesis
% y otros documentos académicos” (Esc. Sup. Informática-UCLM)
%%%%%%%%%%%%%%



%%%%%%%%%%%%%%
% Preámbulo del documento
%%%%%%%%%%%%%%
\documentclass[11pt,a4paper]{article} 
\usepackage[utf8]{inputenx}
\usepackage[left=2cm,right=2cm,top=2cm,bottom=2cm]{geometry} % Márgenes 

\usepackage[spanish]{babel}

\usepackage{newpxtext}
\usepackage{newpxmath}
\usepackage{marvosym}
\usepackage{pifont} % Generación de símbolos especiales
\usepackage{textcomp}
\usepackage[T1]{fontenc} % Codificación de salida    
\usepackage{microtype} % Mejoras de microtipografía en la obtención de PDF (sólo para pdflatex)
\usepackage{url} % Para escritura de URL
\urlstyle{sf} % Estilo de URL sin serifas

\usepackage{paralist} % Mayor control de listas
\usepackage{multicol} % Elementos en varias columnas


% Generación de hiperenlaces
\usepackage[pdftex,breaklinks,colorlinks,
	citecolor=blue, % Color de la citas
	urlcolor=blue, % Color de las URL
	bookmarksnumbered=true % Incluye números en bookmarks
    ]{hyperref}

% Con estas instrucciones se ajustan los valores del índice
\setcounter{secnumdepth}{2} % Ajusta el valor del último nivel numerado
\setcounter{tocdepth}{2} %Ajusta el valor del último nivel que aparece en TOC


\author{Jesús Salido}
\title{Fundamentos de \LaTeX{} para principiantes}
\date{\today}

%\selectspanish

%%%%%%%%%%%%%%
% Comienzo del documento
%%%%%%%%%%%%%%
\begin{document}

\maketitle 

\begin{abstract}
	En este ejemplo se han reunido varios de los ejemplos anteriores para mostrar el uso de secciones. El contenido de cada ejemplo se ha organizado como una sección de este nuevo documento. También se muestra como preparar un índice de contenido.
\end{abstract}



% Cuando se generan índices de contenido y referencias cruzadas es preciso compilar el documento mútiples veces (al menos dos) hasta obtener el resultado apropiado. 
% Esto no es un funcionamiento incorrecto sino necesario.
% Sin embargo, al emplear TeXstudio el entorno detecta la necesidad de compilar varias veces para obtener las referencias correctas.

\hrule % Adornos para enmarcar la tabla de contenido
\tableofcontents % Así de fácil
\vspace{0.5cm} % Espacio vertical adicional
\hrule

\section{Tipografía con \LaTeX{}}
\label{sec:tipos} % En esta sección se ha incluido el contenido del documento EjTipos.tex

\LaTeX{} genera su salida con los tipos \emph{Computer Modern} (\textbf{CM}) creados por D.~Knuth con ayuda del programa \textsc{metafont}. Por suerte \LaTeX{} está configurado en la actualidad para incorporar estas <<fuentes>> en los ficheros PDF como una fuente \emph{Postscript} Tipo 1 (vectorial). Con las fuentes Tipo 1 se obtiene una calidad al obtenido con las fuentes Tipo 3 o de mapa de bits (\textbf{PK}) que solo ofrecen máxima calidad a la escala para la que fueron creadas. Por suerte en la actualidad \LaTeX{} ofrece la posibilidad de incluir las fuentes estándar \emph{Postscript} de \textsf{Adobe} utilizando los paquetes apropiados. Así es muy sencillo alternar en un texto entre las tres familias disponibles: Roman (redonda), \textsf{Sans Serif (paloseco, sin serifa o sin adornos)} y \texttt{Teletype (teletipo o monoespaciada)}. Y por último unas MAYÚSCULAS MÁS \textsc{pequeñas llamadas versalitas}. Por favor, no utilices el \underline{subrayado} para \emph{enfatizar}. Aunque en algún documento (p.ej.\ ejercicios) su uso tampoco es grave.



\section[Símbolos]{Símbolos y caracteres especiales en \LaTeX{}}
\label{sec:simb} % En esta sección se ha incluido el contenido del documento EjSimb.tex

En \TeX{} y \LaTeX{} las palabras reservadas o <<comandos>> del lenguaje están precedidos por la barra inclinada o \emph{backslash} (\textbackslash). Otros caracteres especiales son: \# \$ \% \textasciicircum \& \_ \{ \} \~{}. Para escribir estos caracteres se emplea:\\
\verb!\# \$ \% \textasciicircum \& \_ \{ \} \~!

Recordar los usos de las ``comillas dobles'' y las <<latinas>>.

\LaTeX{} también puede generar un conjunto muy amplio de símbolos especiales como el \EUR{} o \texteuro, \ding{45} y \Coffeecup. En los textos informáticos un carácter habitual es \verb+~+ empleado en las direcciones URL. Este carácter se puede generar de varias formas (\verb+~+, \~{}, $\sim$). Aunque empleando el paquete \texttt{url} la escritura de direcciones electrónicas se simplifica, por ejemplo:

{\url{http://osl.ugr.es/CTAN/info/symbols/comprehensive/symbols-a4.pdf}}

\noindent dirección URL de \emph{The Comprehensive \LaTeX{} Symbol List} de Scott Pakin (2009) donde se hace un repaso de todos los símbolos y caracteres que se pueden generar en \LaTeX{}.

\LaTeX{} genera su salida con los tipos Computer Modern (CM) (ver sec.~\ref{sec:tipos}, pág.~\pageref{sec:tipos}) creados por D.~Knuth con ayuda del programa \textsc{metafont}. Por suerte \LaTeX{} está configurado en la actualidad para incorporar estas <<fuentes>> en los ficheros PDF como una fuente Postscript Tipo 1 (vectorial). Con las fuentes Tipo 1 se obtiene una calidad al obtenido con las fuentes Tipo 3 o de mapa de bits (PK) que solo ofrecen máxima calidad a la escala para la que fueron creadas. Por suerte en la actualidad \LaTeX{} ofrece la posibilidad de incluir las fuentes estándar Postscript de Adobe utilizando los paquetes <<apropiados>>. Así es muy sencillo alternar en un texto entre las tres familias disponibles: Roman (redonda), \textsf{Sans Serif (paloseco, sin serifa o sin adornos)} y \texttt{Teletype (teletipo o monoespaciada)}. 

Al componer documentos en español hay que tener en cuenta las peculiaridades de la tipografía española frente a la inglesa para hacer un uso correcto de los recursos ofrecidos por \LaTeX.

\noindent Con el comando \verb+\verb+ se puede generar texto que \LaTeX{} no procesa.

El entorno:
\begin{verbatim}
verbatim permite hacer lo mismo en un texto más extenso.
\end{verbatim} 





\section{Alineación de textos en \LaTeX{}}
\label{sec:alinea} % En esta sección se ha incluido el contenido del documento EjAlinea.tex

\LaTeX{} emplea justificación completa de los párrafos por defecto. Pero este comportamiento puede interesarnos modificarlo. A continuación se muestran algunos ejemplos relacionados con la justificación de párrafos:

% Ejemplo:
% ============
\begin{flushleft}
	Este texto está alineado a la izquierda. \\
	\LaTeX{} no trata de justificar las líneas, \\
	así que así quedan.
\end{flushleft}

\begin{flushright}
	Texto alineado a la derecha. \\
	\LaTeX{} no trata de justificar las líneas.
\end{flushright}

\begin{center}
	En el centro\\
	de la Tierra
\end{center}

\noindent Y uno sobre el empleo de citas:

Una regla empírica tipográfica para la longitud de renglón es:

% Ejemplo:
% ============
\begin{quote}
	En promedio, ningún renglón debería tener más de 66 signos porque así lo establece la propia experiencia.
\end{quote}

Por ello las páginas de \LaTeX{} tienen márgenes tan anchos por omisión, y los periódicos usan múltiples columnas. 

En alguna situación puede interesar realizar un formateado del los párrafos empleando tabuladores. El entorno \texttt{tabbing} proporciona el uso de tabuladores para conseguir la alineación deseada en los elementos del párrafo:

% Ejemplo:
% ============
% Observar que los espacios en blanco son irrelevantes
% \= fija la posición del tabulador
% \> avanza hasta la posición del tabulador
\begin{tabbing}
	IF \= \textbf{está} lloviendo                \\
	\> THEN \= \textbf{calzar} botas de agua, \\
	\>      \> \textbf{coger} paraguas;       \\
	\> ELSE \> \textbf{sonreir}.              \\
	\textbf{Salir} de casa.
\end{tabbing}







\section{Listas}
\label{sec:listas} % En esta sección se ha incluido el contenido del documento EjListas.tex


Las listas se emplean cuando se desea enumerar una serie características, objetos, etc. A continuación veremos algunos ejemplos de listas.

\noindent Ejemplo de entorno {\tt itemize}:

% Ejemplo: Lista con balas
% ============
\begin{itemize}
	\item peras
	\item manzanas
	\item naranjas
\end{itemize}


Las listas también se pueden anidar:
% Ejemplo: Lista anidada con balas
% ============
\begin{itemize}
	\item peras
	\begin{itemize}
		\item conferencia
		\item ercolina
	\end{itemize}
	\item manzanas
	\begin{itemize}
		\item granny
		\item golden 
	\end{itemize}
	\item naranjas
\end{itemize}



\noindent Ejemplo de entorno {\tt itemize} en el que se sustituye el símbolo (\emph{bullet}) por defecto:

% Ejemplo:
% ============
\begin{itemize}
	\item[*] peras
	\item manzanas
	\item[\ding{170}] naranjas
\end{itemize}


Personalmente pienso que \LaTeX{} proporciona demasiado espaciado en las lista. Pero con el paquete \texttt{paralist} se puede compactar las listas. A continuación se muestra un ejemplo de entorno {\tt compactitem} (proporcionado por el paquete \texttt{paralist}):

% Ejemplo:
% ============
\begin{compactitem}
	\item[*] peras
	\item manzanas
	\item[\ding{170}] naranjas
\end{compactitem}



\bigskip % Salto vertical equivalente a 1 línea (\medskip: 1/2 línea, \smallskip: 1/4 línea)

\noindent Ejemplo de entorno {\tt enumerate} para listas numeradas:

% Ejemplo:
% ============
\begin{enumerate}
	\item peras
	\item manzanas
	\item naranjas
\end{enumerate}


\newpage

\noindent Ejemplo de entorno {\tt compactenum} (proporcionado por el paquete \texttt{paralist}) para listas numeradas:

% Ejemplo:
% ============
\begin{compactenum}
	\item peras
	\item manzanas
	\item naranjas
\end{compactenum}


\bigskip

\noindent Este es otro ejemplo de línea compacta creada en varias columnas con ayuda de los paquetes \texttt{paralist} y \texttt{multicol}:
% Ejemplo:
% ============
\begin{multicols}{2} % El parámetro es el número de columnas de la lista
	\begin{compactenum}
		\item peras
		\item manzanas
		\item naranjas
		\item patatas
		\item calabazas
		\item fresas
	\end{compactenum}
\end{multicols}


\noindent Las listas incluso se pueden personalizar (aunque debería estar justificado el cambiar el estilo por defecto, haciéndolo con mucha prudencia):

% Ejemplo:
% ============
\begin{dinglist}{52} % Lista en la que se cambia el bullet por un símbolo dingbat
	\item peras
	\item manzanas
	\item naranjas
\end{dinglist}


\noindent Así también:

% Ejemplo:
% ============
\begin{dingautolist}{202} % Lista con símbolos sucesivos de la tabla dingbat
	\item peras
	\item manzanas
	\item naranjas
\end{dingautolist}


\noindent y por supuesto anidar:

% Ejemplo:
% ============
\begin{enumerate}
	\item Cítricos
	\begin{enumerate}
		\item Limón
		\item Naranja
	\end{enumerate}
	\item Legumbres
	\item Hortalizas \ldots
\end{enumerate}


\noindent Ejemplo de entorno {\tt description} para listas:

% Ejemplo:
% ============
\begin{description}
	\item[Estupideces] no mejoran por ponerlas en una lista por bonita que esta sea. Aunque la mona se vista de seda, mona se queda.
	\item[Lucideces] sin embargo, pueden parecer hermosas en
	una lista.
\end{description}

\noindent El ejemplo anterior formateado de otro modo:

% Ejemplo:
% ============
\begin{description}
	\item[Estupideces]\hfill \\ 
	no mejoran por ponerlas en una lista por bonita que esta sea. Aunque la mona se vista de seda, mona se queda.
	\item[Lucideces]\hfill \\ 
	sin embargo, pueden parecer hermosas en una lista.
\end{description}

\LaTeX{} permite una configuración más avanzada para presentar listas más sofisticadas, aunque la mayoría de las veces esto no es necesario ya que es suficiente con las listas convencionales.




\end{document}

