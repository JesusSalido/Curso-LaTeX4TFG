\documentclass [15pt,a4paper]{article}
\usepackage[spanish,es-tabla]{babel}
\usepackage[utf8]{inputenc}
\usepackage[T1]{fontenc}
\usepackage{amsmath}
\usepackage{geometry}
\geometry{verbose,a4paper, tmargin=3.5cm,bmargin=3.5cm,lmargin=2.5cm,rmargin=2.5cm,headsep=1cm,footskip=1.5cm}
\usepackage[dvipsnames]{xcolor}
\usepackage{graphicx}
\usepackage{caption}
\usepackage{subcaption}
\usepackage{float}

\begin{document}
\begin{figure}[H]
     \centering
     \begin{subfigure}[b]{0.3\textwidth}
         \centering
         \includegraphics[width=\textwidth]{example-image-a}
         \caption{Ejemplo A}
         \label{fig:ejA}
     \end{subfigure}
     \hfill
     \begin{subfigure}[b]{0.3\textwidth}
         \centering
         \includegraphics[width=\textwidth]{example-image-b}
         \caption{Ejemplo B}
         \label{fig:ejB}
     \end{subfigure}     
     \hfill
     \begin{subfigure}[b]{0.3\textwidth}
         \centering
         \includegraphics[width=\textwidth]{example-image-c}
         \caption{Ejemplo C}
         \label{fig:EjC}
     \end{subfigure}
        \caption{Ejemplo de figura con tres subfiguras.}
        \label{fig:ej3figs}
\end{figure}

Lo mismo con tablas.

\begin{table}[H]
    \begin{subtable}[h]{0.45\textwidth}
        \centering
        \begin{tabular}{l | l | l}
        Day & Max Temp & Min Temp \\
        \hline \hline
        Mon & 20 & 13\\
        Tue & 22 & 14\\
        Wed & 23 & 12\\
        Thurs & 25 & 13\\
        Fri & 18 & 7\\
        Sat & 15 & 13\\
        Sun & 20 & 13
       \end{tabular}
       \caption{First Week}
       \label{tab:week1}
    \end{subtable}
    \hfill
    \begin{subtable}[h]{0.45\textwidth}
        \centering
        \begin{tabular}{l | l | l}
        Day & Max Temp & Min Temp \\
        \hline \hline
        Mon & 17 & 11\\
        Tue & 16 & 10\\
        Wed & 14 & 8\\
        Thurs & 12 & 5\\
        Fri & 15 & 7\\
        Sat & 16 & 12\\
        Sun & 15 & 9
        \end{tabular}
        \caption{Second Week}
        \label{tab:week2}
     \end{subtable}
     \caption{Max and min temps recorded in the first two weeks of July}
     \label{tab:temps}
\end{table}

Mezclando:

\begin{figure}[H]
     \centering
     \begin{subfigure}{0.3\textwidth}
         \centering
         \includegraphics[width=\textwidth]{example-image-a}
         \caption{Ejemplo A}
         \label{fig:figura}
     \end{subfigure}
%     \hfill
    \begin{subtable}{0.45\textwidth}
        \centering
        \begin{tabular}{l | l | l}
        Day & Max Temp & Min Temp \\
        \hline \hline
        Mon & 17 & 11\\
        Tue & 16 & 10\\
        Wed & 14 & 8\\
        Thurs & 12 & 5\\
        Fri & 15 & 7\\
        Sat & 16 & 12\\
        Sun & 15 & 9
        \end{tabular}
        \caption{Second Week}
        \label{tab:tabla}
     \end{subtable}
     \caption{Max and min temps recorded in the first two weeks of July}
     \label{figs:mezcla}
\end{figure}
\end{document}
