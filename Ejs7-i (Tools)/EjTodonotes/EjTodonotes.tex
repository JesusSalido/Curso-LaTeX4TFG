%%%%%%%%%%%%%% 
% Fichero: EjAlinea
% Autor: J. Salido (http://www.esi.uclm.es/www/jsalido)
% Fecha: febrero, 2017
% Descripción: Alineación de texto
% Ejemplo del curso: “LaTeX esencial para preparación de TFG, Tesis
% y otros documentos académicos” (Esc. Sup. Informática-UCLM)
%%%%%%%%%%%%%%


%%%%%%%%%%%%%%
% Preámbulo del documento
%%%%%%%%%%%%%%
\documentclass[11pt,a4paper,draft]{article} 
%\documentclass[11pt,a4paper,final]{article} 
\usepackage[spanish]{babel} 
\usepackage[left=2cm,right=3cm,top=2cm,bottom=2cm]{geometry}

\usepackage[T1]{fontenc} % Codificación de salida    
\usepackage{microtype} % Mejoras de microtipografía en la obtención de PDF (sólo para pdflatex)

% Para incluir notas al margen (desactivadas en version final)
\usepackage[spanish,size=small,obeyDraft,colorinlistoftodos]{todonotes} 


\title{Ejemplos de alineamiento de texto con \LaTeX}
\author{Jesús Salido}
\date{\today}



%%%%%%%%%%%%%%
% Comienzo del documento
%%%%%%%%%%%%%%
\begin{document}
\maketitle

\begin{abstract}
	Con \LaTeX{} también puede configurarse la alineación por defecto de los párrafos con distintos propósitos. Aunque se trata de situaciones poco frecuentes, en este ejemplo se muestran algunos ejemplos de uso.
\end{abstract}

\listoftodos[Pendiente]

\section{El texto principal}

\LaTeX{} emplea justificación completa de los párrafos por defecto. Pero este comportamiento puede interesarnos modificarlo. A continuación se muestran algunos ejemplos relacionados con la justificación de párrafos:
\todo[inline]{Ejemplos muy simples para mostrar el uso}

% Ejemplo:
% ============
\begin{flushleft}
	Este texto está alineado a la izquierda. \\
	\LaTeX{} no trata de justificar las líneas, \\
	así que así quedan.
\end{flushleft}

\begin{flushright}
	Texto alineado a la derecha. \\
	\LaTeX{} no trata de justificar las líneas.
\end{flushright}

\begin{center}
	En el centro\\
	de la Tierra
\end{center}

\missingfigure{Una figura quizá lo explicaría mejor}

\noindent Y uno sobre el empleo de citas:

Una regla empírica tipográfica para la longitud de renglón es: 
\todo{Se usa poco}
% Ejemplo:
% ============
\begin{quote}
\emph{	<<En promedio, ningún renglón debería tener más de 66 signos porque así lo establece la propia experiencia.>>}
\end{quote}


Aunque también se podría haber dicho: 
% Ejemplo:
% ============
\begin{quote}
\emph{	<<La experiencia establece que, en promedio, ninguna línea de texto debería tener más de 66 signos. Esta y otras reglas derivadas de la experiencia se tendrán en cuenta a la hora de preparar documentos técnicos con corrección...>>}
\end{quote}


Por ello las páginas de \LaTeX{} tienen márgenes tan anchos por omisión, y los periódicos usan múltiples columnas. 

En alguna situación puede interesar realizar un formateado del los párrafos empleando tabuladores. El entorno \texttt{tabbing} proporciona el uso de tabuladores para conseguir la alineación deseada en los elementos del párrafo:





% Ejemplo:
% ============
% Observar que los espacios en blanco son irrelevantes
% \= fija la posición del tabulador
% \> avanza hasta la posición del tabulador
\begin{tabbing}
IF	\= \textbf{está} lloviendo                \\
 	\> THEN \= \textbf{calzar} botas de agua, \\
 	\>      \> \textbf{coger} paraguas;       \\
 	\> ELSE \> \textbf{sonreir}.              \\
\textbf{Salir} de casa.
\end{tabbing}

\end{document}
