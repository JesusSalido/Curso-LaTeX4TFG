% !TeX spellcheck = es_ES
% Tipo de documento:
\documentclass[11pt,a4paper]{article}

% ====================
% Inicio del Preámbulo 
% ====================
\usepackage[spanish]{babel}	% Idioma del texto
% Otros paquetes
\usepackage[left=5cm,right=2cm,top=2cm,bottom=2cm]{geometry} % Márgenes 
%\usepackage[T1]{fontenc}

\title{Ejemplo básico con \LaTeX}
\author{Jesús Salido}
\date{\today}
% ====================
% Fin del Preámbulo
% ====================



% Inicio del texto en sí
\begin{document}
\maketitle
% Documento en sí
%...


“Este” es un ejemplo de documento. Las     palabros        pueden estar       separadas por tantos         espacios     en blanco   como se desee. Esto es indiferente. 



¡Hola! \\[2cm]





           Los párrafos quedan separados por una, dos, o más, líneas en blanco.
           
           
\end{document}
% Fin del documento
Todo lo que se añada por debajo de \end{document} es como si no existiera


