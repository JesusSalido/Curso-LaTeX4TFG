%%%%%%%%%%%%%% 
% Fichero: EjSimb
% Autor: J. Salido (http://www.esi.uclm.es/www/jsalido)
% Fecha: febrero, 2017
% Rev.: febrero 2022
% Descripción: Generación de caracteres especiales con LaTeX
% Ejemplo del curso: “LaTeX esencial para preparación de TFG, Tesis
% y otros documentos académicos” (Esc. Sup. Informática-UCLM)
%%%%%%%%%%%%%%

%%%%%%%%%%%%%%
% Preámbulo del documento
%%%%%%%%%%%%%%
%\documentclass[11pt,a4paper]{report} 
\documentclass[11pt,a4paper]{article} 
\usepackage[utf8]{inputenx}	% Codificación de entrada
\usepackage[spanish]{babel}	% Idioma del texto
%\usepackage[english]{babel}	% Idioma del texto
\usepackage[left=2cm,right=2cm,top=2cm,bottom=2cm]{geometry} % Márgenes 

%--- Interlineado: Estrecho (6 ptos./interlineado 1,15) o Moderado (6 ptos./interlineado 1,5)
% \parskip : long. que se añade al salto entre líneas para obtener el salto entre párrafos.
% \baselinestretch: factor por el que se multiplica el espacio entre líneas respecto al valor por defecto
%---
%\setlength{\parskip}{6pt}
%\renewcommand{\baselinestretch}{1.15}



% OJO: Si los paquetes de símbolos que cargan antes de los de fuentes pueden dar problemas.
\usepackage{newtxtext}
\usepackage{newtxmath}


\usepackage{textcomp}
\usepackage{marvosym}
\usepackage{pifont} % Generación de símbolos especiales
\usepackage{fontawesome5} % Iconos fontawesome


\usepackage[T1]{fontenc} % Codificación de salida    
%\usepackage{microtype} % Mejoras de microtipografía en la obtención de PDF (sólo para pdflatex)
\usepackage{url} % Para escritura de URL
\urlstyle{sf} % Estilo de URL sin serifas para que tengan un mejor aspecto


\title{Generación de símbolos con \LaTeX}
\author{Jesús Salido}
\date{\today}


%%%%%%%%%%%%%%
% Comienzo del documento
%%%%%%%%%%%%%%
\begin{document}
\maketitle


\begin{abstract}
	Este es un ejemplo para mostrar los símbolos que puede producir \LaTeX.
\end{abstract}


\pagenumbering{roman}
\setcounter{page}{25}


En \TeX{} y \LaTeX{} las palabras reservadas o <<comandos>> del lenguaje están precedidos por la barra inclinada o \emph{backslash} (\textbackslash). Otros caracteres especiales son: \# \$ \% \textasciicircum \& \_ \{ \} \~{}. Para escribir estos caracteres se emplea:\\
\verb!\# \$ \% \textasciicircum \& \_ \{ \} \~!

Recordar los usos de las ``comillas dobles'' y las <<latinas>>.

\LaTeX{} también puede generar un conjunto muy amplio de símbolos especiales como el \EUR{} o \texteuro, \ding{45} y \Coffeecup. En los textos informáticos un carácter habitual es \verb+~+ empleado en las direcciones URL. Este carácter se puede generar de varias formas (\verb+~+, \~{}, $\sim$). Aunque empleando el paquete \texttt{url} la escritura de direcciones electrónicas se simplifica, por ejemplo:

\url{http://osl.ugr.es/CTAN/info/symbols/comprehensive/symbols-a4.pdf}

\noindent dirección URL de \emph{The Comprehensive \LaTeX{} Symbol List} de Scott Pakin (2009) donde se hace un repaso de todos los símbolos y caracteres que se pueden generar en \LaTeX{}.

Un gran conjunto de iconos empleados en los textos informáticos se pueden obtener mediante el uso de la tipografía \texttt{awesome5}, como por ejemplo: \faNodeJs, \faNode, \faGooglePlay, \faInternetExplorer, \faGithub, \faGit*, \faWhatsapp, etc.

\LaTeX{} genera su salida con los tipos Computer Modern (CM) creados por D.~Knuth con ayuda del programa \textsc{metafont}. Por suerte \LaTeX{} está configurado en la actualidad para incorporar estas <<fuentes>> en los ficheros PDF como una fuente Postscript Tipo 1 (vectorial). Con las fuentes Tipo 1 se obtiene una calidad al obtenido con las fuentes Tipo 3 o de mapa de bits (PK) que sólo ofrecen máxima calidad a la escala para la que fueron creadas. Por suerte en la actualidad \LaTeX{} ofrece la posibilidad de incluir las fuentes estándar Postscript de Adobe utilizando los paquetes <<apropiados>>. Así es muy sencillo\footnote{No tan sencillo.} alternar en un texto entre las tres familias disponibles: Roman (redonda), \textsf{Sans Serif (paloseco, sin serifa o sin adornos)} y \texttt{Teletype (teletipo o monoespaciada)}. 

Al componer documentos en español hay que tener en cuenta las peculiaridades de la tipografía española frente a la inglesa para hacer un uso correcto de los recursos ofrecidos por \LaTeX.

\noindent Con el comando \verb+\verb+ se puede generar texto que \LaTeX{} no procesa.\footnote{Emplearlo con precaución.}



El entorno:
\begin{verbatim}
   verbatim permite hacer lo 
   mismo en un texto más extenso.
\end{verbatim} 

Las versiones con estrella permiten destacar los espacios en blanco así:

\verb*|Texto sin  procesar|

Y también en el entorno verbatim:

\begin{verbatim*}
   verbatim permite hacer lo 
   mismo en un texto más extenso.
\end{verbatim*} 


\end{document}




