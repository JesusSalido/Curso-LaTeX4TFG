%%%%%%%%%%%%%% 
% Fichero: EjTipos
% Autor: J. Salido (http://www.esi.uclm.es/www/jsalido)
% Fecha: febrero, 2017
% Rev.: febrero 2022
% Descripción: Generación de tipografías con LaTeX
% Ejemplo del curso: “LaTeX esencial para preparación de TFG, Tesis
% y otros documentos académicos” (Esc. Sup. Informática-UCLM)
%%%%%%%%%%%%%% 



%%%%%%%%%%%%%%
% Preámbulo del documento
%%%%%%%%%%%%%%
\documentclass[11pt,a4paper]{article} 
\usepackage[spanish]{babel}	% Idioma del texto
\usepackage[left=2cm,right=2cm,top=2cm,bottom=2cm]{geometry} % Márgenes 
% NOTA: Mejoras en el espacio vertical entre párrafos.
%\usepackage[skip=.3\baselineskip plus 2pt,indent]{parskip} % Salto entre párrafos 
% skip= .5\baselineskip plus 2pt -> Valor por defecto
% 


%-------------
%-------------
% Tipografías apropiadas para textos técnicos (con matemáticas)

% -> Latin Modern | LM Sans Serif | LM teletype (antigua)
%\usepackage{lmodern}  

% NOTA: Esta es mi opción preferida para documentos académicos
% -> Libertine + math (Times)
%\usepackage[osf]{libertine}
%\usepackage[libertine]{newtxmath}

% -> Times + math
%\usepackage[osf]{newtxtext}
%\usepackage{newtxmath}

% -> Palatino + math 
\usepackage[osf]{newpxtext}
\usepackage{newpxmath}

% -> Utopia:
%\usepackage{fourier}

% -> Romanas + math (austera)
%\usepackage{kpfonts}

% NOTA: Opción bastante bonita para textos de humanidades
% -> Garamond + math:
%\usepackage{ebgaramond}
%\usepackage{ebgaramond-maths}

% -> Barskervile + math
%\usepackage{baskervaldx}
%\usepackage[baskervaldx]{newtxmath}

% -> Charter + math
%\usepackage[charter,expert]{mathdesign}
%\usepackage[scaled=.96,osf]{XCharter}% matches the size used in math
%\linespread{1.04}

% -> Caladea (sin fuentes para modo matemático)
%\usepackage{caladea}

% -> Raleway (fuente sin serifas)
%\usepackage[default]{raleway}

\usepackage[T1]{fontenc} % Codificación de salida    
\usepackage[
    protrusion=true,
    activate={true,nocompatibility},
    final,
    tracking=true,
    kerning=true,
    spacing=true,
    factor=1100]{microtype}
\SetTracking{encoding={*}, shape=sc}{40}



\title{Ejemplo de tipografías con \LaTeX}
\author{Jesús Salido}
\date{\today}

%%%%%%%%%%%%%%
% Comienzo del documento
%%%%%%%%%%%%%%
\begin{document}
\maketitle


\section{Tipografías}
Por defecto, \LaTeX{} genera su salida con la tipografía \emph{Computer Modern} (\textbf{CM}) creada por D.~Knuth con ayuda del programa 
\textsc{metafont}. Por suerte, \LaTeX{} está configurado actualmente 
para incorporar estas <<fuentes>> en los ficheros PDF como una tipografía 
\emph{Postscript} Tipo 1 (vectorial). Con las fuentes Tipo 1 se obtiene una 
calidad superior a la obtenida con fuentes Tipo 3 o de mapa de bits 
(\textbf{PK}), que solo ofrecen máxima calidad a la escala con la que se
crearon. En la actualidad \LaTeX{} ofrece la posibilidad de 
incluir las tipografías estándar \emph{Postscript} de \textsf{Adobe} utilizando los paquetes apropiados. De este modo, es muy claro alternar en un texto entre las 
tres familias disponibles: Roman (redonda), \textsf{Sans Serif (paloseco, 
sin serifa o sin adornos)} y \texttt{Teletype (teletipo o monoespaciada)}. 
Y por último unas MAYÚSCULAS MÁS \textsc{pequeñas llamadas versalitas}.

Por favor, no utilices el \underline{subrayado} para \emph{enfatizar}. Aunque en algún documento (p.~ej.\ ejercicios) su uso es aceptable.

Por supuesto \LaTeX{} tiene en cuenta las particularidades de la tipografía 
en modo matemático como en: $f(x)=y^2$ y números $1,2,3,4,5,6,7,8,9$, o en 
modo texto: 0123456789.

$$g(x)=\int_{0}^{\infty}y^{3}dy$$

\end{document}
