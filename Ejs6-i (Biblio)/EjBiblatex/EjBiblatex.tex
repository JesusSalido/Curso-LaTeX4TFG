%%%%%%%%%%%%%% 
% Fichero: EjBiblatex
% Autor: J. Salido (http://www.uclm.es/profesorado/jsalido)
% Fecha: febrero, 2017
% Descripción: Método 3. Gestión de bibliografía con Bib(La)TeX.
% En este ejemplo se añaden referencias correctamente en el idioma 
% del documento.
% Ejemplo del curso: “LaTeX esencial para preparación de TFG, Tesis
% y otros documentos académicos” (Esc. Sup. Informática-UCLM)
%%%%%%%%%%%%%%



%%%%%%%%%%%%%%
% Preámbulo del documento
%%%%%%%%%%%%%%
\documentclass[11pt,a4paper]{article} 
\usepackage[utf8]{inputenx} 
\usepackage[spanish]{babel} 
\usepackage[left=2cm,
			right=2cm,
			top=2cm,
			bottom=2cm]{geometry} % Márgenes 

% Tipografía
\usepackage{marvosym}
\usepackage{pifont} % Generación de símbolos especiales
\usepackage{textcomp}
\usepackage{newpxtext}
\usepackage{newpxmath}
\usepackage[T1]{fontenc} % Codificación de salida    
\usepackage{microtype} % Mejoras de microtipografía en la obtención de PDF (sólo para pdflatex)

\usepackage{url} % Para escritura de URL
\urlstyle{sf}


% Generación de hiperenlaces
\usepackage[pdftex]{hyperref}
\hypersetup{
	breaklinks,
	colorlinks=true, % Pone color en los link o un borde
	linkcolor=red,    % Color de los links
	%	hidelinks,       % Oculta el color y borde de los links
	citecolor=blue,  % Color de la citas
	urlcolor=blue,   % Color de las URL
	bookmarksnumbered=true, % Incluye números en bookmarks		
	pdftitle={Fundamentos de LaTeX para principiantes}, % Título
	pdfauthor={Jesús Salido}, % Autor
	pdfsubject={LaTeX}, % Tema
	pdftoolbar=true, % Muestra la toolbar de Acrobat
	pdfmenubar=true % Muestra la menubar de Acrobat
}



\usepackage{paralist} % Mayor control de listas
\usepackage{multicol} % Elementos en varias columnas


% Con estas instrucciones se ajustan los valores del índice
\setcounter{secnumdepth}{1} % Ajusta el valor del último nivel numerado
\setcounter{tocdepth}{2} %Ajusta el valor del último nivel que aparece en TOC


%\usepackage[backend=biber,
%			sortcites,
%			style=numeric-comp]{biblatex}
%\usepackage[autostyle]{csquotes}

\usepackage[backend=biber,
	sortcites,
	style=numeric-comp]{biblatex}
\usepackage[autostyle]{csquotes}

% OJO: Línea añadida para eliminar el idioma de las fuentes bibliográficas que no están en el idioma principal del documento.
\AtEveryBibitem{\clearfield{note} \clearlist{language}}


% OJO: Poner siempre fichero con extensión
\addbibresource{../bibfiles/biblatex.bib}



\author{Jesús Salido}
\title{Bibliografía con Bib\LaTeX{}}
\date{\today}

%%%%%%%%%%%%%%
% Comienzo del documento
%%%%%%%%%%%%%%
\begin{document}
\maketitle


\begin{abstract}l
Explicación sobre la gestión de bibliografía con Bib\LaTeX{} para los documentos escritos con \LaTeX{} \cite{wikibookLaTex10}.
\end{abstract}

\tableofcontents

\section{La creación de bibliografía}
lEn la gran mayoría de documentos científicos~\cite[ver][5]{salido15} sus autores citan las fuentes consultadas durante la realización del trabajo presentado. \LaTeX{} proporciona herramientas muy flexibles para elaborar la \emph{lista de fuentes bibliográficas} e incluir las citas a ellas en el texto (ver~\cite{cascales00,cascales03,goos04,kopka04,lamport94}). La bibliografía aporta los detalles esenciales de los documentos externos citados y por lo general se imprime al final del documento principal como una sección separada del resto. El título de la sección varía según el tipo de documento principal y \LaTeX{} los denomina \emph{Referencias} o \emph{Bibliografía}, aunque dicho título se puede modificar fácilmente empleando el comando \texttt{renewcommand}.\footnote{Véase cómo se ha cambiado en este documento el título de dicha sección.}




\section{Bibliografías flexibles con Bib\LaTeX}
Bib\LaTeX{} es un paquete creado para superar las limitaciones de Bib\TeX. Los ficheros de fuentes bibliográficas siguen una estructura muy similar a los empleados con Bib\TeX{} que añade mayor flexibilidad a estos. Entre otros, es posible añadir los campos de \textsc{url} para las fuentes \emph{online}. Bib\LaTeX{} es compatible con \textsc{utf8} y proporciona una gran variedad de posibilidades de citación y configuración del estilo de la bibliografía. 



\addcontentsline{toc}{section}{Bibliografía} % Para añadir la bibliografía al TOC 

\nocite{*} % Se incluyen todas las fuentes bibliográficas aunque no hayan sido citadas en el texto.(PROHIBIDO EN LA ESI)
\printbibliography[title=Bibliografía]

\end{document}

