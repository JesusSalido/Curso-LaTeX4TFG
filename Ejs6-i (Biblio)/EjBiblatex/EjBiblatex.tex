% !BIB TS-program = biber
%%%%%%%%%%%%%% 
% Fichero: EjBiblatex
% Autor: J. Salido (http://www.esi.uclm.es/www/jsalido)
% Fecha: febrero, 2017
% Descripción: Gestión de bibliografía con Bib(La)TeX.
% En este ejemplo se añaden referencias correctamente en el idioma 
% del documento.
% Ejemplo del curso: “LaTeX esencial para preparación de TFG, Tesis
% y otros documentos académicos” (Esc. Sup. Informática-UCLM)
%%%%%%%%%%%%%%



%%%%%%%%%%%%%%
% Preámbulo del documento
%%%%%%%%%%%%%%
\documentclass[11pt,a4paper]{article} 
\usepackage[utf8]{inputenx} 
\usepackage[english,main=spanish]{babel} 
\usepackage[left=2cm,
			right=2cm,
			top=2cm,
			bottom=2cm]{geometry} % Márgenes 

% Tipografía
\usepackage{newpxtext}
\usepackage{newpxmath}

\usepackage{marvosym}
\usepackage{pifont} % Generación de símbolos especiales
\usepackage{textcomp}

\usepackage[T1]{fontenc} % Codificación de salida    
\usepackage{microtype} % Mejoras de microtipografía en la obtención de PDF (sólo para pdflatex)

\usepackage{url} % Para escritura de URL
\urlstyle{sf}


% Generación de hiperenlaces
\usepackage[pdftex]{hyperref}
\hypersetup{
	breaklinks,
	colorlinks=true, % Pone color en los link o un borde
	linkcolor=red,    % Color de los links
	%	hidelinks,       % Oculta el color y borde de los links
	citecolor=blue,  % Color de la citas
	urlcolor=blue,   % Color de las URL
	bookmarksnumbered=true, % Incluye números en bookmarks		
	pdftitle={Fundamentos de LaTeX para principiantes}, % Título
	pdfauthor={Jesús Salido}, % Autor
	pdfsubject={LaTeX}, % Tema
	pdftoolbar=true, % Muestra la toolbar de Acrobat
	pdfmenubar=true % Muestra la menubar de Acrobat
}



\usepackage{paralist} % Mayor control de listas
\usepackage{multicol} % Elementos en varias columnas


% Con estas instrucciones se ajustan los valores del índice
\setcounter{secnumdepth}{1} % Ajusta el valor del último nivel numerado
\setcounter{tocdepth}{2} %Ajusta el valor del último nivel que aparece en TOC



\usepackage[%
	backend=biber,
	% Estilos: numeric, numeric-comp, alphabetic, authoryear, authoryear-comp
	% Otros: apa, chicago, ieee, mla-new, iso-numeric, iso-authoryear
	% Estilos tradicionales BibTeX: trad-plain, trad-unsrt, trad-alpha y trad-abbrv
%	style=apa, % Estilo APA
	style=ieee,
	% Citación: numeric, numeric-comp, numeric-verb, 
	%           authoryear, authoryear-comp,...
	%           otros aplicados con estilo gral.: ieee,apa,aml,chem-acs,iso-numeric,iso-authoryear
	citestyle=numeric-comp,
	sortcites, % Ordenación de citas múltiples cuando son numéricas
	maxbibnames=3, % Máximo número listado de autores en la bibliografía
	minbibnames=1, % Mínimo número de autores cuando se abrevia la lista de autores
	% Descomentar las opciones siguientes para bibliografía multilingüe
	autolang=other, % Requerido para opción multilingüe
	language=auto,   % Requerido para opción multilingüe
	% Criterio de ordenación de las referencias.
	sorting=nyt%
					% Para cambiar criterio de ordenación de las referencias.
 					% =nty (name-title-year), nyt (name-year-title), nyvt (name-year-volume-title), 
					% =anyt (alphabetic-name-year-title), anyvt (alphabetic-name-year-volume-title), 
					% =ynt (year-name-title), ydnt (yeardescendent-name-title), 
					% =none (por orden de citación, como en ETSII-UCLM).			
	]{biblatex}


% OJO: Delaración para ajuste de indioma en estilo apa
\DeclareLanguageMapping{spanish}{spanish-apa}

\usepackage[autostyle]{csquotes}

% OJO: Línea añadida para eliminar el idioma de las fuentes bibliográficas que no están en el idioma principal del documento.
\AtEveryBibitem{\clearfield{note} \clearlist{language}}


% OJO: Poner siempre fichero con extensión
% Se pueden añadir varios ficheros si se desea
\addbibresource{biblatex.bib}



\author{Jesús Salido}
\title{Bibliografía con Bib\LaTeX{}}
\date{\today}

%%%%%%%%%%%%%%
% Comienzo del documento
%%%%%%%%%%%%%%
\begin{document}
\maketitle


\begin{abstract}
Explicación sobre la gestión de bibliografía con Bib\LaTeX{} para los documentos escritos con \LaTeX{} \cite{wikibookLaTex10}.
\end{abstract}

\tableofcontents

\section{La creación de bibliografía}
En la gran mayoría de documentos científicos \cite[ver][5]{salido15} sus autores citan las fuentes consultadas durante la realización del trabajo presentado. \LaTeX{} proporciona herramientas muy flexibles para elaborar la \emph{lista de fuentes bibliográficas} e incluir las citas a ellas en el texto \cite[ver][]{lamport94,cascales00,cascales03,goos04,kopka04}. La bibliografía aporta los detalles esenciales de los documentos externos citados y por lo general se imprime al final del documento principal como una sección separada del resto. El título de la sección varía según el tipo de documento principal y \LaTeX{} los denomina \emph{Referencias} o \emph{Bibliografía}, aunque dicho título se puede modificar fácilmente empleando el comando \texttt{renewcommand}.\footnote{Véase cómo se ha cambiado en este documento el título de dicha sección.}


\section{Bibliografías flexibles con Bib\LaTeX}
Bib\LaTeX{} es un paquete creado para superar las limitaciones de Bib\TeX. Los ficheros de fuentes bibliográficas siguen una estructura muy similar a los empleados con Bib\TeX{} a los que añade mayor flexibilidad. Entre otras mejoras, es posible añadir los campos de \textsc{url} para las fuentes \emph{online}. Bib\LaTeX{} es compatible con \textsc{utf8} y proporciona una gran variedad de posibilidades de citación y configuración del estilo de la bibliografía. 



\nocite{*} % Se incluyen todas las fuentes bibliográficas aunque no hayan sido citadas en el texto.(PROHIBIDO EN LA ESI)

% Inclusión de bibliografía en toc y con nombre indicado (def: Referencias)
\printbibliography[heading=bibintoc,title=Bibliografía]

\end{document}

