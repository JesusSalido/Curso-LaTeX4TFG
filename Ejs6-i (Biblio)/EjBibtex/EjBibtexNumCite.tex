%%%%%%%%%%%%%% 
% Fichero: EjBibtex
% Autor: J. Salido (http://www.esi.uclm.es/www/jsalido)
% Fecha: febrero, 2017
% Descripción: Gestión de bibliografía con BibTeX
% Ejemplo del curso: “LaTeX esencial para preparación de TFG, Tesis
% y otros documentos académicos” (Esc. Sup. Informática-UCLM)
%%%%%%%%%%%%%%




%%%%%%%%%%%%%%
% Preámbulo del documento
%%%%%%%%%%%%%%
\documentclass[11pt,a4paper]{article} 
\usepackage[spanish,es-noindentfirst]{babel} 
\usepackage[left=2cm,right=2cm,top=2cm,bottom=2cm]{geometry} % Márgenes 
\usepackage[skip=.3\baselineskip plus 2pt,indent]{parskip} % Salto entre párrafos 
% skip= .5\baselineskip plus 2pt -> Valor por defecto


% Tipografía
\usepackage{newpxtext}
\usepackage{newpxmath}

\usepackage{marvosym,pifont,textcomp,fontawesome5}

\usepackage[T1]{fontenc} % Codificación de salida    
\usepackage[
    protrusion=true,
    activate={true,nocompatibility},
    final,
    tracking=true,
    kerning=true,
    spacing=true,
    factor=1100]{microtype}
\SetTracking{encoding={*}, shape=sc}{40}


% Definición de colores
\usepackage[table,dvipsnames,svgnames,x11names]{xcolor}
\definecolor{sombra}{HTML}{F0F0F0} % Sombra código
\definecolor{gris}{gray}{0.4} % Gris numeración código
\definecolor{palered}{rgb}{0.78, 0.03, 0.08}
\definecolor{ocre}{RGB}{243,102,25} % Ocre
\definecolor{aquaESI}{RGB}{0,151,215} % Aqua
\definecolor{turquesa}{RGB}{64,132,64} % Turquesa
\definecolor{UCLMred}{cmyk}{0.19, 1.0, 0.74, 0.12} % Color imagen corporativa UCLM



% Generación de hiperenlaces
\usepackage[pdftex]{hyperref}
\hypersetup{
	breaklinks,
	colorlinks=true, % Pone color en los link o un borde
    linktocpage=true,% enlace al nº de pág., false=texto completo
	linkcolor=UCLMred,   % Color de los links
	%	hidelinks,   % Oculta el color y borde de los links
	citecolor=aquaESI,  % Color de la citas
	urlcolor=aquaESI,   % Color de las URL
	bookmarksnumbered=true, % Incluye números en bookmarks		
	pdftoolbar=true, % Muestra la toolbar de Acrobat
	pdfmenubar=true % Muestra la menubar de Acrobat
}
\urlstyle{sf}



\usepackage[shortlabels]{enumitem} % Mayor control de listas
\usepackage{multicol} % Elementos en varias columnas


% Con estas instrucciones se ajustan los valores del índice
\setcounter{secnumdepth}{1} % Ajusta el valor del último nivel numerado
\setcounter{tocdepth}{2} %Ajusta el valor del último nivel que aparece en TOC

\author{Jesús Salido}
\title{Bibliografía con Bib\TeX{} y citación numérica}
\date{\today}

%%%%%%%%%%%%%%
% Comienzo del documento
%%%%%%%%%%%%%%
\begin{document}
\maketitle


\begin{abstract}
Generación de bibliografía con \LaTeX{} empleando un método más elegante consistente en la utilización de Bib\TeX{} \cite{wikibookLaTex10} y estilo de citación numérica.
\end{abstract}



\section{La creación de bibliografía}
En la gran mayoría de documentos científicos~\cite[pág.~5]{salido15} sus autores citan las fuentes consultadas durante la realización del trabajo presentado. \LaTeX{} proporciona herramientas muy flexibles para elaborar la \emph{lista de fuentes bibliográficas} e incluir las citas a ellas en el texto (ver~\cite{cascales00,cascales03,goos04,kopka04,lamport94}). La bibliografía aporta los detalles esenciales de los documentos externos citados y por lo general se imprime al final del documento principal como una sección separada del resto. El título de la sección varía según el tipo de documento principal y \LaTeX{} los denomina \emph{Referencias} o \emph{Bibliografía}, aunque dicho título se puede modificar fácilmente empleando el comando \texttt{renewcommand}.\footnote{Véase cómo se ha cambiado en este documento el título de dicha sección.}


El conjunto de referencias que serán citadas en el documento principal puede contemplarse como una \emph{base de datos bibliográfica} en la que cada registro (o entrada) contiene la información relevante del documento citado. Dicha base bibliográfica puede ser:
\begin{enumerate}
	\item \emph{Interna o autocontenida} en el propio documento \LaTeX{}.\\
	 En este caso se emplea el entorno \texttt{thebibliography}.
	
	\item Externa al documento fuente. En este caso los registros de la base bibliográfica están contenidos en un fichero \texttt{.bib} de texto plano (sin formato, como \texttt{.tex}).
\end{enumerate}

Para implementar el segundo método se utiliza una herramienta adicional denominada Bib\TeX{} (comando \texttt{bibtex} en las distribuciones de \LaTeX). Este programa se encarga de procesar ficheros (\texttt{.bib}) de texto que sólo contienen registros de fuentes bibliográficas en los que constan todos los datos relativos a la fuente. El fichero \texttt{.bib} puede ser contemplado como un archivo de una base de datos bibliográfica. En este caso Bib\TeX{} junto con \LaTeX{} gestionan el estilo en que se imprime la bibliografía e incluso el de las citas. Con este procedimiento la tarea de elaboración de una bibliografía consiste en la creación de la base de datos con las referencias bibliográficas deseadas. Por supuesto los registros tendrán que respetar una estructura definida que Bib\TeX{} pueda comprender.

Cada registro de la base bibliográfica proporciona información sobre el documento concreto al que se refiere:
\begin{itemize}[noitemsep]
	\item Autor(es).
	\item Título.
	\item Revista, libro, congreso u otra forma de publicación del documento.
	\item Número, volumen, etc.
	\item Editorial.
	\item Fecha de publicación.
\end{itemize}



\section{Elección del método más apropiado}
En mi opinión el procedimiento apropiado para la inclusión de la bibliografía depende del tamaño de ésta y las características deseadas. Las principales ventajas de la inclusión directa de la bibliografía en el documento presenta graves inconvenientes: 
\begin{itemize}
	\item Reutilización tediosa (básicamente se trata de un <<corta y pega>>), y
	\item Dificultad para mantener la homogeneidad del estilo en la bibliografía, tanto más cuanto más voluminosa es esta.
\end{itemize}

Por el contrario, el trabajo con bases bibliográficas externas (\texttt{.bib}) al documento fuente tiene grandes ventajas:
\begin{itemize}
	\item Reutilización sencilla e inmediata en diferentes documentos de trabajo,
	\item Facilidad para compartir las bases bibliográficas entre varios colaboradores, y
	\item Facilidad para emplear diferentes ficheros de bibliografía en un mismo documento (se evita que los ficheros sean muy voluminosos y puedan organizarse mejor)
\end{itemize}

Aunque son numerosas las ventajas del empleo de Bib\TeX{} para la elaboración de la bibliografía existen algunas limitaciones que conviene tener presentes:
\begin{itemize}
	\item No puede tratar con bibliografías multilingües. Esto es, aquellas en que las fuentes están en más de un idioma (p.~ej.\ español e inglés).
	
	\item Bib\TeX{} no es compatible con UTF8 de modo que aunque los ficheros de bibliografía se pueden codificar así con algunos estilos bibliográficos se obtienen errores.
\end{itemize}




%%%%%%
% Bibliografía

\renewcommand{\refname}{Bibliografía} % Cambio de nombre de la sección
\addcontentsline{toc}{section}{Bibliografía} % Para añadir la bibliografía al TOC 

%%%%%
% Definición del estilo empleado en la bibliografía
%\bibliographystyle{plain}
% Estilos nativos incluidos con LaTeX (plain, abbrv, alpha, unsrt).
%
% plain: las referencias se numeran y en la bibliografía las entradas
%        aparecen en orden alfabético.
% abbrv: igual que el anterior pero en la bibliografía los nombres se
%        escriben sólo con la inicial.
% unsrt: la bibliografía no aparece por orden alfabético sino por 
%        orden de cita en el texto.

% Estilos incluidos con BibTeX “no nativos” pero muy “populares” para ingenierías
%
% acm:       Numérica refs. ordenadas por orden alfabético y nombres de
%            autores en mayúsculas.
% ieeetr:    Numérica para los IEEE Transactions, referencias ordenadas
%            por orden de cita en el texto.
% 
% Estilo incluido con BibTeX para citar en formato autor-año
% apacite:   No numérica, con referencias ordenadas alfabéticamente por apellido de autor.


%\nocite{*} % OJO: Con \nocite{*} se incluyen todas las fuentes bibliográficas aunque no hayan sido citadas en el texto. CUIDADO todas las referencias en la bibliografía deben estar citadas en el texto.
%\bibliography{bibtex} % Nombre del fichero .bib (sin extensión)

\begin{thebibliography}{1}

\bibitem{Bloch17}
Joshua Bloch.
\newblock {\em Effective {J}ava}.
\newblock Addison-Wesley Professional, 2017.

\bibitem{cascales00}
B.~Cascales, P.~Lucas, J.~M. Mira, A.~Pallarés, y S.~Sánchez-Pedreño.
\newblock {\em {\LaTeX{}}. Una imprenta en sus manos}.
\newblock Aula Documental de Investigación, 2000.

\bibitem{cascales03}
B.~Cascales, P.~Lucas, J.M. Mira, A.~Pallarés, y S.~{Sánchez-Pedreño}.
\newblock {\em El libro de {\LaTeX}}.
\newblock Pearson/Prentice Hall, 2003.

\bibitem{goos04}
M.~Goossens, F.~Mittelbach, and A.~Samarin.
\newblock {\em The {\LaTeX{}} companion}.
\newblock Addison-Wesley Reading, MA, 2nd edition, 2004.

\bibitem{kopka04}
H.~Kopka and {P.W.} Daly.
\newblock {\em A guide to {\LaTeX}}.
\newblock Addison-Wesley, 4th edition, 2004.

\bibitem{lamport94}
L.~Lamport.
\newblock {\em {\LaTeX}: A document preparation system}.
\newblock Addison-Wesley, 2nd edition, 1994.

\bibitem{salido15}
Jesús Salido.
\newblock Curso de {\LaTeX2e{}}.
\newblock {URL:} \url{http://visilab.etsii.uclm.es/?page_id=1468}, 2015.
\newblock Último acceso: 7 sep. 2021.

\bibitem{wikibookLaTex10}
WikiMedia.
\newblock {\LaTeX{} Wikibook}.
\newblock {URL:} \url{http://en.wikibooks.org/wiki/LaTeX}, 2010.
\newblock Último acceso 23 feb. 2011.

\end{thebibliography}

%\begin{thebibliography}{1}
%
%\bibitem{Bloch17}
%Joshua Bloch.
%\newblock {\em Effective {J}ava}.
%\newblock Addison-Wesley Professional, 2017.
%
%\bibitem{cascales00}
%B.~Cascales, P.~Lucas, J.~M. Mira, A.~Pallarés, y S.~Sánchez-Pedreño.
%\newblock {\em {\LaTeX{}}. Una imprenta en sus manos}.
%\newblock Aula Documental de Investigación, 2000.
%
%\bibitem{cascales03}
%B.~Cascales, P.~Lucas, J.M. Mira, A.~Pallarés, y S.~{Sánchez-Pedreño}.
%\newblock {\em El libro de {\LaTeX}}.
%\newblock Pearson/Prentice Hall, 2003.
%
%\bibitem{goos04}
%M.~Goossens, F.~Mittelbach, and A.~Samarin.
%\newblock {\em The {\LaTeX{}} companion}.
%\newblock Addison-Wesley Reading, MA, 2nd edition, 2004.
%
%\bibitem{kopka04}
%H.~Kopka and {P.W.} Daly.
%\newblock {\em A guide to {\LaTeX}}.
%\newblock Addison-Wesley, 4th edition, 2004.
%
%\bibitem{lamport94}
%L.~Lamport.
%\newblock {\em {\LaTeX}: A document preparation system}.
%\newblock Addison-Wesley, 2nd edition, 1994.
%
%\bibitem{salido15}
%Jesús Salido.
%\newblock Curso de {\LaTeX2e{}}.
%\newblock {URL:} \url{http://visilab.etsii.uclm.es/?page_id=1468}, 2015.
%\newblock Último acceso: 7 sep. 2021.
%
%\bibitem{wikibookLaTex10}
%WikiMedia.
%\newblock {\LaTeX{} Wikibook}.
%\newblock {URL:} \url{http://en.wikibooks.org/wiki/LaTeX}, 2010.
%\newblock Último acceso 23 feb. 2011.
%
%\end{thebibliography}


\end{document}

