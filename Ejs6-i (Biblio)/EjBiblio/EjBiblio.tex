%%%%%%%%%%%%%% 
% Fichero: EjBiblio
% Autor: J. Salido (http://www.esi.uclm.es/www/jsalido)
% Fecha: febrero, 2017
% Descripción: Bibliografía incluida en el fichero tex
% empleando el entorno thebibliography
% Ejemplo del curso: “LaTeX esencial para preparación de TFG, Tesis
% y otros documentos académicos” (Esc. Sup. Informática-UCLM)
%%%%%%%%%%%%%%



%%%%%%%%%%%%%%
% Preámbulo del documento
%%%%%%%%%%%%%%
\documentclass[11pt,a4paper]{article} 
\usepackage[spanish,es-noindentfirst]{babel} 
\usepackage[left=2cm,
			right=2cm,
			top=2cm,
			bottom=2cm]{geometry} % Márgenes 

% Tipografía
\usepackage{newpxtext}
\usepackage{newpxmath}

\usepackage{marvosym,pifont,textcomp,fontawesome5}

\usepackage[T1]{fontenc} % Codificación de salida    
\usepackage[
    protrusion=true,
    activate={true,nocompatibility},
    final,
    tracking=true,
    kerning=true,
    spacing=true,
    factor=1100]{microtype}
\SetTracking{encoding={*}, shape=sc}{40}


% Definición de colores
\usepackage[table,dvipsnames,svgnames,x11names]{xcolor}
\definecolor{sombra}{HTML}{F0F0F0} % Sombra código
\definecolor{gris}{gray}{0.4} % Gris numeración código
\definecolor{palered}{rgb}{0.78, 0.03, 0.08}
\definecolor{ocre}{RGB}{243,102,25} % Ocre
\definecolor{aquaESI}{RGB}{0,151,215} % Aqua
\definecolor{turquesa}{RGB}{64,132,64} % Turquesa
\definecolor{UCLMred}{cmyk}{0.19, 1.0, 0.74, 0.12} % Color imagen corporativa UCLM



% Generación de hiperenlaces
\usepackage[pdftex]{hyperref}
\hypersetup{
	breaklinks,
	colorlinks=true, % Pone color en los link o un borde
    linktocpage=true,% enlace al nº de pág., false=texto completo
	linkcolor=UCLMred,   % Color de los links
	%	hidelinks,   % Oculta el color y borde de los links
	citecolor=aquaESI,  % Color de la citas
	urlcolor=aquaESI,   % Color de las URL
	bookmarksnumbered=true, % Incluye números en bookmarks		
	pdftoolbar=true, % Muestra la toolbar de Acrobat
	pdfmenubar=true % Muestra la menubar de Acrobat
}
\urlstyle{sf}



\usepackage[shortlabels]{enumitem} % Mayor control de listas
\usepackage{multicol} % Elementos en varias columnas


% Con estas instrucciones se ajustan los valores del índice
\setcounter{secnumdepth}{1} % Ajusta el valor del último nivel numerado
\setcounter{tocdepth}{2} %Ajusta el valor del último nivel que aparece en TOC


\author{Jesús Salido}
\title{Bibliografía con el entorno \texttt{thebibliography}}
\date{\today}

%%%%%%%%%%%%%%
% Comienzo del documento
%%%%%%%%%%%%%%
\begin{document}


\maketitle


\begin{abstract}
	Explicación sobre la generación de bibliografía para los documentos escritos con \LaTeX{}. En este caso se muestra el método más sencillo consistente en que la bibliografía está incluida directamente en el fichero de texto. Para ello se emplea en entorno \texttt{thebibliography}.
\end{abstract}

\hrule
\tableofcontents
\bigskip
\hrule

\section{Introducción}
Todos los documentos técnicos (TFG, TFM, Tesis, artículo científico, memoria, etc.) deben incluir una sección de bibliografía en el que se hace un listado de todas las fuentes consultadas para realizar el trabajo. Es preciso otorgar el crédito a los trabajos llevados a cabo por otros cuyas ideas se utilizan de algún modo u otro en el trabajo propio. Por esta razón en el texto se debe citar\footnote{Similar a las referencias cruzadas pero en este caso la referencia es a una fuente bibliográfica que aparece listada en la sección de bibliografía.} las fuentes empleadas intentando evitar incurrir en plagio \cite{usbplagio2010} aun cuando esta no sea nuestra intención. Todas las fuentes usadas deberían aparecer citadas en nuestro trabajo. Las referencias bibliográficas pueden consistir en: un libro, un capítulo o sección de libro, un artículo de revista o congreso, un manual técnico, una Tesis, un TFG, etcétera.

Una de las dificultades que los autores encuentran al realizar la bibliografía y su citación es mantener la coherencia. Suele suceder que al hacer manualmente la bibliografía la inclusión de nuevas fuentes puede alterar el sistema de numeración (si la citación es de tipo numérico) y convertir en un <<dolor de cabeza>> la tarea de mantener un sistema coherente. Expliquemos esto en más detalle. Al elaborar este documento puedo darme cuenta de que olvidé incluir una fuente bibliográfica que deseo añadir (p.~ej.\ un libro). En ese momento cómo deseo que el listado de fuentes en la bibliografía aparezca por orden alfabético de autores, esta nueva fuente puede alterar la numeración previa de las citas. Si todo el proceso lo estoy haciendo manualmente, la tarea de mantenimiento de las citas en el texto puede convertirse en un trabajo demasiado <<ingrato>>. Para solventar esta cuestión la gran mayoría de procesadores actuales incluyen algún sistema de citación a las fuentes de la bibliografía,\footnote{Mi experiencia personal es que en general en los textos preparados por el alumnado, apenas se citan las fuentes empleadas como referencias.} \LaTeX{} no es una excepción y para ello permite asignar una etiqueta (\emph{label}) a cada fuente que se utiliza al citarla. 









\section{Métodos de creación de bibliografía}
\LaTeX{} aporta varios métodos de elaboración de bibliografía. El primero de ellos utiliza un entorno dedicado\footnote{ \texttt{thebibliography}} con la bibliografía incluida explícitamente en el fichero del documento. Dicho entorno funciona como una lista especial en la que se aplica el formato directamente en el texto. Este documento emplea a modo de ejemplo este tipo de bibliografía. Este procedimiento es apropiado cuando el documento tiene un número reducido de fuentes bibliográficas y no se exige un estilo muy estricto para el formateado final de la bibliografía. Todas las fuentes aparecen en el orden que se incluyen en el entorno \texttt{thebibliography} y con el formateado que allí se indique. La citación a la bibliografía es de tipo numérico. Esto es, consiste en un número entre corchetes que es generado de forma automática a partir de la etiqueta asociada a la referencia y el orden de esta en la lista.
	

\noindent Este método tiene las ventajas siguientes:
\begin{itemize}[\ding{51}]
	\item Es sencillo de llevar a cabo.
	
	\item Se puede emplear el formato deseado a cada una de las fuentes de bibliografía, incluso mezclando idiomas.
\end{itemize}

\noindent Como principales inconvenientes se puede señalar:
\begin{itemize}[\ding{55}]
	\item La ordenación de las fuentes se debe hacer de modo manual.
	
	\item El formato de las fuentes no se maneja con estilos y, por tanto, es manual. La modificación del formato es tediosa.
	
	\item El mantenimiento de la bibliografía se complica bastante cuando aumenta el número de fuentes.
	
	\item La bibliografía es reutilizable, pero con manipulación no automatizada.
\end{itemize}


Para consultar más información sobre el modo de elaboración de bibliografías con \LaTeX{} puedes consultar los textos más populares de referencia en diferentes idiomas \cite[en inglés]{KopDal2003,Lam1994,MitGooBra2004} \cite[en español]{CasLucMir2000,CasLucMir2003}. Aunque el documento \cite{lshort2009} se ha recomendado para este curso, se trata sólo de una guía rápida que no entra en detalles sobre la elaboración de bibliografías. Para una referencia rápida y bastante completa sobre las bibliografías se recomienda especialmente las fuentes electrónicas \cite{Rob2010,wiki2010} disponibles en Internet.

\LaTeX{}, por defecto, denomina como <<Referencias>> a la sección con el listado de fuentes bibliográficas. Por esta razón si se decide cambiar el nombre para que este sea <<Bibliografía>> hay que modificar el comportamiento de \LaTeX. Por defecto la sección de bibliografía no está numerada y no aparece en el índice de contenidos (TOC ---del inglés, \emph{Table of Contents}---).


%%%%%%
% Bibliografía
% Inclusión explícita de las fuentes bibliográficas en el formáto deseado
\renewcommand{\refname}{Bibliografía} % Cambio de título de la sección a Bibliografía <- Referencias
%%
\addcontentsline{toc}{section}{Bibliografía} % Para añadir la bibliografía al TOC

\begin{thebibliography}{99} % El string debe ser más largo que la referencia más larga del documento
	\bibitem{CasLucMir2000} B. Cascales, P. Lucas, J. M. Mira, A. Pallarés y S. Sánchez-Pedreño. \emph{\LaTeX. Una imprenta en sus manos}. Aula Documental de Investigación, Madrid (2000).

	\bibitem{CasLucMir2003} B. Cascales, P. Lucas, J. M. Mira, A. Pallarés y S. Sánchez-Pedreño. \emph{El libro de \LaTeX}. Pearson Educación, Madrid (2003).
	
	\bibitem{usbplagio2010} A. García y E. Klein. \emph{¿Por qué ocurre el plagio en las Universidades y cómo evitarlo?} Universidad Simón Bolívar. Último acceso: 10 febrero, 2010.\\
\url{http://prof.usb.ve/eklein/plagio/}

	\bibitem{Knu1986} D. E. Knuth. \emph{The \TeX{}book}. Addison-Wesley, Reading (1986).

	\bibitem{KopDal2003} H. Kopka y P. W. Daly. \emph{A Guide to \LaTeX} (4\sptext{a} edición). Addison-Wesley, Reading (2003).
	
	\bibitem{Lam1994} L. Lamport \emph{\LaTeX: A Document Preparation System} (2\sptext{a} edición). Addison-Wesley, Reading (1994).
	
	\bibitem{MitGooBra2004} F. Mittelbach, M. Goossens, J. Braams, D. Carlisle y C. Rowley. \emph{The \LaTeX{} Companion} (2\sptext{a} edición). Addison-Wesley, Reading (2004).
	
	\bibitem{lshort2009} T. Oetiker et al. \emph{La introducción no-tan-corta a LaTeX2e}. Versión 4.20, (2009).

	\bibitem{Rob2010} Andrew Roberts. \emph{Bibliography Management}. Último acceso: 10 febrero, 2010.\\
\url{http://www.andy-roberts.net/misc/latex/latextutorial3.html}

	\bibitem{wiki2010} Wikibooks LaTeX. \emph{Bibliography Management}. Último acceso: 10 febrero, 2010.\\
\url{http://en.wikibooks.org/wiki/LaTeX/Bibliography_Management}
\end{thebibliography}



\end{document}
