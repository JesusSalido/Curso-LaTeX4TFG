%%%%%%%%%%%%%% 
% Fichero: EjBiblatexx
% Autor: J. Salido (http://www.uclm.es/profesorado/jsalido)
% Fecha: febrero, 2017
% Descripción: Método 3. Gestión de bibliografía con Bib(La)TeX
% Este ejemplo aún presenta errores pues trabaja con un fichero
% de bibliografía incompleto.
% Ejemplo del curso: “LaTeX esencial para preparación de TFG, Tesis
% y otros documentos académicos” (Esc. Sup. Informática-UCLM)
%%%%%%%%%%%%%%




%%%%%%%%%%%%%%
% Preámbulo del documento
%%%%%%%%%%%%%%
\documentclass[11pt,a4paper]{article} 
\usepackage[utf8]{inputenx} 
\usepackage[spanish]{babel} 
\usepackage[left=2cm,right=2cm,top=2cm,bottom=2cm]{geometry} % Márgenes 

% Tipografía
\usepackage{marvosym}
\usepackage{pifont} % Generación de símbolos especiales
\usepackage{textcomp}
\usepackage{newpxtext}
\usepackage{newpxmath}
\usepackage[T1]{fontenc} % Codificación de salida    
\usepackage{microtype} % Mejoras de microtipografía en la obtención de PDF (sólo para pdflatex)

\usepackage{url} % Para escritura de URL
\urlstyle{sf}


% Generación de hiperenlaces
\usepackage[pdftex]{hyperref}
\hypersetup{
	breaklinks,
	colorlinks=true, % Pone color en los link o un borde
	linkcolor=red,    % Color de los links
	%	hidelinks,       % Oculta el color y borde de los links
	citecolor=blue,  % Color de la citas
	urlcolor=blue,   % Color de las URL
	bookmarksnumbered=true, % Incluye números en bookmarks		
	pdftitle={Fundamentos de LaTeX para principiantes}, % Título
	pdfauthor={Jesús Salido}, % Autor
	pdfsubject={LaTeX}, % Tema
	pdftoolbar=true, % Muestra la toolbar de Acrobat
	pdfmenubar=true % Muestra la menubar de Acrobat
}


\usepackage{paralist} % Mayor control de listas
\usepackage{multicol} % Elementos en varias columnas



% Con estas instrucciones se ajustan los valores del índice
\setcounter{secnumdepth}{1} % Ajusta el valor del último nivel numerado
\setcounter{tocdepth}{2} %Ajusta el valor del último nivel que aparece en TOC


\usepackage[backend=biber,
			sortcites]{biblatex}
\usepackage[autostyle]{csquotes}

% OJO: Poner siempre fichero con extensión
\addbibresource{../bibfiles/bibtex.bib} 

\author{Jesús Salido}
\title{Bibliografía con Bib(La)\TeX{}}
\date{\today}

%%%%%%%%%%%%%%
% Comienzo del documento
%%%%%%%%%%%%%%
\begin{document}
\maketitle


\begin{abstract}
Explicación sobre la gestión de bibliografía con Bib(La)\TeX{} para los documentos escritos con \LaTeX{} \cite{wikibookLaTex10}.
\end{abstract}

\tableofcontents

\section{La creación de bibliografía}
En la gran mayoría de documentos científicos~\cite{salido15} sus autores citan las fuentes consultadas durante la realización del trabajo presentado. \LaTeX{} proporciona herramientas muy flexibles para elaborar la \emph{lista de fuentes bibliográficas} e incluir las citas a ellas en el texto (ver~\cite{cascales00,cascales03,goos04,kopka04,lamport94}). La bibliografía aporta los detalles esenciales de los documentos externos citados y por lo general se imprime al final del documento principal como una sección separada del resto. El título de la sección varía según el tipo de documento principal y \LaTeX{} los denomina \emph{Referencias} o \emph{Bibliografía}, aunque dicho título se puede modificar fácilmente empleando el comando \texttt{renewcommand}.\footnote{Véase cómo se ha cambiado en este documento el título de dicha sección.}

El conjunto de referencias que serán citadas en el documento principal puede contemplarse como una \emph{base de datos bibliográfica} en la que cada registro (o entrada) contiene la información relevante del documento citado. Dicha base bibliográfica puede ser:
\begin{enumerate}
	\item \emph{Interna o autocontenida} en el propio documento \LaTeX{}.\\
	 En este caso se emplea el entorno \texttt{thebibliography}.
	
	\item Externa al documento fuente. En este caso los registros de la base bibliográfica están contenidos en un fichero \texttt{.bib} de texto plano (sin formato, como \texttt{.tex}).
\end{enumerate}



Cada registro de la base bibliográfica proporciona información sobre el documento concreto al que se refiere:
\begin{itemize}
	\item Autor(es).
	\item Título.
	\item Revista, libro, congreso u otra forma de publicación del documento.
	\item Número, volumen, etc.
	\item Editorial.
	\item Fecha de publicación.
\end{itemize}



\section{Elección del método}
En mi opinión el procedimiento apropiado para la inclusión de la bibliografía depende del tamaño de ésta y de las desventajas que tienen los procedimientos que incluyen los registros bibliográficos en el documento fuente, a saber: 
\begin{itemize}
	\item la reutilización de la base bibliográfica se hace tediosa (básicamente se trata de un <<corta y pega>>), y
	\item cuanto mayor es el número de elementos en la lista de bibliografía más difícil es mantener homogéneo el formato de dicha lista.
\end{itemize}

Por el contrario, el trabajo con bases bibliográficas externas (\texttt{.bib}) al documento fuente tiene grandes ventajas:
\begin{itemize}
	\item los registros se pueden reutilizar en diferentes documentos de trabajo,
	\item las bases bibliográficas se pueden compartir entre un conjunto de colaboradores, y
	\item las bases bibliográficas no tienen porqué ser de gran tamaño pues pueden combinarse varias de ellas en el documento de trabajo.
\end{itemize}

\section{Limitaciones de Bib\TeX}
Para tratar con las bases bibliográficas externas \LaTeX{} ha empleado durante muchos años la herramienta Bib\TeX{}. Sin embargo, esta herramienta presenta varias limitaciones importantes:
\begin{enumerate}
	\item No está preparado para la inclusión de fuentes bibliográficas \emph{online}, como páginas web, \emph{blogs} y similares.
	
	\item No es compatible con la codificación \textsc{utf8}. Aunque si se emplean estilos de bibliografía numéricos no genera errores.
	
	\item No está preparado para adaptar la bibliografía al idioma del documento. Ya que esta se trata como si fuese en inglés. Por supuesto cuando se emplean varios idiomas en la bibliografía dichas variaciones no se contemplan.
\end{enumerate}


\section{Bibliografías flexibles con Bib\LaTeX}
Bib\LaTeX{} es un paquete creado para superar las limitaciones de Bib\TeX. Los ficheros de fuentes bibliográficas siguen una estructura muy similar a los empleados con Bib\TeX{} que añade mayor flexibilidad a estos. Entre otros, es posible añadir los campos de \textsc{url} para las fuentes \emph{online}. Bib\LaTeX{} es compatible con \textsc{utf8} y proporciona una gran variedad de posibilidades de citación y configuración del estilo de la bibliografía. 





\addcontentsline{toc}{section}{Bibliografía} % Para añadir la bibliografía al TOC 

\nocite{*} % Se incluyen todas las fuentes bibliográficas aunque no hayan sido citadas en el texto.(PROHIBIDO EN LA ESI)
\printbibliography[title=Bibliografía]

\end{document}

