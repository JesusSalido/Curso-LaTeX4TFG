%%%%%%%%%%%%%% 
% Fichero: EjTablas
% Autor: J. Salido (http://www.esi.uclm.es/www/jsalido)
% Fecha: febrero, 2017
% Descripción: Ejemplo básico de inclusión de tablas.
% Ejemplo del curso: “LaTeX esencial para preparación de TFG, Tesis
% y otros documentos académicos” (Esc. Sup. Informática-UCLM)
%%%%%%%%%%%%%%




%%%%%%%%%%%%%%
% Preámbulo del documento
%%%%%%%%%%%%%%
\documentclass[11pt,a4paper]{article} 
\usepackage[spanish,es-tabla,es-noindentfirst]{babel} % Opción para cambio de nombre
\usepackage[left=2cm,right=2cm,top=2cm,bottom=2cm]{geometry} % Márgenes 

% Tipografía
\usepackage{newpxtext}
\usepackage{newpxmath}

\usepackage{marvosym}
\usepackage{pifont} % Generación de símbolos especiales
\usepackage{textcomp}

\usepackage[T1]{fontenc} % Codificación de salida    
\usepackage{microtype} % Mejoras de microtipografía en la obtención de PDF (sólo para pdflatex)

\usepackage{siunitx} % Formateado de números y unidades en el SI.

% NOTA: Este paquete incluido más abajo produce -> LaTeX Error: Option clash for package xcolor
% Color
\usepackage[table,hyperref,usenames,dvipsnames,svgnames]{xcolor}

% Generación de hiperenlaces
\usepackage[%
   pdftex,
   breaklinks,
   hidelinks=true,      % Oculta colores en los enlaces (negro)
    linktocpage=true,    % true = enlace al nº de pág., false=texto completo
%    colorlinks=true,         % true=colorea texto del enlace, false=recuadra el texto
	citecolor=red, % Color de la citas
	urlcolor=blue, % Color de las URL
	bookmarksnumbered=true % Incluye números en bookmarks
]{hyperref}
\usepackage{url}
\urlstyle{sf} % Estilo de URL sin serifas

% Listas
\usepackage{enumitem} % Mayor control de listas
\usepackage{multicol} % Elementos en varias columnas

% Gráficos
\usepackage{graphicx}  % Inclusión de figuras y escalado de cajas
\usepackage{float} % Control de posición de objetos flotantes y estilos
\usepackage[margin=10pt,labelfont=bf]{caption}
\usepackage[margin=10pt,font=small,labelfont=bf]{subcaption}	% Inclusión de subfiguras
\usepackage{rotating}	% Rotación de figuras
\usepackage{pdflscape}

\graphicspath{{../figs/}}  
\DeclareGraphicsExtensions{.pdf,.png,.jpg}

% Tablas
\usepackage{array,tabularx,booktabs} % Tablas más complejas
\usepackage{multirow}% Tablas con celdas extendidas
\captionsetup[table]{skip=5pt} 	% Separación del caption en las tablas aumentando el valor por defecto 

% Con estas instrucciones se ajustan los valores del índice
\setcounter{secnumdepth}{2} % Ajusta el valor del último nivel numerado
\setcounter{tocdepth}{2} %Ajusta el valor del último nivel que aparece en TOC


\author{Jesús Salido}
\title{Inclusión de tablas complejas en \LaTeX{}}
\date{\today}


%%%%%%%%%%%%%%
% Comienzo del documento
%%%%%%%%%%%%%%
\begin{document}
\maketitle
\begin{abstract}
	Explicación sobre la inclusión de tablas complejas en \LaTeX{}.
\end{abstract}

\hrule
\tableofcontents
\listoftables
\bigskip
\hrule



\newpage
\section{Tablas profesionales}
Un pequeño consejo antes de introducir las líneas; en las tablas las líneas deben ser usadas con mesura y normalmente las líneas verticales no parecen ser de un uso muy profesional. De hecho, para las tablas profesionales no debe usar ninguna de las líneas estándar; en su lugar usted debe familiarizarse con las que le facilita el paquete \texttt{booktabs}. Este paquete proporciona cuatro tipos diferentes de líneas. Cada uno de los comandos que permiten definirlas, deben ser el primer elemento de una fila o seguir a otra línea ya definida. Tres de estos comandos son: \textbackslash \texttt{toprule}, \textbackslash \texttt{midrule} y \textbackslash \texttt{bottomrule} que son usados para situar una línea en la parte alta, en las filas intermedias o en la parte baja de la tabla, respectivamente, 

El cuarto comando proporcionado por \texttt{booktabs} es \textbackslash \texttt{cmidrule}. Puede ser usado para dibujar una línea que no se extienda a toda la fila de una tabla, sino a un intervalo específico de columnas de esa fila. Debe especificarse un intervalo de columnas de la forma: \texttt{\{n-n\}}. Incluso si se desea dibujar una línea para una única columna, se debe especificar como un intervalo (siendo los extremos del intervalo el mismo número).

\begin{table}[H]
   \centering
   	  \caption{Tabla con paquete \texttt{booktabs}}
   	  \label{tab:booktabs}      
   	  \begin{tabular}{llr}
      \toprule
      \multicolumn{2}{c}{Item} \\
      \cmidrule(r){1-2}
      Animal & Description & Price (\$) \\
      \midrule
      Gnat  & per gram & 13.65 \\
            & each     &  0.01 \\
      Armadillo & frozen & 8.99 \\
      \bottomrule
      \end{tabular}
\end{table}

Existe otro uso útil de \texttt{\textbackslash cmidrule}. Puede acortar el principio o fin de una línea o incluso con un argumento opcional entre paréntesis, donde \texttt{r} y \texttt{l} significan que la línea se acorta a la derecha (right) o a la izquierda (left), respectivamente.

\begin{table}[H]
   \centering
   	  \caption{Tabla con varios tipos de líneas.}
   	  \label{tab:cmidrule}      
    \begin{tabular}{lll}
      \toprule
      Animal  & Comida & Tamaño  \\
      \midrule
      perro   & carne  & mediano \\
      \cmidrule{1-2}
      caballo & heno   & grande  \\
      \cmidrule(r){1-1}
      \cmidrule(rl){2-2}
      \cmidrule(l){3-3}
      rana    & moscas & pequeño \\
      \bottomrule
    \end{tabular}
\end{table}

Algunas veces una línea puede implicar una fuerte separación, no deseada, entre dos filas, pero en aras de una mayor claridad es posible que se desee separar esas líneas de alguna manera. En este caso se puede usar \texttt{\textbackslash addlinespace} que añadirá un pequeño espacio vertical entre ambas.

\begin{table}[H]
   \centering
   	  \caption{Tabla con espacio extra entre filas.}
   	  \label{tab:espace}      
    \begin{tabular}{cp{9cm}}
      \toprule
      Animal & Descripción \\
      \midrule
      perro  & El perro es un miembro del género Canis, el cual forma parte 
               de los cánidos derivados del lobo. \\
      \addlinespace
      gato   & El gato es una especie doméstica de pequeños mamíferos carnívoros. Es la 
               única especie domesticada de la familia de los félidos y es a menudo llamado 
    		   gato doméstico. \\
      \bottomrule
    \end{tabular}
\end{table}






\section{Expansión de celdas a filas y columnas}
El paquete \texttt{multirow} permite la expansión de celdas a varias columnas o filas empleando los comandos \texttt{multicolum} y \texttt{multirow} como se muestra en los siguientes ejemplos:

\begin{table}[H]%
	\centering
	\caption{Ejemplo de expansión en columnas}
	\label{tab:expcolumnas}
	\begin{tabular}{|l|l|} % Expansión en columnas
	\hline
	\multicolumn{2}{|c|}{Alineación} \\
	\hline
	GK & Paul Robinson \\
	LB & Lucus Radebe \\
	FW & Jamie McMaster \\
	ST & Alan Smith \\
	ST & Mark Viduka \\
	\hline
	\end{tabular}
\end{table}

\begin{table}[H]%
	\centering
	\caption{Ejemplo de expansión en filas}
	\label{tab:expfilas}
	\begin{tabular}{|l|l|l|} % Expansión en filas
	\hline
	\multicolumn{3}{|c|}{Alineación} \\
	\hline
	Guardameta & GK & Paul Robinson \\ \hline
	\multirow{4}{*}{Defensas} & LB & Lucus Radebe \\
	 & DC & Michael Duberry \\
	 & DC & Dominic Matteo \\
	 & RB & Didier Domi \\ \hline
	\multirow{3}{*}{Centrocampistas} & MC & David Batty \\
	 & MC & Eirik Bakke \\
	 & MC & Jody Morris \\ \hline
	Delanteros & FW & Jamie McMaster \\ \hline
	\multirow{2}{*}{Puntas} & ST & Alan Smith \\
	 & ST & Mark Viduka \\
	\hline
	\end{tabular}
\end{table}



\begin{table}[H]%
	\centering
	\caption{Ejemplo de expansión simultánea en filas y columnas}
	\label{tab:expsimul}
	\begin{tabular}{cc|c|c|c|c|l} % Expansión simultánea en filas y columnas
	\cline{3-6}
	& & \multicolumn{4}{|c|}{Primos} \\ \cline{3-6}
	& & 2 & 3 & 5 & 7 \\ \cline{1-6}
	\multicolumn{1}{|c|}{\multirow{2}{*}{Potencias}} &
	\multicolumn{1}{|c|}{504} & 3 & 2 & 0 & 1 &     \\ \cline{2-6}
	\multicolumn{1}{|c|}{}                        &
	\multicolumn{1}{|c|}{540} & 2 & 3 & 1 & 0 &     \\ \cline{1-6}
	\multicolumn{1}{|c|}{\multirow{2}{*}{Potencias}} &
	\multicolumn{1}{|c|}{mcd} & 2 & 2 & 0 & 0 & min \\ \cline{2-6}
	\multicolumn{1}{|c|}{}                        &
	\multicolumn{1}{|c|}{mcm} & 3 & 3 & 1 & 1 & max \\ \cline{1-6}
	\end{tabular}
\end{table}








\section{Alienación de columnas con información numérica}
Cuando se emplean tablas con datos numéricos a veces puede interesar la alineación de datos por el signo de puntuación empleado para la separación de decimales (\texttt{`,'} en español, \texttt{`.'} en inglés). Un ejemplo de esto se ilustra a continuación:

\begin{table}[H]%
	\centering
	\caption{Tabla numérica con alineación al carácter `,'}
	\label{tab:alineada}
	\begin{tabular}{c r@{,} l}
    \toprule
	Expresión con pi & \multicolumn{2}{c}{Valor} \\
	\midrule
	$\pi$                   &      3 & 14159 \\
	$\pi^{\pi}$             & 36     &    46 \\
	$(\pi^{\pi})^{\pi}$     &  80662 & 7     \\
    \bottomrule
	\end{tabular}
\end{table}

Cuando se trata de trabajar con números y unidades físicas, existen paquetes especializados en su tratamiento tipográfico adecuado. Quizá el más interesante es \texttt{siunitx} que se encarga de los aspectos mencionados para el idioma concreto de uso y las unidades empleadas del SI. Dicho paquete también añade un nuevo identificador de columna en las tablas cuando estas contienen información numérica (compara Tablas~\ref{tab:alineada} y \ref{tab:siunitx}).

\begin{table}[H]%
	\centering
	\caption{Tabla numérica con alineación al carácter `,' obtenido mediante paquete \texttt{siunitx}}
	\label{tab:siunitx}
	\begin{tabular}{c S}
    \toprule
	Expresión con pi & Valor \\
	\midrule
	$\pi$                   & 3,14159 \\
	$\pi^{\pi}$             & 36,46 \\
	$(\pi^{\pi})^{\pi}$     & 80662,7 \\
    \bottomrule
	\end{tabular}
\end{table}









\section{Tablas con mayor control del espaciado}
En el entorno tabular el control del espaciado entre columnas es demasiado <<burdo>>. Para facilitar esta labor existen paquetes como \texttt{tabularx}, cuyo uso se muestra en los ejemplos siguientes:

\begin{table}[H]
   \centering
   	\caption{Ejemplo de tabla con entorno\texttt{tabularx}}					\label{tab:tabularx1}
   \begin{tabularx}{\textwidth}% Se especifica el ancho completo de la tabla
   { |X|X|X|r| }
   \hline
   etiqueta 1 & etiqueta 2 & etiqueta 3 & etiqueta 4 \\
   \hline
   ítem 1     & ítem 2     & ítem 3     & ítem 4  \\
   \hline
	\end{tabularx}
\end{table}


\begin{table}[H]
	\newcolumntype{R}{>{\raggedleft\arraybackslash}X}%
	\centering
    \caption{Otro ejemplo de tabla ampliada}
    \label{tab:tabularx2}
	\begin{tabularx}{\textwidth}{ |l|R|l|R| }
  	\hline
   etiqueta 1 & etiqueta 2 & etiqueta 3 & etiqueta 4 \\
   \hline
   ítem 1     & ítem 2     & ítem 3     & ítem 4  \\
  	\hline
   \end{tabularx}
\end{table}

A continuación se muestran varios ejemplos en los que puedes observar que las tablas no están incluidas como objetos flotantes. Por tanto, no tienen título y no pueden ser referenciadas en el texto.

\begin{center}
\begin{tabular}{lp{2cm}}
\hline
A & B B B B B B B B B B B B B B B B B B B B B B B B\\
C & D D D D D D D\\
\hline
\end{tabular}
\end{center}

\begin{center}  
\begin{tabularx}{.5\textwidth}{lX}
\hline
A & B B B B B B B B B B B B B B B B B B B B B B B B\\
C & D D D D D D D\\
\hline
\end{tabularx}
\end{center}

\begin{center}  
\begin{tabularx}{\textwidth}{lX}
\hline
A & B B B B B B B B B B B B B B B B B B B B B B B B\\
C & D D D D D D D\\
\hline
\end{tabularx}
\end{center}

\begin{center}
\begin{tabular}{lll}
  \toprule
  Animal  & Comida & Tamano  \\
  \midrule
  perro   & carne  & mediano \\
  \cmidrule{1-2}
  caballo & heno   & grande  \\
  \cmidrule{1-1}
  \cmidrule{3-3}
  rana    & moscas & pequeño \\
  \bottomrule
\end{tabular}
\end{center}




\section{Tablas con color}
El uso del color en los documentos académicos debe realizarse de modo muy comedido y sólo cuando el mismo sea fundamental para facilitar la comprensión de los contenidos expuestos. En las tablas se puede hace uso del color si se emplea el paquete \texttt{xcolor} con la opción \texttt{table}.

\begin{table}[H]
	\centering
	\caption{Tabla con filas de colores alternados}
	\rowcolors{2}{gray!10}{gray!40}
	\begin{tabular}{lll}
    \toprule
		Col. 1  & Col. 2  & \cellcolor{red!40}Col. 3 \\
    \midrule
		par    & par    & par   \\
		impar  & impar  & impar \\
		par    & par    & par   \\
		impar  & impar  & impar \\
		par    & par    & par   \\
    \bottomrule
	\end{tabular}
\end{table}

\begin{table}[H]
	\centering
	\caption{Tabla con una columna en color}
	\begin{tabular}{ll>{\columncolor{green!20}}l}
    \toprule
		Col. 1  & Col. 2  & Col. 3 \\
    \midrule
		par    & par    & par   \\
		impar  & impar  & impar \\
		par    & par    & par   \\
    \bottomrule
	\end{tabular}
\end{table}





\section{Tablas de gran tamaño}
El paquete \texttt{graphicx} aporta comandos para realizar el escalado de objetos en un documento \LaTeX. Un caso interesante es el escalado para adaptar el tamaño de una tabla, aunque esto se debe hacer con sumo cuidado, ya que el tamaño de las fuentes empleadas en la tabla quedará afectado en el mismo factor de escala aplicado y no se corresponderá con el tamaño del texto de una tabla normal. En la tabla~\ref{tab:escalada} se muestra el efecto de escalado para ajustar su tamaño al de la página.


\begin{table}[H]
	\centering
	\caption{Tabla escalada}
	\label{tab:escalada}
	\resizebox{\textwidth}{!}{% En el escalado ajustamos sólo el ancho para mantener la proporción
		\begin{tabular}{llllllllp{1in}lp{1in}}
			\toprule
			Context   &Length   &Breadth/   &Depth   &Profile   &Pottery   &Flint   &Animal   &Stone   &Other    &C14 Dates \\
			&         &Diameter   &        &          &          &        & 
			Bones&&&\\
			\midrule
			&&&&&&&&&&\\
			\multicolumn{10}{l}{\bf Grooved Ware}&\\
			784       &---        &0.9m       &0.18m   &Sloping U &P1       &$\times$46  &  $\times$8      &&       $\times$2 bone&  2150$\pm$ 100 BC\\
			785       &---        &1.00m      &0.12    &Sloping U &P2--4    &$\times$23  &  $\times$21     & Hammerstone &---&---\\
			962       &---        &1.37m      &0.20m   &Sloping U &P5--6    &$\times$48  &  $\times$57*    & ---&     ---&1990 $\pm$ 80 BC (Layer 4) 1870 $\pm$90 BC (Layer 1)\\
			983       &0.83m      &0.73m      &0.25m   &Stepped U &---      &$\times$18  &  $\times$8      & ---& Fired clay&---\\
			&&&&&&&&&&\\
			\multicolumn{10}{l}{\bf Beaker}&\\
			552       &---        &0.68m      &0.12m   &Saucer    &P7--14   &---           & ---       & ---       &---        &---\\
			790       &---        &0.60m      &0.25m   &U         &P15      &$\times$12    & ---       & Quartzite-lump&---    &---\\
			794       &2.89m      &0.75m      &0.25m   &Irreg.    &P16      &$\times$3     & ---       & ---       &---        &---\\
			\bottomrule
	\end{tabular}}
\end{table}



Puede ocurrir que debido al tamaño de la tabla, quede mejor girada en la página para obtener una orientación apaisada. Este efecto se puede conseguir de un modo sencillo con el paquete \texttt{rotating}. Para conseguir el efecto mencionado se empleará el entorno \texttt{sidewaystable}, como se muestra en el caso de la tabla~\ref{tab:girada}.
\begin{sidewaystable}
	\centering
	\caption[Ejemplo de tabla girada]{Tabla girada en una página}\label{tab:girada}
	\begin{tabular}{llllllllp{1in}lp{1in}}
		\toprule
		Context   &Length   &Breadth/   &Depth   &Profile   &Pottery   &Flint   &Animal   &Stone   &Other    &C14 Dates \\
		&         &Diameter   &        &          &          &        & 
		Bones&&&\\
		\midrule
		&&&&&&&&&&\\
		\multicolumn{10}{l}{\bf Grooved Ware}&\\
		784       &---        &0.9m       &0.18m   &Sloping U &P1       &$\times$46  &  $\times$8      &&       $\times$2 bone&  2150$\pm$ 100 BC\\
		785       &---        &1.00m      &0.12    &Sloping U &P2--4    &$\times$23  &  $\times$21     & Hammerstone &---&---\\
		962       &---        &1.37m      &0.20m   &Sloping U &P5--6    &$\times$48  &  $\times$57*    & ---&     ---&1990 $\pm$ 80 BC (Layer 4) 1870 $\pm$90 BC (Layer 1)\\
		983       &0.83m      &0.73m      &0.25m   &Stepped U &---      &$\times$18  &  $\times$8      & ---& Fired clay&---\\
		&&&&&&&&&&\\
		\multicolumn{10}{l}{\bf Beaker}&\\
		552       &---        &0.68m      &0.12m   &Saucer    &P7--14   &---           & ---       & ---       &---        &---\\
		790       &---        &0.60m      &0.25m   &U         &P15      &$\times$12    & ---       & Quartzite-lump&---    &---\\
		794       &2.89m      &0.75m      &0.25m   &Irreg.    &P16      &$\times$3     & ---       & ---       &---        &---\\
		\bottomrule
	\end{tabular}
\end{sidewaystable}

\begin{landscape}
Un efecto similar se consigue mediante el paquete pdflscape, pero en este caso es toda la página la que se configura en modo apaisado, como vemos en la tabla~\ref{tab:apaisada}. 

\begin{table}[H]
	\centering
	\caption[Ejemplo de tabla apaisada]{Tabla en una página apaisada}\label{tab:apaisada}
	\begin{tabular}{llllllllp{1in}lp{1in}}
		\toprule
		Context   &Length   &Breadth/   &Depth   &Profile   &Pottery   &Flint   &Animal   &Stone   &Other    &C14 Dates \\
		&         &Diameter   &        &          &          &        & 
		Bones&&&\\
		\midrule
		&&&&&&&&&&\\
		\multicolumn{10}{l}{\bf Grooved Ware}&\\
		784       &---        &0.9m       &0.18m   &Sloping U &P1       &$\times$46  &  $\times$8      &&       $\times$2 bone&  2150$\pm$ 100 BC\\
		785       &---        &1.00m      &0.12    &Sloping U &P2--4    &$\times$23  &  $\times$21     & Hammerstone &---&---\\
		962       &---        &1.37m      &0.20m   &Sloping U &P5--6    &$\times$48  &  $\times$57*    & ---&     ---&1990 $\pm$ 80 BC (Layer 4) 1870 $\pm$90 BC (Layer 1)\\
		983       &0.83m      &0.73m      &0.25m   &Stepped U &---      &$\times$18  &  $\times$8      & ---& Fired clay&---\\
		&&&&&&&&&&\\
		\multicolumn{10}{l}{\bf Beaker}&\\
		552       &---        &0.68m      &0.12m   &Saucer    &P7--14   &---           & ---       & ---       &---        &---\\
		790       &---        &0.60m      &0.25m   &U         &P15      &$\times$12    & ---       & Quartzite-lump&---    &---\\
		794       &2.89m      &0.75m      &0.25m   &Irreg.    &P16      &$\times$3     & ---       & ---       &---        &---\\
		\bottomrule
	\end{tabular}
\end{table}
\end{landscape}
\end{document}
